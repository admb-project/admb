% $Id$
%
% Author: David Fournier
% Copyright (c) 2008 Regents of the University of California
%

%\documentclass{book}
%\usepackage{hyperref}
%\makeindex
%\begin{document}
%\input docmacpdf.tex
%\def\pageno{\thepage}
%\def\aone{\hbox{$a$\kern -1pt $1$}}
%\def\Pone{\hbox{$P$\kern -1pt $1$}}
%\chapno=5
%\chapter{Multivariate Probit Models}
\section{Description of the Mprobit discrete choice problem}
Assume that there are $L$ categories to choose from. There are $N$
observations where an observation consists of a choice from one of the
$L$ categories as well as a set of covariates $X_i$. 
Let $i$ index the observations and 
$j$ index the categories. It is assumed that
the observed choices are determined by the following mechanism.
For each $i$  There is an $L$ dimensional
 multivariate normal random variable 
$Y_i$ given by
$$Y_i=\phi(X_i,\beta) +\varepsilon_i$$ 
where the $\varepsilon_i$ are multivariate normal with mean vector $0$
and unstated (for the moment) covariance matrix.
The random vectors $Y_i$ are not observed. It is assumed that
they affect the observations in the following way.
$Y_{ij}$ is the largest component of $Y_i$ if and only if
the $j'$th category is chosen for observation i.

Let the $L-1$ component random vectors $Z_i$ be defined by
$$Z_{i,j-1}=Y_{ij}-Y_{i1} \qquad\hbox{\rm for}\qquad 2\le j\le L$$
A multivariate normal structure on the $Y_i$ induces one on the
$Z_i$ but different covariance structures on $Y_i$ will induce the
same covariance structure on the $Z_i$.  This means that
trying to parameterize convariance matrices on the $Z_i$
via ones on the $Y_i$ will lead to identifiablility problems
on the parameters. In the present contex it is more convenient
to simply parameterize $\Sigma=\sigma_{jk}$,
the $L-1 \times L-1$ covariance matrix,
 directly
on the $Z_i$. Also since the observations are scale invariant
(i.e. do not change if all the components of the $Z_i$ are multiplied
by the same positive number) we can assume that  $\sigma_{11}=1$.
The mean vector $\psi_i(X_i,\beta)$ of the $Z_{ij}$ is give by
$$\psi_{ij}(X_i,\beta)=\phi_{i,j+1}(X_i,\beta)-\phi_{i1}(X_i,\beta)
 \quad \hbox{\rm for} \quad 2\le j\le J$$

\section{Form of the Likelihood}

The probability that the first category is chosen is give by 
$$Pr\{Y_1\ge Y_j\} \qquad \hbox{\rm for} \qquad 2\le j\le J$$
which is the same as 
$$Pr\{Z_j\le 0\} \qquad \hbox{\rm for} \qquad 1\le j\le J-1$$

The probability that the $j$'th  category is chosen where 
$2\le j\le J$ is  give by 
$$Pr\{V_k\le 0\} \qquad \hbox{\rm for} \qquad 1\le k\le J-1, 
  k\ne i$$
where $V_k=Z_j-Z_k$.

For $1\le k\le J-2$
\end{document}
