
\documentclass[10pt]{article}
%%%%%%%%%%%%%%%%%%%%%%%%%%%%%%%%%%%%%%%%%%%%%%%%%%%%%%%%%%%%%%%%%%%%%%%%%%%%%%%%%%%%%%%%%%%%%%%%%%%%%%%%%%%%%%%%%%%%%%%%%%%%
\usepackage{harvard}

%TCIDATA{OutputFilter=LATEX.DLL}
%TCIDATA{Created=Friday, September 19, 2003 12:48:26}
%TCIDATA{LastRevised=Wednesday, October 15, 2003 15:23:56}
%TCIDATA{<META NAME="GraphicsSave" CONTENT="32">}
%TCIDATA{<META NAME="DocumentShell" CONTENT="General\Blank Document">}
%TCIDATA{Language=American English}
%TCIDATA{CSTFile=Math with theorems suppressed.cst}
%TCIDATA{PageSetup=72,72,72,72,0}
%TCIDATA{AllPages=
%F=36,\PARA{038<p type="texpara" tag="Body Text" >\hfill \thepage}
%}


\input{tcilatex}

\begin{document}

\title{Weibull regression in censored survival analysis; The kidney data}
\author{}
\maketitle

A typical setting in survival analysis is that we observe the time point $t$
at which the death of a patient occurs. Patients may leave the study (for
some reason) before they die. In this case the survival time is said to be
censored, and $t$ refers to the time point when the patient left the study.
The indicator variable $\delta $ is used to indicate whether $t$ refers to
the death of the patient ($\delta =1$) or to a censoring event ($\delta =0$%
). The key quantity in modelling the probability distribution of $t$ is the
hazard function $h(t)$, which measures the instantaneous death rate at time $%
t$. We also define the cumulative hazard function $\Lambda
(t)=\int_{0}^{t}h(s)ds$, implicitly assuming that the study started at time $%
t=0$. The loglikelihood contribution from our patient is $\delta \log
(h(t))-H(t)$. A commonly used model for $h(t)$ is Cox's proportional hazard
model, in which the hazard rate for the $i$th patient is assumed to be on
the form%
\[
h_{i}(t)=h_{0}(t)\exp (\eta _{i}\mathbf{),\qquad }i=1,\ldots n.
\]%
Here, $h_{0}(t)$ is the ``baseline'' hazard function (common to all
patients) and $\eta _{i}=\mathbf{X}_{i}\mathbf{\beta }$, where $\mathbf{X}%
_{i}$ is a covariate vector specific to the $i$th patient and $\mathbf{\beta 
}$ is a vector of regression parameters. In this example we shall assume
that the baseline hazard belongs to the Weibull family: $h_{0}(t)=rt^{r-1}$
for $r>0$.

In the collection of examples following the distribution of WinBUGS this
model is used to analyse a dataset on times to kidney infection for a set of 
$n=38$ patients (`Kidney:~Weibull~regression~with~random~effects', Examples
Volume 1, WinBUGS 1.4). The dataset contains two observations per patient
(the time to first and second recurrence of infection). In addition there
are three covariates:\ `age' (continuous), `sex' (dichotomous) and `type of
disease' (categorical, four levels), and an individual spesific random
effect $u_{i}\sim N(0,\sigma ^{2})$. Thus, the linear predictor becomes%
\begin{equation}
\eta _{i}=\beta _{0}+\beta _{\text{sex}}\,\cdot \text{sex}_{i}+\beta _{\text{%
age}}\,\cdot \text{age}_{i}+\mathbf{\beta }_{\text{D}}\,\mathbf{x}_{i}+u_{i},
\label{eta_survival}
\end{equation}%
where $\mathbf{\beta }_{\text{D}}=(\beta _{1},\beta _{2},\beta _{3})$ and $%
\mathbf{x}_{i}$ is a dummy vector coding for the disease type. Parameter
estimates are shown in the table below. Posterior means as calculated by
BUGS are also shown in the table, and are similar to the maximum likelihood
estimates.

\begin{center}
\begin{tabular}{lllllllll}
& $\beta _{0}$ & $\beta _{\text{age}}$ & $\beta _{1}$ & $\beta _{2}$ & $ \beta _{3}$ & $\beta _{\text{sex}}$ & $r$ & $\sigma $ \\ \hline
ADMB-RE & -4.344 & 0.003 & 0.1208 & 0.6058 & -1.1423 & -1.8767 & 1.1624
& 0.5617 \\ 
Std. dev. & 0.872 & 0.0137 & 0.5008 & 0.5011 & 0.7729 & 0.4754 & 0.1626 & 
0.297 \\ 
BUGS & -4.6 & 0.003 & 0.1329 & 0.6444 & -1.168 & -1.938 & 1.215 & 0.6374 \\ 
Std. dev. & 0.8962 & 0.0148 & 0.5393 & 0.5301 & 0.8335 & 0.4854 & 0.1623 & 0.357%
\end{tabular}%
\end{center}

\end{document}
