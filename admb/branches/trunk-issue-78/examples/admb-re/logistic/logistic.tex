
\documentclass[10tp]{article}
%%%%%%%%%%%%%%%%%%%%%%%%%%%%%%%%%%%%%%%%%%%%%%%%%%%%%%%%%%%%%%%%%%%%%%%%%%%%%%%%%%%%%%%%%%%%%%%%%%%%%%%%%%%%%%%%%%%%%%%%%%%%
%TCIDATA{OutputFilter=LATEX.DLL}
%TCIDATA{Created=Friday, September 19, 2003 12:48:26}
%TCIDATA{LastRevised=Friday, October 17, 2003 14:18:55}
%TCIDATA{<META NAME="GraphicsSave" CONTENT="32">}
%TCIDATA{<META NAME="DocumentShell" CONTENT="General\Blank Document">}
%TCIDATA{CSTFile=Math with theorems suppressed.cst}
%TCIDATA{PageSetup=72,72,72,72,0}
%TCIDATA{AllPages=
%F=36,\PARA{038<p type="texpara" tag="Body Text" >\hfill \thepage}
%}


\newtheorem{theorem}{Theorem}
\newtheorem{acknowledgement}[theorem]{Acknowledgement}
\newtheorem{algorithm}[theorem]{Algorithm}
\newtheorem{axiom}[theorem]{Axiom}
\newtheorem{case}[theorem]{Case}
\newtheorem{claim}[theorem]{Claim}
\newtheorem{conclusion}[theorem]{Conclusion}
\newtheorem{condition}[theorem]{Condition}
\newtheorem{conjecture}[theorem]{Conjecture}
\newtheorem{corollary}[theorem]{Corollary}
\newtheorem{criterion}[theorem]{Criterion}
\newtheorem{definition}[theorem]{Definition}
\newtheorem{example}[theorem]{Example}
\newtheorem{exercise}[theorem]{Exercise}
\newtheorem{lemma}[theorem]{Lemma}
\newtheorem{notation}[theorem]{Notation}
\newtheorem{problem}[theorem]{Problem}
\newtheorem{proposition}[theorem]{Proposition}
\newtheorem{remark}[theorem]{Remark}
\newtheorem{solution}[theorem]{Solution}
\newtheorem{summary}[theorem]{Summary}
\newenvironment{proof}[1][Proof]{\textbf{#1.} }{\ \rule{0.5em}{0.5em}}
\input{tcilatex}

\begin{document}

\title{Mixed logistic regression; A comparison with BUGS}
\author{}
\maketitle

\paragraph{Model description}

Let $\mathbf{y}=(y_{1},\ldots ,y_{n})$ be a vector of dichotomous
observations ($y_{i}\in \{0,1\}$), and let $\mathbf{u}=(u_{1},\ldots ,u_{q})$
be a vector of independent random effects, each with Gaussian distribution
(expectation $0$ and variance $\sigma ^{2}$). Define the success probability 
$\pi _{i}=\Pr (y_{i}=1)$. The following relationship between $\pi _{i}$ and
explanatory variables (contained in matrices $\mathbf{X}$ and $\mathbf{Z}$)
is assumed:%
\[
\log \left( \frac{\pi _{i}}{1-\pi _{i}}\right) =\mathbf{X}_{i}\mathbf{\beta }%
+\mathbf{Z}_{i}\mathbf{u}, 
\]%
where $\mathbf{X}_{i}$ and $\mathbf{Z}_{i}$ are the $i$'th rows of the known
covariates matrices $\mathbf{X}$ ($n\times p$) and $\mathbf{Z}$ ($n\times q$%
), respectively, and $\mathbf{\beta }$ is a $p$-vector of regression
parameters. Thus, the vector of fixed-effects vector is $\mathbf{\theta }=(%
\mathbf{\beta },\log \left( \sigma \right) )$.

\bigskip

\paragraph{Results}

The goal here is to compare computation times with BUGS on a simulated data
set. For this purpose we use $n=200$, $p=5$, $q=30$, and values of the the
hyper parameters as showd in the table below (`True values'). The matrices $%
\mathbf{X}$ and $\mathbf{Z}$ were generated randomly with each element
uniformly distributed on $[-2,2]$. As start values for both AD Model Builder
and BUGS we used $\beta _{\text{init},j}=-1$ and $\sigma _{\text{init}}=4.5$%
. In BUGS we used a uniform $[-10,10]$ prior on $\beta _{j}$ and a standard
(in the BUGS literature) noninformative gamma prior on $\tau =\sigma ^{-2}$.
In AD Model Builder the parameter bounds $\beta _{j}\in \lbrack -10,10]$ and 
$\log (\sigma )\in \lbrack -5,3]$ were used in the optimization process.

\begin{center}
\begin{tabular}{lllllll}
& $\beta _{1}$ & $\beta _{2}$ & $\beta _{3}$ & $\beta _{4}$ & $\beta _{5}$ & 
$\sigma $ \\ \hline
True values & 0.0 & 0.0 & 0.0 & 0.0 & 0.0 & 0.1 \\ \hline
ADMB-RE & 0.03 & -0.07 & 0.08 & 0.08 & -0.11 & 0.17 \\ 
Std. dev. & 0.15 & 0.15 & 0.15 & 0.14 & 0.16 & 0.05 \\ 
WinBUGS & 0.03895 & -0.07871 & 0.07727 & 0.08405 & -0.1041 & 0.1862%
\end{tabular}
\end{center}

On the simulated dataset AD Model Builder used $27$ seconds to converge to
the optimum of likelihood surface. On the same dataset we first ran WinBUGS
(Version 1.4) for $5,000$ iterations. The recommended convergence dianostic
in WinBUGS is the Gelman-Rubin plot (see the help files available from the
menues in WinBUGS) which require that two Markov chains are run in paralell.
From the Gelman-Rubin plot it was clear that convergence appeared after
approximately $2,000$ iterations. The time taken by WinBUGS to perform
generate the first $2,000$ was approximately $700$ seconds.

\end{document}
