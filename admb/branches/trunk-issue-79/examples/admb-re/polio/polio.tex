
\documentclass[10tp]{article}
%%%%%%%%%%%%%%%%%%%%%%%%%%%%%%%%%%%%%%%%%%%%%%%%%%%%%%%%%%%%%%%%%%%%%%%%%%%%%%%%%%%%%%%%%%%%%%%%%%%%%%%%%%%%%%%%%%%%%%%%%%%%
\usepackage{harvard}

%TCIDATA{OutputFilter=LATEX.DLL}
%TCIDATA{Created=Friday, September 19, 2003 12:48:26}
%TCIDATA{LastRevised=Monday, October 20, 2003 11:15:39}
%TCIDATA{<META NAME="GraphicsSave" CONTENT="32">}
%TCIDATA{<META NAME="DocumentShell" CONTENT="General\Blank Document">}
%TCIDATA{CSTFile=Math with theorems suppressed.cst}
%TCIDATA{PageSetup=72,72,72,72,0}
%TCIDATA{AllPages=
%F=36,\PARA{038<p type="texpara" tag="Body Text" >\hfill \thepage}
%}


\input{tcilatex}

\begin{document}

\title{A discrete valued time series; The polio dataset}
\author{}
\maketitle

\paragraph{Model description}

\bigskip \citeasnoun{zege:1988} analyzed a time series of monthly numbers of
poliomyelitis cases during the period 1970--1983 in the US. We make
comparison to the performance of the Monte Carlo Newton-Raphson method as
reported in \citeasnoun{kuk:chen:1999}. We adopt their model formulation.

Let $y_{i}$ denote the number of polio cases in the $i$th period ($%
i=1,\ldots ,168)$. It is assumed that the distribution of $y_{i}$ is
governed by a latent stationary AR(1) process $\{u_{i}\}$ satisfying 
\[
u_{i}=\rho u_{i-1}+\varepsilon _{i}, 
\]%
where the $\varepsilon _{i}\sim N(0,\sigma ^{2})$ variables. To account for
trend and seasonality the following covariate vector is introduced 
\[
\mathbf{x}_{i}=\left( 1,\frac{i}{1000},\cos \left( \frac{2\pi }{12}i\right)
,\sin \left( \frac{2\pi }{12}i\right) ,\cos \left( \frac{2\pi }{6}i\right)
,\sin \left( \frac{2\pi }{6}i\right) \right) . 
\]%
Conditionally on the latent process $\{u_{i}\}$, the counts $y_{i}$ are
independently Poisson distributed with intensity 
\[
\lambda _{i}=\exp (\mathbf{x}_{i}^{^{\prime }}\mathbf{\beta }+u_{i}). 
\]

\paragraph{Results}

Estimates of hyper-parameters are shown in the following table.

\begin{center}
\begin{tabular}{lllllllll}
& $\beta _{1}$ & $\beta _{2}$ & $\beta _{3}$ & $\beta _{4}$ & $\beta _{5}$ & 
$\beta _{6}$ & $\rho $ & $\sigma $ \\ 
\hline
ADMB-RE & 0.242 & -3.81 & 0.162 & -0.482 & 0.413 & -0.0109 & 0.627 & 0.538
\\ 
Std. dev. & 0.27 & 2.76 & 0.15 & 0.16 & 0.13 & 0.13 & 0.19 & 0.15 \\ 
\citeasnoun{kuk:chen:1999} & 0.244 & -3.82 & 0.162 & -0.478 & 0.413 & -0.0109
& 0.665 & 0.519%
\end{tabular}
\end{center}

We note that not the standard deviation is large for several regression
parameters. The ADMB-RE estimates (which are based on the Laplace
approximation) very are very similar to the exact maximum likelihood
estimates as obtained with the method of \citeasnoun{kuk:chen:1999}.

\bibliographystyle{agsm}
\bibliography{skaug}

\end{document}
