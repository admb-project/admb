% $Id$
%
% Author: David Fournier
% Copyright (c) 2008 Regents of the University of California
%

\ADM\ has a number of options which can be invoked at the command line.
A list of current options can be displayed by typing the name of the applicationfollowed by {\tt -?}. YOu will see a display like:

\XX{command line arguments}{-ind NAME}
\XX{command line arguments}{only estimate parameters}
\beginexample
 AD Model Builder Copyright (c) 2008 Regents of the University of California
 USAGE--kalman options
 where an option consists of -option_tag followed by arguments if necessary
 -ainp NAME      change default ascii input parameter file name to NAME
 -binp NAME      change default binary input parameter file name to NAME
 -est            only do the parameter estimation
 -ind NAME       change default input data file name to NAME
 -lmn N          use limited memory quasi newton -- keep N steps
 -lprof          perform profile likelihood calculations
 -prsave         save the independent variables from the profile calculations
 -maxph N        increase the maximum phase number to N
 -mcdiag         use diagonal covariance matrix for mcmc with diagonal values 1
 -mcmc [N]       perform markov chain monte carlo with N simulations
 -mcmult N       multiplier N for mcmc default
 -mcr            resume previous mcmc
 -mcrb N         modify the covariance matrix to reduce extremely high correlation      
 -mcnoscale      don't rescale step size for mcmc depending on acceptance rate
 -mcgrope N      use probing strategy for mcmc with factor N
 -mcseed N       seed for random number generator for markov chain monte carlo
 -mccale N       rescale step size for first N evaluations
 -mcsave N       save the parameters for every N'th simulation
 -mceval         Go throught the saved mcmc values from a previous mcsave
 -mcpin NAME      Read the starting values for MCMC from the file NAME
 -crit N         set gradient magnitude convergence criterion to N
 -iprint N       print out function minimizer report every N iterations
 -maxfn N        set maximum number opf function eval's to N
 -rs             if function minmimizer can't make progress rescale and try again
 -nox            don't show vector and gradient values in function minimizer screen report
 -phase N        start minimization in phase N
 -simplex        use simplex algorithm for minimization (new test version)
 -sdonly         do delta method for std dev estimates without redoing hessian
 -ams N          set arrmblsize to n (ARRAY_MEMBLOCK_SIZE) 
 -cbs N          set CMPDIF_BUFFER_SIZE TO N 
 -mno N          set the maximum number of independent variables to N
 -gbs N          set GRADSTACK_BUFFER_SIZE TO N 
 -mdl N          set the maximum number of dvariables to N
 -? or -help     this message
\endexample
The version of \ADM\ is printed. This can be useful to determine
the version which the application was compiled with.

\XX{command line arguments}{setting the maximum number of dvariables}
\XX{command line arguments}{-mno N}
\XX{command line arguments}{-mdl N}
\XX{command line arguments}{-ind NAME}
\XX{command line arguments}{changing the input data file name}
\beginexample
 -aind NAME 
\endexample
By default the program named {\tt xxxx(.exe)} tries to read in its data
from the file {\tt xxxx.dat}. This option changes the data file to NAME.

\XX{command line arguments}{-ainp NAME}
\XX{command line arguments}{changing the input data file name}
\beginexample
 -ainp NAME 
\endexample
This option changes the file from which the inital parameter estimates
are read in when to NAME. The program expects the parameters to
be in  ASCII format with comment lines beginning with {\tt \#}.

When a program is running it produces parameter estimates in ASCII
format in files named {\tt xxxx.p01}, $\ldots$, {\tt xxxx.par}.
These files are in the proper format to be input back into the model
and permit restarts at any phase of the minimization.

\XX{command line arguments}{-binp NAME}
\XX{command line arguments}{changing the input data file name}
\beginexample
 -binp NAME 
\endexample
This option changes the file from which the inital parameter estimates
are read in when to NAME. The program expects the parameters to
be in  binary format.

When a program is running it produces parameter estimates in binary
format in files named {\tt xxxx.b01}, $\ldots$, {\tt xxxx.bar}.
These files are in the proper format to be input back into the model
and permit restarts at any phase of the minimization.

Both ASCII and binary forms of the parameter files are supplied because
they have different advantages and disadvantages. ASCII files can be
easily examkned and edited. Binary files supply parameter values 
to the limit of machine precision in a compact format.
\XX{command line arguments}{-lmn}
\XX{command line arguments}{limited memory Newton}
\beginexample
 -lmn N
\endexample
The limited memory Newton minimization option reduces the amount
of memory necessary for holdinbg the approximate Hessian inverse.
It is of use particularly in problems with a large number of
parameters (typically > 1000). For man problems it is not as efficient
per function evaluation as the default quasi-Newton method although
this is not always the case. N is the number of pas steps of information
kept for the quasi-Newton update. Typcially a value of about 5-20
is good.
\XX{command line arguments}{-likeprof}
\XX{command line arguments}{likelihood profiles}
\beginexample
 -lprof     
\endexample 
This option turns on the profile likelihood calculations. 
A variable for which profile likelihood calculations are performed
must have been declared with the {\tt likeprof\_number} in the
{\tt TPL} file.
\XX{command line arguments}{-prsave}
\beginexample
 -prsave     
\endexample 
This option causes the values of the independent variables 
for the profile likelihood points to be save in a file named
{\tt xxx.pvl} where {\tt xxx} is the name of the variable
being profiled. These values can be used later for starting
the MCMC analysis at different values which is useful for
testing the mixing of the chain with respect to that parameter.
\XX{command line arguments}{-maxph N}
\XX{command line arguments}{set maximum phase of minimization}
\beginexample
 -maxph N     
\endexample 
You may want to add extra phases to the minimization -- usually
because the standard set of phases has not converged. This will
set the number of phases to N.
\XX{command line arguments}{-mcmc}
\XX{command line arguments}{Markov Chain Monte Carlo}
\beginexample
 -mcmc [N]    
\endexample 
This option turns on the the calculation of the Markov chain Monte Carlo
rotine. By default the model will recalculate the approximate Hessian
so you may want to use the {\tt -nohess} option if you don't wish
to recalculate the Hessian. It is your responsibility to ensure that
the Hessian data in the current directory are current.
The mcmc routine will perform {\tt N} simulations.
\XX{command line arguments}{-mcr}
\XX{command line arguments}{restart Markov Chain Monte Carlo}
\beginexample
 -mcr    
\endexample 
Restart (and continue) a previous Markov chain Monte Carlo
routine. This will continue from where the previous routine
left off.
\XX{command line arguments}{-mcrb}
\beginexample
 -mcrb N  
\endexample
See discussion of this option elsewhere in the manual.

\XX{command line arguments}{-mcsave}
\XX{command line arguments}{save results from Markov Chain Monte Carlo}
\beginexample
 -mcsave N  
\endexample
For the usual MCMC routine the results from consecutive
steps of the simulation are highly corellated. If the
parameters of interest are expensive to compute it may be advantages to
only compute every N'th one. This option saves the results so that they
can be used in subsequent calculations.
\XX{command line arguments}{-mceval}
\XX{command line arguments}{use the saved results from Markov Chain Monte Carlo}
\beginexample
 -mcsave N  
\endexample
\X{mceval\_phase()}
This option will use the previusly save results from MCMC to evaluate
parameters of interest. The function {\tt mceval\_phase()} can
be useful here to only calculate the parameters during this phase.
\XX{command line arguments}{-nox}
\XX{command line arguments}{suppressing printing}
\beginexample
 -nox  
\endexample
This option suppresses the printing of the current x vector
being sampled by the function minimizer.
Printing this out can be a siginifcant overhead for models with
a large number of parameters -- also it simply irritates some users.
\XX{command line arguments}{-ams N}
\X{arrmblsize}
\beginexample
 -ams N          set arrmblsize to n (ARRAY_MEMBLOCK_SIZE) 
\endexample
This option has the same effect as setting arrmblsize in the
program code, but has the advantage that it can be done at run time.
\X{gradient\_structure::set\_CMPDIF\_BUFFER\_SIZE} 
\XX{command line arguments}{-cbs N}
\beginexample
 -cbs N          set CMPDIF_BUFFER_SIZE to n 
\endexample
This option has the same effect as using the 
\goodbreak
{\tt gradient\_structure::set\_CMPDIF\_BUFFER\_SIZE } function in the code 
program code, but has the advantage that it can be done at run time.
\X{gradient\_structure::set\_GRADSTACK\_BUFFER\_SIZE} 
\XX{command line arguments}{-gbs N}
\beginexample
 -gbs N          set GRADSTACK_BUFFER_SIZE 
\endexample
This option has the same effect as using the \goodbreak
{\tt gradient\_structure::set\_GRADSTACK\_BUFFER\_SIZE } function in the code 
program code, but has the advantage that it can be done at run time. 
{\bf Also note that the size is in bytes here whereas for the included code it is inchunks of about 36 bytes.}

