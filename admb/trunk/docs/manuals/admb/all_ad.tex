% $Id$
%
% Author: David Fournier
% Copyright (c) 2008 Regents of the University of California
%
%\tracingmacros=1
% for slides
%\documentclass[12pt,executivepaper]{book}
\documentclass[12pt]{book}
\usepackage{hyperlatex}
%\usepackage{m-pictex}
\usepackage{pictexwd}
\usepackage{makeidx}
\usepackage{xy}
\usepackage{amssymb,amsmath,bm}
%\usepackage{tocbibind}
%\usepackage{fancyhdr}
% $Id$
%
% Author: David Fournier
% Copyright (c) 2008 Regents of the University of California
%
\def\plainline#1{\hbox to \hsize{#1}}
\def\sin{\hbox{\rm sin}}
\def\cos{\hbox{\rm cos}}
% beginning of easy.tex
%\openup -1.3truept
%\openup 1.25truept
\baselineskip = 11.5pt plus 1pt
\lineskiplimit=1truept
\lineskip=2truept minus .5truept
\newdimen\spread
\newdimen\myeqwidth
\myeqwidth=\hsize
\newdimen\mydp
\newdimen\myhtt
\newdimen\mydpp
\newdimen\mywidth
\newdimen\myht
\newdimen\allht
\newdimen\dimtwofive
\newdimen\pagewidth
\newdimen\pageheight
\newdimen\ruleht
\pagewidth=\hsize \pageheight=\vsize \ruleht=.5pt
\spread=2pt
%\baselineskip = 11pt plus 1pt
%\lineskip=5pt minus 4pt
%\def\ifundefined#1{\expandafter\ifx\csname#1\endcsname\relax}
\def\bestbreak{\par\penalty-9000}
%\font\odd=cmsy18 at 18truept
%\font\bmf=cmmib10 
\font\Lrm=cmss10 at 12truept 
%%\font\LLrm=cmss10 at 12truept
\font\Lbf=cmbx10 % scaled \magstep2
\font\Lit=cmti10 % scaled \magstep2
%\font\lrm=cmss10 at 12truept 
%\font\eighteenrm=cmr18 %scaled \magstep1
%\font\lsl=cmssi10% scaled \magstep1
%\font\it=cmssi10 at12truept
%\font\lit=cmssi10 at12truept
%\font\ltt=cmtt10 at12truept
%\font\rm=cmr18 at 18true pt  %scaled \magstep1
%\font\rmtwelve=cmr10 at 12 true pt
%\font\bf=cmbx10 at 12 truept
%\font\sl=cmssi10 at12truept
\font\slsmall=cmsl9 %scaled \magstep1
%\font\it=cmssi10 at12truept
%\font\srm=cmss10 at 8.333pt
%\font\stt=cmtt8   at 8truept
\font\tenrm=cmr10 at 10truept
%\font\tenbfmi=cmmib10
%\font\sevenbfmi=cmmib10 at 7pt
%\font\ninecmmi=cmmi9 at 9truept
%\font\sevencmmi=cmmi7 at 7truept
%\font\largebss=cmssbx12
\font\rmsmall=cmr9
\font\ninerm=cmr9 
\font\fiverm=cmr5 
\font\eightrm=cmr8 at 8 truept
%\font\largett=cmtt12
\font\ttexample=cmtt9
\font\tteight=cmtt8 at 8truept
%\rm
\newfam\boldfam
%\textfont\boldfam=\tenbfmi
%\scriptfont\boldfam=\sevenbfmi
%\def\bmit{\fam\boldfam\tenbfmi}

%\newfam\smallfam
%\textfont\smallfam=\ninecmmi
%\scriptfont\smallfam=\sevencmmi
%\def\smit{\fam\smallfam\ninecmmi}
\def\supereject{\newpage}


%\def\_{\hbox{\kern .04em\vbox{\kern .04em\hrule width.3em\kern -.04em}
 %                     \kern -.45em}} 


%\def\bs{$\backslash$}
\def\bs{\char92}

\def\sloppy{\tolerance 2000 \hfuzz .5pt \vfuzz .5pt}
\def\fussy{\tolerance 200 \hfuzz .1pt \vfuzz .1pt}

\let\3=\ss
\tolerance=10000

\def\makeheadline{\vbox to 0pt{\vskip-.5truein
  \plainline{\vbox to 8.5pt{}\the\headline}\vss}\nointerlineskip}
\def\makefootline{\baselineskip=.5truein \plainline{\footline}} % redefined below
\def\noheadlines{\def\makeheadline{}
    \advance\voffset by -.5truein \advance\vsize by .5truein}
\def\nofootlines{\def\makefootline{}\advance\vsize by .5truein}

\def\nopageno{\def\footline{}}

%\def\startpageno{
%\def\footline{ {\rm
%  \ifnum\pageno<0 
%    \hfil\romannumeral -\pageno\hfil
%  \else
%    \ifodd\pageno
%       \ifnum\chapno>0 \hfil\number\chapno -\number\pageno
%       \else\hfil\uppercase\expandafter{\romannumeral\appendixno}-\number\pageno \fi
%    \else
%       \ifnum\chapno>0 \number\chapno -\number\pageno\hfil
%       \else\uppercase\expandafter{\romannumeral\appendixno}-\number\pageno\hfil \fi
%
%   \fi
%  \fi
%  } }
%
%}%startpageno
%


\def\advancepageno{\ifnum\pageno<0 \global\advance\pageno by -1
  \else\global\advance\pageno by 1 \fi}

%\newbox\thecompany
%\global\setbox\thecompany=\hbox{\Lrm {\odd O}\kern .1em TTER {\odd S}OFTWARE}


\parindent=2em
\parskip=.5\baselineskip plus .5\baselineskip minus.25\baselineskip
%\parskip= 6truept plus 2truept minus 2truept %\medskipamount

\def\today{\hbox{\number\day.~\number\month.~\number\year}}
\def\date{\hbox{\number\year-\number\month-\number\day}}

\def\Narrower{\advance\leftskip by \displayindent
	     \advance\rightskip by \displayindent}
\def\singlespacing{\baselineskip=\normalbaselineskip}
\def\halfspacing{\baselineskip=1.3\normalbaselineskip}
\def\doublespacing{\baselineskip=1.6\normalbaselineskip}

\def\condbreak#1{\vskip 0pt plus #1\penalty-500\vskip 0pt plus-#1}

\def\center{\vskip\parskip\begingroup
	    \parindent=0pt \parskip=0pt plus 1pt
	    \rightskip=0pt plus 1fill \leftskip=0pt plus 1fill
	    \obeylines}
\def\endcenter{\endgroup}
\def\flushleft{\vskip\parskip\begingroup
	       \parindent=0pt \parskip=0pt plus 1pt
	       \rightskip=0pt plus 1fill \leftskip=0pt
	       \obeylines}
\def\endflushleft{\endgroup}
\def\flushright{\vskip\parskip\begingroup
		\parindent=0pt \parskip=0pt plus 1pt
		\rightskip=0pt \leftskip=0pt plus 1fill
		\obeylines}
\def\endflushright{\endgroup}

\def\itemize#1{\par\begingroup\parindent=#1}
\def\litem#1{\par\hang\noindent\hbox to \parindent{#1\hfil}}
\def\ttitem#1{\litem{{\catcode`\_=\other \tt #1}\hfill --~}}
\def\ttt{\catcode`\_=\other \tt} % from mixmac
\def\ritem#1{\par\hang\noindent\hbox to \parindent{\hfil#1\enspace}}
\def\enditemize{\par\endgroup\par}
\def\bitem{\ritem{$\bullet$}}

\def\disbox{\vskip\parskip \hbox to\hsize\bgroup
	    \hskip\displayindent \vbox\bgroup
	    \advance\hsize by-2\displayindent \plainline{}
	    \prevdepth=0pt \vskip-\parskip }
\def\enddisbox{\egroup\hss\egroup}

\def\example{\disbox \flushleft\tt}
\def\endexample{\endflushleft\enddisbox} % note this is redefined below

\def\framebox{\hrule height.6pt \hbox to\hsize\bgroup
	      \vrule width.6pt \hskip 9.5pt
	      \vbox\bgroup
	      \advance\hsize by-20.2pt \plainline{}
	      \vskip 9.5pt \nointerlineskip}
\def\endframebox{\vskip 9.5pt \egroup \hss \vrule width.6pt
		 \egroup \hrule height.6pt}

% This is the end of easy.tex

\newcount\secno \secno=0
\newcount\subsecno \subsecno=0
\newcount\subsubsecno \subsubsecno=0
\newcount\appendixno \appendixno=0
\newcount\mychapno \mychapno=0
\newcount\chapnonext \chapnonext=0
\newcount\chappageno \chappageno=0
%\newcount\allpageno \allpageno=0

\newwrite\newindex
\immediate\openout\newindex=xxx.sed % file for table of contents

  \def\myrulea{\hrule height4pt depth-3pt}
  \def\myrulefill{\leaders\myrulea\hfill}

\def\exp{{\,\hbox{\rm exp}}}
\def\det{{\,\hbox{\rm det}}}
\def\log{{\,\hbox{\rm log}}}
\def\ln{{\,\hbox{\rm ln}}}
%\scriptfont0=\sevenrm
\def\smax{{\hbox{\sevenrm max}}}
\def\smin{{\hbox{\sevenrm min}}}
\def\min{\mathop{\,\hbox{\rm min}}}
\def\max{\mathop{\,\hbox{\rm max}}}
\def\lia{\xi_{i\alpha}}
\def\qia{Q_{i\alpha}}
\def\qija{q_{ij\alpha}}
\def\cia{\psi_{i\alpha}}
\def\ta{\tau_\alpha}
\def\sja{\sigma_{j\alpha}}
\def\Qia{Q_{i\alpha}}
\def\pja{p_{j\alpha}}
\def\mja{\mu_{j\alpha}}
\def\MNJ{m_N}
\def\month{m}
\def\MONE{m_1}
\def\proda{\prod_{\alpha=1}^{N_A}}
\def\prodi{\prod_{i=1}^{N_I}}
\def\suma{\sum_{\alpha=1}^{N_A}}
\def\sumi{\sum_{i=1}^{N_I}}
\def\gt{>}
\def\sigd{\sigma_{\hbox{$\scriptscriptstyle D$}}}
\def\myrule{\hrule height 5truept width 3truept depth 3truept}
\long\def\switchinsert#1 #2{\parindent=40pt\item{\myrule switch #1}
  \ #2\penalty 10000\myrule\par}

\def\switchmsg#1#2{\smallskip\
\vbox height 5pt width 2pt\quad switch #1 \ \ #2\quad$\bullet\,\bullet$ \smallskip}

\def\mytextindent#1{\vrule height 8pt width 3pt depth 1pt\kern 5pt
{\bf Switch #1}\quad\ignorespaces}
\def\myitem{\par\hang\mytextindent}

\def\mf{MULTIFAN}

\def\mystrut{\vrule height 8.0truept depth 3.0 truept width 0pt}

\def\mykern{\kern 1.25truept}

\def\myhrule{\hrule height .6truept}

\def\myvrule{\vrule width .6truept}

\def\keybox#1{\lower 4.5truept\hbox{\vbox{\myhrule\hbox{\myvrule\mykern
  \vbox{\mykern\hbox{\mystrut\tt #1}\mykern}\mykern\myvrule}\myhrule}}}
\def\bigkey#1{\keybox{$\>#1\>$}}

  \def\ts{\thinspace}
\def\mNJ{m_{\lower.4truept \hbox{$\scriptstyle N_J$}}}
\def\mone{m_{\lower.4truept \hbox{$\scriptstyle 1$}}}

\def\mmone{$\mone$}
\def\mmNJ{$\mNJ$}
 \def\ipar{switch}


% john's additions to docmac since 30 aug

\def\menusymbol{        \hbox to .6em{\hrulefill}\kern-.6em
  \raise.25\baselineskip\hbox to .6em{\hrulefill}\kern-.6em
  \raise.50\baselineskip\hbox to .6em{\hrulefill}
   }

\def\jpar{Control Flag}

\def\enter{\keybox{Enter}}
\def\esc{\keybox{Esc}}
\def\alt{\keybox{Alt}}
\def\ctrl{\keybox{Ctrl}}
\def\pgdn{\keybox{Page Down}}
\def\pgup{\keybox{Page Up}}
\def\fone{\keybox{F1}}
\def\prnscrn{\ctrl\ \keybox{Print~Screen}}
\def\larrow{\keybox{$\leftarrow$}}
\def\rarrow{\keybox{$\rightarrow$}}
\def\dnarrow{\keybox{$\>\downarrow\>$}}
\def\uarrow{\keybox{$\>\uparrow\>$}}
\def\lfa{{length frequency analysis}}
\def\lf{{length frequency}}
\def\Lfa{{Length frequency analysis}}

%\def\uncatcodespecials{\def\do##1{\catcode`##1=12 }\dospecials}
%\def\set{\stt \overfullrule=0pt \hyphenpenalty=10000 \exhyphenpenalty=10000
  %\baselineskip = 10pt plus 2pt\lineskip=1pt\obeylines \parskip=0pt
  %\uncatcodespecials \obeyspaces}
%{\obeyspaces\global\let =\ } %

% john's additions since 2 sept

\def\ctrlf{\ctrl\ \keybox{F}}
\def\ctrlg{\ctrl\ \keybox{G}}
\def\vB{von Bertalanffy}
\def\vBK{\vB\ $K$}
%\def\mone{$m_1$}
%\def\mNJ{$m_{N_J}$}

\newwrite\shitout
\newwrite\indx
\immediate\openout\indx=index % file for table of index
\newcount\kludgenumber

%\def\toindex{{\let\the=0
%\edef\next{\ifnum\chapno>0\write\indx{\string\entry\string{\indextext\string}
%\string{\number\kludgenumber\string}
%\string{\indextext\string}}
%\else\write\indx{\string\entry\string{\indextext\string}\string
%\uppercase\expandafter{\romannumeral\appendixno} -\number\pageno}
%\string{\indextext\string}\fi}\next}}

%\def\toindex{{\let\the=0
%\edef\next{\ifnum\chapno>0\write\indx{\string\entry\string{\indextext\string}
%\string{\number\kludgenumber\string}
%\string{\indextext\string}}
%\else\write\indx{\string\entry\string{\indextext\string}\string
%\uppercase\expandafter{\romannumeral\appendixno} -\number\pageno}
%\string{\indextext\string}\fi}\next}}

%\def\tosubindex{{\let\the=0
%\edef\next{\ifnum\chapno>0\write\indx{\string\entry
%\string{\subindextextone\char9\subindextexttwo\string}
%\string{\number\kludgenumber\string}
%\string{\subindextextone\string} \string{\subindextexttwo\string}}  
%\else\write\indx{\string\entry
%\string{\subindextextone\char9\subindextexttwo\string}
%\string\uppercase\expandafter{\romannumeral\appendixno}-\number\pageno}
%\string{\subindextextone\string}\string{\subindextexttwo\string}\fi}
%\next}}        

\def\INDEX#1{\kludgenumber=\number\chapno\multiply\kludgenumber by 50\advance\kludgenumber by \number\pageno\xdef\indextext{#1}\toindex}

\def\SUBINDEX#1#2{\kludgenumber=\number\chapno\multiply\kludgenumber by 50\advance\kludgenumber by \number\pageno
\xdef\subindextextone{#1}
\xdef\subindextexttwo{#2}\tosubindex}
\newwrite\toc
\immediate\openout\toc=contents % file for table of contents

\def\X#1{\index{#1}}
\def\XX#1#2{\index{#1!#2}}

%\def\X#1{\INDEX{#1}}
%\def\XX#1#2{\SUBINDEX{#1}{#2}}

\def\tocline{{\let\the=0
  \edef\next{\ifnum\chapno>0\write\toc{\text\string\dotfill\ \number\chapno -\number\pageno}
             \else\write\toc{\text\string\dotfill\ \uppercase\expandafter{\romannumeral\appendixno} -\number\pageno}\fi}
  \next}
   }        

\def\toclinea{{\let\the=0
  \edef\next{\write\toc{\text\string\hfill\ }}
  \next}} 

 %
 %
 %\def\chapter#1{\secno=0 \pageno=1 % \advance\chapno by 1 
 %  \xdef\text{Chapter\ \number\chapno\ \ #1}
 %  \par {\Lbf
 %  \line{\text\ \ \myrulefill} }
 %  %\write\toc{\text\string\hfill} %added this line
 %  \toclinea
 %  \kern -1 true pt
 %  \line{\myrulefill}
 %  \medskip
 %  \par
 %
 %  }
 %


\def\endchapter{%\showthe\pageno
                \ifodd\pageno \vfill\supereject\plainline{}\vfill\eject
            \else
                 \vfill\supereject
                  \fi
  }

% \def\section#1{\advance\secno by 1 \subsecno=0
%   \goodbreak\bigbreak
%   \xdef\text{\number\chapno .\number\secno.\ #1}
%   {\noindent\bf \text}\nobreak \par
%   \tocline
%    }
% \def\endsection{\goodbreak\bigskip}
% 
% \def\subsection#1{\advance\subsecno by 1 \subsubsecno=0
%   \xdef\text{\number\chapno .\number\secno .\number\subsecno.\ #1}
%   \par {\lrm \text} \par 
%   \tocline
%   }
% \def\endsubsection{\goodbreak\medskip}
% 
% \def\subsubsection#1{\advance\subsubsecno by 1
%   \xdef\text{\number\chapno .\number\secno .\number\subsecno .\number\subsubsecno.\ #1}
%   \par {\rm \text}\par
%   \tocline
%   }
% \def\endsubsubsection{\goodbreak\smallskip}
% 
% \def\appendix#1{\advance \appendixno by 1 \pageno=1\chapno=0\secno=0
%   \xdef\text{Appendix\ \uppercase\expandafter{\romannumeral\appendixno}.\ #1}
%   \goodbreak
%   \par {\lrm \text} \par
%   \tocline
%   }
% \def\endappendix{\vfill\supereject}
% 
% dave's new stuff

\def\otherbox#1{\lower 4.5truept\hbox{\vbox{\myhrule\hbox{\myvrule\mykern
  \vbox{\mykern\hbox{\tt#1}\mykern}\mykern\myvrule}\myhrule}}}

\def\menubox#1{\otherbox{\simple{#1}}}

\def\simple#1{\hbox{\vbox{\myhrule\hbox{\myvrule\mykern
  \vbox{\mykern\hbox{\mystrut\tt#1}\mykern}\mykern\myvrule}\myhrule}}}


\def\llf{log-likelihood function}




\def\adtypes{{\tt dvariable, dvar\_array, dvar\_matrix, dvector,}
and {\tt dmatrix}}
\def\nr{{Newton-Raphson}}
\def\question#1{\bigbreak\noindent{\bf #1?}\medskip}
\def\cplus{C\raise1pt\hbox{\rmsmall++}}
\def\ts{\thinspace}
\def\dec{dec}
\def\AD{AUTODIF}
\def\classdec#1{\goodbreak\bigbreak\noindent{\largett #1}\medskip}
\def\db{{$\backslash\backslash$}}
\def\n{\char92 n}
%\def\n{{$\backslash$n}}
\parskip=.5\baselineskip plus .5\baselineskip minus.25\baselineskip

\chardef\other=12
\def\uncatcodespecials{
  \catcode`\{=\other \catcode`\}=\other \catcode`\&=\other
  \catcode`\_=\other \catcode`\#=\other
  } 

\def\set{\ttexample \overfullrule=0pt \hyphenpenalty=10000 
  \exhyphenpenalty=10000\openup-1truept \openup -\spread
  \def\par{\leavevmode\endgraf} 
  %\obeylines \parskip=0pt \parindent=0pt 
  \obeylines \parskip=0pt \advance\leftskip by .25truein \parindent=0pt
  \openup-1pt
  \uncatcodespecials \obeyspaces}

{\obeyspaces\global\let =\ } %

%\def\beginexample{\smallskip\begingroup\catcode`\%=11\set}
\def\beginexample{\par\begingroup\catcode`\%=11\set}
\def\beginexamplea{\par\vskip -5truept\begingroup\set}
\def\endexample{\endgroup\smallbreak}
\def\beginexampledf{\par\begingroup\set}
\def\endexampledf{\endgroup\par\vskip-10pt}

\def\listing#1{\smallskip\begingroup\set\input#1\endgroup\smallbreak}

\def\logrob#1#2#3{-#1\log(#2)+\sum_{i=1}^#1 
    \log\bigg[.95\exp\Big\{ {-{\big(#3\big)}^2\over 2{#2}^2} \Big\} +
    .05\Big\{1+{{\big(#3\big)}^4\over{#2}^4}\Big\}^{-1}\bigg]} 


\def\opensquiggle{\begingroup\catcode`\{=1{\catcode`\{=12\endgroup} 
\def\closesquiggle{\catcode`\}=2}\catcode`\}=12} 
%\def\onepageout#1{\shipout\vbox{#1}\advancepageno}

\def\onepageout#1{\shipout\vbox{
    #1  \ifvoid\footins\else
     \vskip\skip\footins \kern-3pt
     \hrule height\ruleht width\pagewidth \kern-\ruleht \kern3pt
     \unvbox\footins\fi \makefootline}\advancepageno}


% see page 417 of the TexBook
\newbox\partialpage
%\def\begindoublecolumns{\begingroup
%  \output={\global\setbox\partialpage=\vbox{\unvbox255\bigskip}}\eject
% \output={\doublecolumnout} \hsize=3.1truein \vsize=18truein}
%  \output={\doublecolumnout} \hsize=3.1truein \vsize=17truein}

%\def\enddoublecolumns{\output={\balancecolumns}\eject
%  \endgroup \pagegoal=\vsize}

%\def\doublecolumnout{\splittopskip=\topskip \splitmaxdepth=\maxdepth

%\def\doublecolumnout{\splittopskip=1pt \splitmaxdepth=\maxdepth
%  \dimtwofive=8.5truein \advance\dimtwofive by-\ht\partialpage
%  \dimtwofive=9truein \advance\dimtwofive by-\ht\partialpage
%  \setbox0=\vsplit255 to \dimtwofive \setbox2=\vsplit255 to\dimtwofive
%  \onepageout\pagesofar \unvbox255 \penalty\outputpenalty}
%\def\pagesofar{\unvbox\partialpage
%  \wd0=\hsize \wd2=\hsize \hbox to\pagewidth{\box0\hfil\box2}}
%\def\balancecolumns{\setbox0=\vbox{\unvbox255} \dimtwofive=\ht0
%  \advance\dimtwofive by\topskip \advance\dimtwofive by-\baselineskip

 % \divide\dimtwofive by2 \splittopskip=\topskip

%  \divide\dimtwofive by2 \splittopskip=1pt
%  {\vbadness=10000 \loop \global\setbox3=\copy0
%    \global\setbox1=\vsplit3 to\dimtwofive
%    \ifdim\ht3>\dimtwofive \global\advance\dimtwofive by1pt \repeat}
%  \setbox0=\vbox to\dimtwofive{\unvbox1}\setbox2=\vbox to\dimtwofive{\unvbox3}
%  \pagesofar }


%\def\makeheadline{\vbox to 0pt{\vskip-.5truein
%  \plainline{\vbox to 8.5pt{}\the\headline}\vss}\nointerlineskip}

%\def\makefootline{\baselineskip=.5truein \hbox to \pagewidth{\footline}}
%\def\makefootline{\vskip 0.75truein \hbox to \pagewidth{\footline}}

%\def\noheadlines{\def\makeheadline{}
%    \advance\voffset by -.5truein \advance\vsize by .5truein}

%\def\nofootlines{\def\makefootline{}\advance\vsize by .5truein}

\def\footline{ {\rm
  \ifnum\pageno<0
   % \hfil\romannumeral -\pageno\hfil
  \else
    \ifodd\pageno
       \ifnum\pageno>0
       {\ninerm Copyright\ \copyright\ 2008 Regents of the University of California}
                                   \hfill\number\chapno \ -- \number\pageno\fi
    \else
       \ifnum\pageno>0 
           \number\chapno \ -- \number\pageno\hfill
        %  {\ninerm {\slsmall AUTODIF}\ \ User's Manual}
          {\ninerm {\slsmall AD Model Builder}}
                                                                 \fi
    \fi
   \fi
  } }

\def\mychapter#1{\mark{\number\mychapno} \chapnonext=\mychapno
     \message{mychapno = \number\mychapno}
  \global\advance\chapnonext by 1
     \message{chapnonext = \number\chapnonext}
  \chapter{#1} \mark{\number\chapnonext}}

%\def\mysection#1{ \section{#1} }


\def\myover#1#2{{\displaystyle #1\over \displaystyle #2}}


\def\hrefname#1#2{{
  \setbox0\hbox{#2 }
\myht=\ht0
\myhtt=\ht0
\mydp=\dp0
\mydpp=\dp0
\mywidth=\wd0
\advance \myht by 4pt
\advance \myhtt by 3pt
\advance \mydp by 4pt
\advance \mydpp by 3pt
\advance \mywidth by 2pt
\hbox{\vrule height\myht depth\mydp width 0pt
% this has been changed
%\pdfannotlink height \myhtt depth \mydpp attr {/C [0.9 0 0.0] /Border [0 0 1] } goto name {page.#1}
\pdfstartlink height \myhtt depth \mydpp attr {/C [0.9 0 0.0] /Border [0 0 1] } goto name {page.#1}
 \ #2

\pdfendlink}}}


\makeatletter

\def\@startindexsection#1#2#3#4#5#6{%
  \if@noskipsec \leavevmode \fi
  \par
  %\@tempskipa #4\relax
  \@afterindenttrue
  \ifdim \@tempskipa <\z@
    \@tempskipa -\@tempskipa \@afterindentfalse
  \fi
  \if@nobreak
    \everypar{}%
  \else
    \addpenalty\@secpenalty\addvspace\@tempskipa
  \fi
  \@ifstar
    {\@indexssect{#3}{#4}{#5}{#6}}% !!!!
    {\@dblarg{\@sect{#1}{#2}{#3}{#4}{#5}{#6}}}}

\def\@sect#1#2#3#4#5#6[#7]#8{%
  \ifnum #2>\c@secnumdepth
    \let\@svsec\@empty
  \else
    \refstepcounter{#1}%
    %\protected@edef\@svsec{\@seccntformat{#1}\relax}%
    %\protected@edef\@svsec{{#1}\relax}%
    \let\@svsec\@empty
  \fi
  \@tempskipa #5\relax
  \ifdim \@tempskipa>\z@
    \begingroup
      %#6 {%
      %  \@hangfrom{\hskip #3\relax\@svsec}%
      %    \interlinepenalty \@M #8\@@par}%
    \endgroup
    \csname #1mark\endcsname{#7}%
    \addcontentsline{toc}{#1}{%
      \ifnum #2>\c@secnumdepth \else
        \protect\numberline{\csname the#1\endcsname}%
      \fi
      #7}%
  \else
    \def\@svsechd{%
      #6{\hskip #3\relax
      \@svsec #8}%
      \csname #1mark\endcsname{#7}%
      \addcontentsline{toc}{#1}{%
        \ifnum #2>\c@secnumdepth \else
          \protect\numberline{\csname the#1\endcsname}%
        \fi
        #7}}%
  \fi
  \@xsect{#5}}

\def\@indexssect#1#2#3#4#5{%
    \@tempskipa #3\relax
    \ifdim \@tempskipa>\z@
      \begingroup
        #4{%
          \@hangfrom{\hskip #1}%
            \interlinepenalty \@M #5\@@par}%
      \endgroup
    \else
      \def\@svsechd{#4{\hskip #1\relax #5}}%
    \fi
  \@xsect{#3}}

\def\@indexsect#1#2#3#4#5#6[#7]#8{%
  \ifnum #2>\c@secnumdepth
    \let\@svsec\@empty
  \else
    \refstepcounter{#1}%
    %\protected@edef\@svsec{\@seccntformat{#1}\relax}%
    \protected@edef\@svsec{{#1}\relax}%
  \fi
  \@tempskipa #5\relax
  \ifdim \@tempskipa>\z@
    \begingroup
      #6{%
        \@hangfrom{\hskip #3\relax\@svsec}%
          \interlinepenalty \@M #8\@@par}%
    \endgroup
    \csname #1mark\endcsname{#7}%
    \addcontentsline{toc}{#1}{%
      \ifnum #2>\c@secnumdepth \else
        \protect\numberline{\csname the#1\endcsname}%
      \fi
      #7}%
  \else
    \def\@svsechd{%
      #6{\hskip #3\relax
      \@svsec #8}%
      \csname #1mark\endcsname{#7}%
      \addcontentsline{toc}{#1}{%
        \ifnum #2>\c@secnumdepth \else
          \protect\numberline{\csname the#1\endcsname}%
        \fi
        #7}}%
  \fi
  \@xsect{#5}}
\makeatother


%\advance\myeqwidth by -.1in

\def\myfrac#1#2{{\displaystyle #1\over \displaystyle #2}}
\hfuzz=5truept
\hoffset=0.1truein
\hsize=6.5truein
\vsize=8.5truein
%*******************************************************
%*******************************************************
%:*******************************************************
%*******************************************************

\input pdfcolor
%\usepackage{chappg}
%\usepackage{graphics}
\usepackage{hyperref}
\makeindex
\newcount\tm
\newcount\bm
\newcount\tom
\newcount\bom
\newtoks\myname
\newcounter{chapterpg}[chapter]
\pagestyle{empty}
\pagestyle{myheadings}
\makeatletter
\def\ps@yyy{
\def\@oddhead{}
\def\@evenhead{}
\def\@evenfoot{}
\def\@oddfoot{}
}

\def\mysection#1{\section{#1}{\bigbf \medbreak\noindent\number\c@chapter.\number\c@section\ \ #1\medbreak}}
%\def\mysection#1{\section{#1}{\bigbf \medbreak\noindent #1\medbreak}}
%\def\mysection#1{\section{#1}{\bigbf \medbreak\noindent\number\chapter\ #1\medskip}}
%\input adtitle.tex
\def\ps@xxx{
\def\@oddhead{\message{Here DE}}
\def\@evenhead{\message{Here FE}}
\def\@oddfoot{
\plainline{ 
  \addtocounter{chapterpg}{1}
  \ifnum \number\c@chapter > 0
    \message{c@chapter = \number\c@chapter}
     \makebox[\textwidth]{
       {\ninerm admb-project.org}
      \hfill
     \arabic{chapter}-\arabic{chapterpg}\quad}
   \else
    \makebox[\textwidth]{\roman{chapterpg}}
   \fi
    \expandafter\write\newindex {%
      \thepage-\arabic{chapter}-\arabic{chapterpg}
     }
  %\makebox[\textwidth]{\arabic{page}}
 }} 
\def\@evenfoot{
\plainline{ 
  \addtocounter{chapterpg}{1}
  \ifnum \number\c@chapter > 0
    \message{c@chapter = \number\c@chapter}
       \makebox[\textwidth]{
       \arabic{chapter}-\arabic{chapterpg}\hfill
          {\ninerm {\slsmall AD Model Builder}}\quad}
   \else
    \makebox[\textwidth]{\roman{chapterpg}}
   \fi
    \expandafter\write\newindex {%
      \thepage-\arabic{chapter}-\arabic{chapterpg}
     }
  %\makebox[\textwidth]{\arabic{page}}
 }} 
%   \if \topmark \neq \botmark
%     \global\advance\mychapno by 1
%     \global\chappageno=1
%   \else
%       \ifnum\mychapno>=0
%         \global\advance\chappageno by 1
%       \else
%         \global\advance\chappageno by -1
%       \fi
%   \fi
%       {\ninerm Copyright\ \copyright\ 2008 Regents of the University of California}
%     \ifnum\mychapno>0
%      \message{here A}
%   \hfill\number\mychapno \ -- \number\chappageno
%  \else 
%      \message{here AA}
%   \hfill
%    \roman -\number\chappageno
%  \fi
%      \message{here BB}
%    \expandafter\write\newindex {%
%      \thepage-\number\mychapno-\number\chappageno
%     }
%} }}
%
%
%\def\@evenfoot{
%\plainline{ 
%  \makebox[\textwidth]{ 
%      \message{here YE}
%   \if \topmark \neq \botmark
%     \message{tm>bm = \number\tm > \number\bm}
%     \global\advance\mychapno by 1
%     \global\chappageno=1
%   \else
%       \ifnum\mychapno>=0
%         \global\advance\chappageno by 1
%       \else
%         \global\advance\chappageno by -1
%       \fi
%   \fi
%       \ifnum\mychapno>0
%      \message{here B}
%           \number\mychapno \ -- \number\chappageno\hfill
%  \else 
%      \message{here BB}
%   \hfill
%    \roman -\number\chappageno
%    %\hfil\roman\number\thepage\hfil
%  \fi
%          {\ninerm {\slsmall AD Model Builder}}
%    \expandafter\write\newindex {%
%      \thepage-\number\mychapno-\number\chappageno
%     }
%}}}
%
}
\makeatother

\pagestyle{xxx}

% use these for standard book
\oddsidemargin=-.0truein
\evensidemargin=-.25truein
\textheight=8.5truein
\textwidth=6.5truein
% use these for pdf viewer
%\oddsidemargin=-.75truein
%\evensidemargin=-.75truein
%\textheight=8.0truein
%\textwidth=6.0truein
\headheight=-0.0truein
\headsep=0pt
\topmargin=-.35truein
\topskip=-2.75truein
\begin{document}
     %\topmark=0
     %\botmark=0
%\pagestyle{fancy}
%\fancypagestyle{plain} {
%\fancyhf[C]{ }
%\fancyfoot[C}{\thepage}
%  \ifnum\thepage<0
%    \hfil\romannumeral -\thepage\hfil
%  \else
%    \ifodd\thepage
%       \ifnum\thepage>0
%       {\ninerm Copyright\ \copyright\ 2008 Regents of the University of California}
%   \hfill\number\chapno \ -- \number\thepage\fi
%    \else
%       \ifnum\thepage>0 
%           \number\chapno \ -- \number\thepage\hfill
%          {\ninerm {\slsmall AD Model Builder}}
%                                                                 \fi
%    \fi
%   \fi
%\renewcommand{\headrulewidth}{0pt}
%\renewcommand{\footrulewidth}{0pt}

\def\eqspace{\hskip 1000pt mi 990pt}

\def\endchapter{}
%\def\pageno{\thepage}
%\input ./docmacpdf.tex
\def\eqno{}
\def\thespace{ }
%\def\xfolio{\the\linknumber}
%\def\yfolio{\the\linknumber-\number\chapno-\number\pageno}
%\input pictex.tex
\advance\voffset by .5truein
\def\ADM{AD~Model~Builder}
\def\ADMS{AD~Model~Builder }
%\def\chapno{1}
\font\bigbf=cmbx12 at 14truept
\font\rmmedium=cmr8 at 8truept
\font\rmfoot=cmr8 at 8truept 
\font\ttfoot=cmtt8 at 8truept 
%\font\ttsix=cmtt6 at 6truept 
%\font\stt=cmtt6 at 6truept 
\font\ttsix=cmtt8 at 6truept 
\font\stt=cmtt8 at 6truept 
\def\htmlnewfile{}
\def\htmlbeginignore{}
\def\htmlbegintex{}
\def\htmlendtex{}
\def\htmlendignore{}
\def\mybackslash{$\backslash$}
\def\msk{\hskip -1truept}
\def\bmax{B_{\hbox{\ninerm MAX}}}
\def\DS{\hbox{\tt DATA\_SECTION}}
\def\sDS{{\stt DATA\_SECTION}}
\def\sPS{{\stt PARAMETER\_SECTION}}
\def\RS{\hbox{\tt REPORT\_SECTION}}
\def\PS{\hbox{\tt PARAMETER\_SECTION}}
\def\PCS{\hbox{\tt PRELIMINARY\_CALCS\_SECTION}}
\def\IS{\hbox{\tt INITIALIZATION\_SECTION}}
\def\PROS{\hbox{\tt PROCEDURE\_SECTION}}
\def\RUNS{\hbox{[RUNTIME\_SECTION]}}
\def\bmax{B_{\hbox{\ninerm MAX}}}
\def\DS{\hbox{\tt DATA\_SECTION}}
\def\PS{\hbox{\tt PARAMETER\_SECTION}}
\def\PCS{\hbox{\tt PRELIMINARY\_CALCS\_SECTION}}
\def\IS{\hbox{\tt INITIALIZATION\_SECTION}}
\def\PROS{\hbox{\tt PROCEDURE\_SECTION}}
\def\apl{profile likelihood} 
\def\myeq#1{{\myeqwidth=\hsize \advance\myeqwidth by -.5in
   \hbox to \myeqwidth{\hfil${\displaystyle#1}$\hfil}}}
\thispagestyle{yyy}
\input adtitle.tex
\thispagestyle{xxx}
%\tableofcontents
\newpage
%\vfill\eject
%\tracingmacros=1%
%\htmlnewfile
%\def\chapno{5}
%\mychapter{Parallel Processing on clusters}
%\input parallel1.tex
%\endchapter


%\mychapter{Multivariate Probit discrete Choice Models}
%\input mprobit.tex
%\mychapter{Writing Adjoint Code}
%\input adjoint.tex
%\endchapter
\htmlnewfile
% !! for slides
%\openup 5pt
\mychapter{Getting started with AD Model Builder}
%\tracingmacros=0
This manual describes \ADM, the fastest, most powerful
software for rapid development and fitting of general nonlinear statistical
models available. The accompanying demonstration disk 
has a number of example  programs from various fields
including chemical engineering, natural resource modeling, and
financial modeling. As you will see, with a few
statements you can build powerful programs to solve
problems that would completely defeat other
modeling environments. The \ADM\ environment makes
it simple to deal with recurring difficulties in nonlinear
modeling, such as restricting the values which parameters can assume,
carrying out the optimization in a stepwise manner, and
producing a report of the estimates of the standard deviations
of the parameter estimates. And these techniques scale up
to models with at least 5000 independent parameters on a 1000 MH Pentium III 
and more on more powerful platforms. So, if you are interested in
a really powerful environment for nonlinear
modeling -- read on!
\X{template}
\X{default behaviour}
%\bye
\ADM\ provides a template-like approach to code generation.
Instead of needing to write all the code for 
the model the user can employ any ASCII file editor to simply fill
 in the template, describing the 
particular aspects of the model -- data, model parameters, and
the  fitting criterion to be used. 
With this approach the specification of 
the model is reduced to the absolute minimum number of
statements. Reasonable default behaviour
for various aspects of modeling such as the input of data and
initial parameters and reporting of results are provided.
Of course it is possible to override
this default behaviour to customize an application when desired.
The command line argument -ind NAME followed by the string NAME changes the
default data input file to NAME. 
\XX{command line arguments}{-ind NAME input data file}

The various concepts embodied in \ADM\ are introduced
in a series of examples. You should at least skim through each of
the examples in the order they appear so that you will be
familiar with the concepts used in the later examples.
The examples disk 
contains the \ADM\ template code, the \cplus\ code produced
by \ADM\ and the executable programs produced by
compiling the \cplus\ code.  This process of producing the
executable is automated so that the user who doesn't wish to
consider the vagaries of \cplus\  programming can go from
the \ADM\ template to the compiled executable in one step.
Assuming that the \cplus\ compiler and \ADM\ and \AD\ libraries 
have been properly installed, then
to produce a \ADM\ executable it is only necessary to 
type {\tt makeadm  root} where {\tt root.tpl} is the name of the 
ASCII file containing the template specification. 
To simplify model development two modes of operation are
provided, a safe mode with bounds checking on all 
array objects and an optimized mode for fastest execution.
 
\X{Automatic differentiation}
\X{adjoint code}
\ADM\ achieves its high performance levels by employing
 the \AD\ \cplus\ class library. \AD\ combines an array language
 with the reverse mode of Automatic differentiation supplemented
with precompiled adjoint code for the derivatives of common
array and matrix operations. However, all of this is completely
transparent to the \ADM\ user. It is only necessary to
provide a simple description of the statistical 
model desired and the entire process of fitting the model to
data and reporting the results is taken care of automatically.

Although \cplus\ potentially provides good support for 
mathematical modeling, the language is rather complex -- it cannot
be learned in a few days. 
Moreover many features of the language are not needed for mathematical
modeling. A novice user who wishes to build mathematical
models may have a difficult time deciding which
features of the language to learn and which features can be
ignored until later. \ADM\ is
intended to help overcome these difficulties and to speed up
model development. When using \ADM\ most of the aspects of
\cplus\ programming are hidden from the user. 
 In fact the beginning
user can be almost unaware that \cplus\ underlies the implementation
of \ADM.  It is only necessary to be familiar with some of the simpler 
aspects of C or \cplus\ syntax.

\XX{Markov chain simulation}{Hastings-Metropolis algorithm}
\XX{Markov chain simulation}{to estimate the posterior distribution}
\X{Hastings-Metropolis algorithm}
To interpret the results of the statistical analysis \ADM\ provides
simple methods for calculating the profile likelihood and Markov
chain simulation estimates of the posterior distribution
for parameters of interest (Hastings-Metropolis algorithm).

\mysection{What are nonlinear statistical models?}
\ADMS is software for creating computer programs to 
estimate the parameters (or the probability distibution of
parameters) for nonlinear statistical models. 
ithis raises the question ``What is a nonlinear statistical
model?''.  Consider the following model. We have a set
of observations $Y_i$ and $x_{ij}$ where is is assumed that
\begin{equation}
\myeq{Y_i=\sum_{j=1}^m a_j x_{ij}+\epsilon_i}\label{nl:xx1}
\end{equation}
where the $\epsilon_i$ are assumed to be normally distributed
random variables with equal variance $\sigma^2$.

Given these assumptions it can be shown that ``good''
estimates for the unknown parameters $a_j$ are obtained by minimizing
\begin{equation}
\myeq{ \sum_i(Y_i-\sum_{j=1}^m a_jx_{ij})^2}\label{nl:xx2}
\end{equation}
with repect to these parameters.
These minimizing values can be found by taking the
derivatives with respect to the $a_j$ and setting them equal to
zero. 
Since \ref{nl:xx1} is linear in the $a_j$ and \ref{nl:xx2}
is quadratic it folows that the equations given by setting the
derivatives equal to 0 are linear in the $a_j$  so that the 
estimates can be found by solving a system of linear
equations. 
For this reason such a statistical model
is referred to as linear. 
Over time very good numerically stable methods have
been developed for calculating these least-squares estimes.
For situations where either
the equations in the model corresponding to  \ref{nl:xx1}
are not linear or the statistical assumptions involve
non normal random variables then the methods for finding
good parameter estimates will involve minimizing functions
which are not quadratic in the unknown parameters $a_j$.

In general these optimization problems are much more
difficult than those arising in least-squares problems.
There are however various techniques which render the
estimation of parameters in such nonlinear models
more tractable. The \ADMS package is intended to
organize these techniques in such a way that they
are easy to employ (where possible
employing them in a way that the user does
not need to be aware of them) so that investigating
nonlinear statical models becomes so far as possible
as simple as linear statistical models.

\mysection{Installing the software}
This section contains a discussion of the installation issues
which are common to all compilers and platforms. More specific
instructions may be found with the binaries wihich are supplied for 
a particular compiler.
The first thing to understand is that \ADMS is not a GUI
environment and there is no pointing or clicking. You will be
using a text editor and command line tools. (It is however
possible to use \ADMS from within an IDE such as comes with
visual C++.)

\X{installation}
\newcounter{beans}
\begin{list}{\arabic{beans}}{\usecounter{beans}}
\item First you must decide which C++ compiler you wish to use.
There are several free ones for WIN32 on the \ADMS CD. Check to see
that your compiler is supported that is that there are \ADMS
binaries on the CD for it.
\item Decide which directory you wish to put the \ADMS
binaries in. {\tt C:\mybackslash ADMODEL} is the easiest to use
because then you will not need to edit the command line
files as much.  Make the directory. On MS machines this is
done my changing to the C: directory in the command window
and typing {\tt mkdir C:\mybackslash ADMODEL}
\item Copy the file containg the \ADMS binaries from the
CD to the directory you have chosen. For example if you want the
binaries for visual C++ version 6.0 they are in the file
{\tt MSC6DISK.EXE} in the directory {\tt \mybackslash files\mybackslash vcpp.60}.
change to the directory where you copied the file and type
{\tt MSC6DISK.EXE -d} The -d is necessary to create the
necessary subdirectories. Of course this is for MS users. If you
are on Linux etc the binaries are in a gzipped or bzipped tar file 
and you should uncompress them and use tar to extract them from the tar file. 
\item You will now need to tell the operating system to look in the
file {\tt C:\mybackslash ADMODEL\mybackslash bin} for the
\ADMS files. On Windows 95-98 this is done by editing the
path statement in the autoexec.bat file. On NT you edit
the path environment string in the environment tab of the
system window in the control panel. Once you have done this
(and perhaps rebooted 25 times) you should be able to 
type {\tt tpl2cpp} and get an error message that tpl2cpp can't find
a file. If the systme complaions that it can't find tpl2cpp
or doesn't recognize {\tt tpl2cpp} you have not done this right.
\item  You will also need to tell the operating system
where to find the command line compiler. If you are using
visual C++ the file is named {\tt cl.exe}, so put the
directory which contains this on your path as above.
then you should be able to type something like {\tt cl}
or {\tt bcc32} for the borland compiler or {\tt gcc}
for the Gnu compiler and get some response. If not
the path is incorrect.  
\item   Now you need to tell the compiler where to find
the \ADMS header files. these are the files in the
directory {\tt C:\mybackslash ADMODEL\mybackslash INCLUDE}
There are several ways to do this. With Visual C++ 
the easiest way is to use the {\tt -I} command line option
for the bat file in
{\tt C:\mybackslash ADMODEL\mybackslash BIN}
which invokes the compiler. For visual C++ it is the file {\tt myvcc.bat}
which contains something like.
\beginexample
cl /c -I. -D__MSVC32__ -DOPT_LIB /O2 -Ic:\mybackslash admodel\mybackslash include -Ic:\mybackslash vc6\mybackslash INCLUDE %1.cpp
\endexample
\item Now you will need to tell the linker where to find the \ADMS libraries.
This can be done by using the {\tt /link /libpath:} option in the bat file which does
the linking. For Visual C++ 6.0 this is in the file {\tt LINKVCC.BAT}
which contains
\beginexample
cl %1.obj admod32.lib adt32.lib ado32.lib libc.lib /link /libpath:c:\mybackslash admodel\mybackslash lib 
\endexample
For Borland and most compilers this can be done with the {\tt -L}
option. There amy be additional instructions on installation in the
readme files contained with the \ADMS files for the various compilers.
\end{list}

\mysection{The sections in an \ADMS TPL file}
\X{template sections}
\X{template}
\X{TPL file}
An AD Model Builder template (TPL file) consists of up to eleven sections.
Eight of these sections are optional. Optional sections are 
enclosed in brackets [ ]. The optional FUNCTION keyword defines a 
subsection  of the \PROS.

\noindent The simplest model contains only the three required sections,
a \DS, a \PS, and a \PROS. 

{\parindent=0pt
\beginexample
  DATA\_SECTION
  
  [INITIALIZATION\_SECTION]
  
  PARAMETER\_SECTION

  [PRELIMINARY\_CALCS\_SECTION]

  PROCEDURE\_SECTION
    [FUNCTION]
  
  [REPORT\_SECTION]

  [RUNTIME_SECTION]

  [TOP_OF_MAIN_SECTION]

  [GLOBALS\_SECTION]

  [BETWEEN\_PHASES\_SECTION]

  [FINAL\_SECTION]
\endexample
 }


\mysection{The Original \ADMS examples}
This section includes a short description of the original
examples distributed with \ADM. There are now many more examples
which are discussed in subsequent chapters.

\XX{examples}{short description of}

%\item{1.} 
A very simple example. This is a trivial least squares
linear model included simply to introduce the basics of 
\ADM.

\XX{regression}{robust}
\XX{regression}{nonlinear}
\X{robust regression}
\X{nonlinear regression}
%\item{2.}  
A simple nonlinear regression model for estimating the parameters
describing a von Bertalanffy growth curve from size-at-age data.
\ADM's robust regression routine is introduced and used to illustrate
how problems caused by ``outliers'' in the data can be avoided. 

%\item{3.}  
A chemical kinetics problem.  A model defined by a
system of ordinary differential equations. The purpose is to
estimate the parameters which describe the chemical reaction.  

%\item{4.} 
A problem in financial modeling. A Generalized Autoregressive 
Conditional \hbox{Heteroskedasticity} or GARCH model  is used
to attempt to describe  the time series of returns from some
market instrument.

%\item{5.}
A problem in natural resource management. The 
Schaeffer-Pella-Tomlinson Model
for investigating the response of an exploited fish population
is developed and extended to include
a Bayesian times series treatment of time-varying carrying 
capacity.  This example is interesting because the model
is rather tempermental and several techniques for
producing reliable convergence of the estimation procedure to
the correct answer are described. For one of the data sets over 
100~parameters are estimated.  

%\item{6.}  
A Simple Fisheries catch-at-age model. These models
are used to try and estimate the exploitation rates etc. in
exploited fish populations. 

%\item{7.} 
More complex examples are presented in subsequent chapters. 
\par

\tracingmacros=1
\mysection{Example 1 -- linear least-squares}
\tracingmacros=0
\X{least squares}
\X{simple example}
To illustrate the method we begin with a simple statistical
model which is to estimate the parameters of
a linear relationship of the form 
$$Y_{i} =ax_i+b\qquad \hbox{\rm for}\ 1<=i<=n$$
where $x_i$ and $Y_i$ are vectors, and $a$ and $b$ are the model parameters
which are to be estimated. The parameters are estimated by the method
of least-squares that is we find the values of $a$ and $b$ so that
the sum of the squared differences between the observed values
$Y_i$ and the predicted values $ax_i+b$ is minimized. That is
we want to solve the problem
$$\min_{a,b}\sum_{i=1}^n(Y_i-ax_i-b)^2$$   

The template for this model is in the file {\tt SIMPLE.TPL}. To
make the model one would type {\tt makeadm simple}. The resulting 
executable for the model is in the file {\tt SIMPLE.EXE}.
The contents of {\tt SIMPLE.TPL} are.
(Anything following // is a comment.)

\XX{SECTIONS}{\tt DATA\_SECTION}
\XX{SECTIONS}{\tt PARAMETER\_SECTION}
\XX{SECTIONS}{\tt PROCEDURE\_SECTION}
\beginexample
DATA_SECTION
  init_int nobs             // nobs is the number of observations
  init_vector Y(1,nobs)     // the observed Y values
  init_vector x(1,nobs)
PARAMETER_SECTION
  init_number a   
  init_number b   
  vector pred_Y(1,nobs)
  objective_function_value f
PROCEDURE_SECTION
  pred_Y=a*x+b;       // calculate the predicted Y values
  f=regression(Y,pred_Y);  // do the regression -- the vector of 
                           // observations goes first
\endexample
The main requirement is that all keywords must begin in column 1
while the code itself must be indented.  

\mysection{The DATA SECTION}
Roughly speaking, the data consist of the stuff in the real
world which you observe and want to analyze. The data section
describes the structure of the data in your model. Data objects 
consist of integers ({\tt int}) and floating point numbers({\tt number}),
and these can be grouped into one dimensional ({\tt ivector} and {\tt vector}) 
and two dimensional ({\tt imatrix} and {\tt matrix}) arrays.
The ``{\tt i}'' in {\tt ivector} distinguishes a vector of
type {\tt int} from a vector of type {\tt number}.
For arrays of type {\tt number} there are currently arrays up to dimension~7.

\X{default file names}
Some of your data must be read in from somewhere, that is,
you need to start with something. These data objects are 
referred to as initial objects and are
distinguished by the prefix {\tt init}, such as {\tt init\_int}
or {\tt init\_number}.
All objects prefaced with {\tt init} in  the \DS\ are 
read in from a data file in the order in which they are declared. 
The default file names for various files are derived from the
name of the executable program. If the executable file is named
{\tt ROOT.EXE} then  the default input data file name is {\tt ROOT.DAT}. 
For this example the executable file is named {\tt SIMPLE.EXE}
so the default data file is {\tt SIMPLE.DAT}.
Notice that
once an object has been read in, its value is available
to be used to describe other data objects. In this case the
value of {\tt nobs} can be used to define the size of the
vectors {\tt Y} and {\tt x}. The next line {\tt init\_vector Y(1,nobs)}
defines an initial vector object {\tt Y} whose minimum valid index is 1,
and whose maximum valid index is {\tt nobs}. This vector
object will be read in next from the data file.
The contents of the file {\tt SIMPLE.DAT} are shown below. 
\X{data file}
\beginexample
# number of observations
     10
# observed Y values
    1.4  4.7  5.1  8.3  9.0  14.5  14.0  13.4  19.2  18 
# observed x values
    -1  0 1  2  3  4  5  6  7  8        
\endexample

\noindent It is possible to put comment lines in the data files.
\X{comments in data file}
Comment lines must have the character {\tt\#} in the first
column.

It is often useful to have data objects which are not initial.
Such objects have their values calculated from the values of
initial data objects. Examples of the use of non initial data objects
are given below.

\X{PARAMETER\_SECTION}
\XX{SECTIONS}{\tt PARAMETER\_SECTION}
\mysection{The Parameter Section}

It is the parameters of your model which
provide the analysis of the data (or perhaps more correctly
is the values of these parameters as picked by the
fitting criterion for the model which provide the
analysis of the data).
The {\tt PARAMETER\_SECTION} is used to describe the structure of the  
parameters in your model.
The description of the model parameters is similar to that used for
the data in the \DS.

All parameters are floating point numbers (or arrays of floating point numbers.)
The \hbox{statement} {\tt init\_number b} defines
a floating point number  (actually a double). The preface {\tt init}
means that this is an initial parameter. Initial parameters have
two properties which distinguish them from other model
parameters. First, all of the other model parameters are calculated from
the initial parameters. This means that in order to calculate the
values of the model parameters it is first necessary to have 
values for the initial parameters. A major difference between
initial data objects (which must be read in from a data file) 
and initial parameters is that since parameters are estimated 
in the model it is possible to assign initial default values to them.

The default file name for the file which contains initial
values for the initial model parameters is
{\tt ROOT.PIN}. If no file named {\tt ROOT.PIN} is found,
default values are supplied for the initial parameters.
(Methods for changing the default values for initial parameters
are described below.)
The statement  {\tt vector pred\_Y(1,nobs)} defines
a vector of parameters. Since it is not prefaced with {\tt init}
the values for this vector will not be read in from a file
or given default values. It is expected that the value of the
elements of this vector will be calculated in terms of
other parameters.

The statement  {\tt objective\_function\_value f} defines a floating 
point number 
(again actually a double). It will hold the
value of the fitting criterion. The parameters of the model are
chosen so that this value is minimized\footnote{Thus it should be set equal to
minus the log-likelihood function if that criterion is used}. 
Every \ADM\ template
must include a declaration of an object of type 
{\tt objective\_function\_value} and  this object must be
set equal to a fitting criterion. (Don't worry, for many
models the
fitting criterion is provided for you as in the {\tt regression}
and {\tt robust\_regression} fitting criterion functions 
in the current and next examples.
\X{objective\_function\_value} 
\X{fitting\_criterion} 

\mysection{The Procedure Section}
The \PROS\ contains the actual model calculations.
This section contains \cplus\ code and \cplus\ syntax must be obeyed.
(Those familiar with C or \cplus\  will notice that the usual
methods for defining and ending a function are not necessary 
and in fact can not be used for the routine in the main part of this
section.) 
 
\X{syntax rules}
\X{use of vector and matrix calculations}
Statements must end with a ``{\tt ;}'' exactly as with C or \cplus.  
The ``{\tt ;}'' is
optional in the \DS\ and the \PS.
The code uses \AD's vector operations which enable you to avoid
writing a lot of code for loops.
In the statement {\tt pred\_Y=a*x+b;} the symbol {\tt a*x}
forms the product of the number {\tt a} and the components of the
vector {\tt x} while {\tt +b} adds the value of the number {\tt b}
to this product so that {\tt pred\_Y} has the components 
$ax_i+b$.
In the line {\tt f=regression(Y,pred\_Y);} the function 
{\tt regression} calculates the log-likelihood function for
the regression and assigns this value to the object {\tt f}
which is of type {\tt objective\_function\_value}. 
This code generalizes immediately to
nonlinear regression models and can be trivially
modified (with the addition of one word) to perform the robust nonlinear
regression discussed in the second example. For the reader who want to know,
the form of the regression function is described in the Appendix.  

\X{use of {\tt regression} function}
\X{PRELIMINARY\_CALCS\_SECTION}
\XX{SECTIONS}{\tt PRELIMINARY\_CALCS\_SECTION}
\noindent Note that the vector of observed values goes first. 
The use of the {\tt regression} function makes the purpose of the
calculations clearer, and it prepares the way for
modifying the routine to use \ADM's robust regression
function.

\mysection{The Preliminary Calcs Section}


\XX{LOCAL\_CALCS}{use instead of the PRELIMINARY\_CALCS\_SECTION}
NOTE:  The use of {\tt LOCAL\_CALCS} and its variants in the
{\tt DATA\_SECTION} and the {\tt PROCEDURE\_SECTION}  has
greatly reduced the need for the {\tt PRELIMINARY\_CALCS\_SECTION}.

The {\tt PRELIMINARY\_CALCS\_SECTION} as its name implies
permits one to do preliminary calculations with the data
before getting into
the model proper.  Often the input data are not in a convenient form
for doing the analysis and one wants to carry out some calculations with
the input data to put them in a more convenient form.
Suppose that the input data for the simple regression model are in the form
\beginexample
# number of observations
     10
# observed Y values  observed x values
    1.4                    -1
    4.7                     0
    5.1                     1
    8.3                     2
    9.0                     3
    14.5                    4
    14.0                    5
    13.4                    6
    19.2                    7
    18                      8
\endexample
\noindent The problem is that the data are in pairs in the 
form $(Y_i,x_i)$, so that we can't read in either the 
$x_i$ or $Y_i$ first.  To read in the data in this format 
we will define a matrix  with {\tt nobs} rows and 2
columns.
The  \DS\ becomes
\beginexample
DATA\_SECTION
  init_int nobs
  init_matrix Obs(1,nobs,1,2) 
  vector Y(1,nobs)
  vector x(1,nobs)
\endexample
\noindent Notice that since we do not want to read in {\tt Y} or
{\tt x} these objects are no longer initial objects, that is their
declarations are no longer prefaced with {\tt int}. 
The observations will be read into the initial matrix object {\tt Obs}
so that {\tt Y}  is in the first column of {\tt Obs} while
{\tt x} is in the second column.
If we don't want to change the rest of the code the next problem is to
get the first column of {\tt Obs} into {\tt Y} and the second column
of {\tt Obs} into {\tt x}.
The following code in the {\tt PRELIMINARY\_CALCS\_SECTION}
will accomplish this objective. It uses the function {\tt column}
which extracts a column from a matrix object so that it can be put into a
vector object. 
\XX{column}{extract a column from a matrix}
\beginexample
PRELIMINARY\_CALCS\_SECTION
 Y=column(Obs,1); // extract the first column
 x=column(Obs,2); // extract the second column
\endexample
\mysection{The use of loops and element-wise operations}
This section can be skipped on first reading.

To accomplish the column-wise extraction presented above 
you would have to know that
\AD\ provides the {\tt column} operation. What if you didn't
know that and don't feel like reading the manual yet?
For those who are familiar with C 
 it is generally possible to use lower level ``C-like'' operations
to accomplish the same objective as \AD's array and matrix operations.
In this case the columns of the matrix {\tt Obs} can also be
copied to the vectors {\tt x} and {\tt Y} by
using a standard {\tt for loop} and the following element-wise operations
\beginexample
PRELIMINARY_CALCS_SECTION
  for (int i=1;i<=nobs;i++)
  {
    Y[i]=Obs[i][1];
    x[i]=Obs[i][2];
  }
\endexample
Incidentally, the C-like operation {\tt []} was used
for indexing members of arrays. \ADM\ also supports 
the use of {\tt ()} so that the above code could be written as
\XX{matrix objects}{accessing elements of}
\X{Accessing elements of matrix objects}
\beginexample
PRELIMINARY_CALCS_SECTION
  for (int i=1;i<=nobs;i++)
  {
    Y(i)=Obs(i,1);
    x(i)=Obs(i,2);
  }
\endexample
\noindent which may be more readable for some users.
Notice that it is also possible to define {\tt C} objects
like the object of type {\tt int i} used as the index for the 
{\tt for loop} ``on the fly" in the
{\tt PRELIMINARY\_CALCS\_SECTION} or the {\tt PROCEDURE\_SECTION}.


\X{output files}
\mysection{The default output from \ADM}
By default \ADM\ produces three or more files {\tt ROOT.PAR} which contains the parameter
estimates  in ASCII format, {\tt ROOT.BAR} which is the parameter estimates in a binary file format,
and {\tt ROOT.COR} which contains the estimated standard
deviations and correlations of the parameter estimates.
The template code for the simple model  is in the file file {\tt SIMPLE.TPL}.
The input data is in the
file {\tt SIMPLE.DAT}. The parameter estimates are in the file
{\tt SIMPLE.PAR}. By default the standard deviations and
the correlation matrix for the
model parameters are estimated. They are in the file {\tt SIMPLE.COR} 
\X{correlation matrix}
\beginexample
 index      value      std dev       1       2   
    1   a  1.9091e+00 1.5547e-01       1
    2   b  4.0782e+00 7.0394e-01  -0.773       1
\endexample
\X{standard deviations report}
\XX{standard deviations report}{\tt STD}
\X{\tt STD}
\noindent The format of the standard deviations report is to give the 
name of the
parameter followed by its value and standard deviation. After that the
correlation matrix for the parameters is given. 
\htmlnewfile
\mysection{Robust Nonlinear regression with \ADM}
\X{robust regression}
The code for the admodel template for this example 
is found in the file {\tt VONB.TPL}.
This example is intended to demonstrate the advantages 
of using \ADM's robust regression routine over standard nonlinear
least square regression procedures. Further discussion about the
underlying theory can be found in the AUTODIF User's Manual, but
it is not necessary to understand the theory to make use of the
procedure.
\medskip
\vbox{
\quad\hbox to \hsize{
\hfill
\beginpicture
  \setcoordinatesystem units <.18in,.04in>
  \setplotarea x from 0 to 16.5, y from 0 to 50 
  \axis left label {Size} ticks
    numbered from 0 to 50 by 10 
  /
  \axis bottom label {Age} ticks
    numbered from 0 to 16 by 2 
  /
 \multiput {\hbox{$\bullet$}} at "vonbg.dat" 
 \put {Figure 1} at 4 45
 \plot  "vv.rep" 
\endpicture
\hfill
}
\medskip
\centerline{Results for nonlinear regression with good data set}
\medskip
{\openup 1pt
\beginexample
 index         value      std dev       1       2       3   
    1   Linf  5.4861e+01 2.4704e+00  1.0000
    2   K     1.7985e-01 2.7127e-02 -0.9191  1.0000
    3   t0    1.8031e-01 2.9549e-01 -0.5856  0.7821  1.0000
\endexample
}}
\medbreak
This example estimates the parameters describing a growth curve
from a set of data consisting of ages and size-at-age data.
The form of the (von Bertalanffy) growth curve is assumed to be
%$$
\begin{equation}
\myeq{s(a)=L_{\infty}\big(\,1-\exp(-K(a-t_0))\,\big)}\label{chp1:xx2}
\end{equation}
%$$
The three parameters of the curve to be estimated are
$L_{\infty}$, $K$, and $t_0$.

Let $O_i$ and $a_i$ be the observed size and age of the $i$'th 
animal. The predicted size $s(a_i)$ is given by equation \ref{chp1:xx2}.
The least squares estimates for the parameters are found by minimizing
$$\min_{L_\infty,K,t_0} \sum_i \big(O_i -s(a_i)\,\big)^2$$
\beginexample
DATA_SECTION
  init_int nobs;
  init_matrix data(1,nobs,1,2)
  vector age(1,nobs);
  vector size(1,nobs);
PARAMETER_SECTION
  init_number Linf;
  init_number K;
  init_number t0;
  vector pred_size(1,nobs)
  objective_function_value f;
PRELIMINARY_CALCS_SECTION
  // get the data out of the columns of the data matrix 
  age=column(data,1);
  size=column(data,2);
  Linf=1.1*max(size);  // set Linf to 1.1 times the longest observed length
PROCEDURE_SECTION
  pred_size=Linf*(1.-exp(-K*(age-t0)));
  f=regression(size,pred_size);
\endexample
Notice the use of the 
{\tt regression function} which calculates the
log-likelihood function of the nonlinear least-squares regression. 
 This part of the code is formally
identical to the code for the linear regression problem in the
simple example even though we are now doing nonlinear regression.
A graph of the least-square estimated growth curve and the observed
data is given in figure~1. The parameter estimates and their estimated 
standard deviations which are produced by \ADM\ are also given.
For example the estimate for $L_\infty$ is  54.86 with a standard
deviation of 2.47. Since a 95\% confidence limit is about 
$\pm$ two standard deviations the usual 95\% confidence limit of
$L_\infty$ for this analysis would be $54.86\pm 4.94$.

A disadvantage of least squares regression is the sensitivity of the 
estimates to a few ``bad'' data points or outliers. Figure~2 show the
least squares estimates when the observed size for age~2 and age~14
have been moved off the curve.
There has been a rather large change in some of the parameter
estimates. For example the estimate for $L_\infty$ has changed
from $54.86$ to $48.91$ and the estimated standard deviation for
this parameter has increased to $5.99.$
This is a common effect of outliers on least-squares estimates.
They greatly increase  the size of the estimates of the standard deviations.
As a result the confidence limits for the parameters are increased.
In this case the 95\% confidence limits  for 
$L_\infty$ have been increased from $54.86\pm 4.94$ to $48.91\pm 11.98$.

\medbreak
Of course for this simple example it could be argued that a visual
examination of the residuals would identify the outliers so that
they could be removed. This is true, but in larger nonlinear models
it is often not possible or convenient to identify and remove
all the outliers in this fashion. Also the process of removing
``inconvenient'' observations from data can be uncomfortably
close to ``cooking'' the data in order to obtain the desired result
from the analysis. An alternative approach which avoids these
difficulties is to employ \ADM's robust regression procedure
which removes the undue influence of outlying points without the
need to expressly remove them from the data.

\vfil
\vbox{
\medskip
\quad\hbox{
\beginpicture
  \setcoordinatesystem units <.18in,.04in>
  \setplotarea x from 0 to 16.5, y from 0 to 50 
  \axis left label {Size} ticks
    numbered from 0 to 50 by 10 
  /
  \axis bottom label {Age} ticks
    numbered from 0 to 16 by 2 
  /
 \multiput {\hbox{$\bullet$}} at "vvbad.dat" 
 \plot  "vvbad.rep" 
 \put {Figure 2} at 4 45
\endpicture
\hfill
}}
\medskip
\quad Nonlinear regression with bad data set
\medskip
{\openup 1pt
\beginexample
 Nonlinear regression with bad data set
 index         value      std dev       1       2       3   
    1   Linf  4.8905e+01 5.9938e+00  1.0000
    2   K     2.1246e-01 1.2076e-01 -0.8923  1.0000
    3   t0   -5.9153e-01 1.4006e+00 -0.6548  0.8707  1.0000
\endexample
}
\mysection{Modifying the model to use robust nonlinear regression}

To invoke the robust regression procedure it is necessary to
make three changes to the existing code. The template for the
robust regression version of the model can be found in the
file {\tt VONBR.TPL}. 
\beginexample
DATA_SECTION
  init_int nobs;
  init_matrix data(1,nobs,1,2)
  vector age(1,nobs)
  vector size(1,nobs)
PARAMETER_SECTION
  init_number Linf
  init_number K
  init_number t0
  vector pred_size(1,nobs)
  objective_function_value f
  init_bounded_number a(0.0,0.7,2)
PRELIMINARY_CALCS_SECTION
  // get the data out of the columns of the data matrix 
  age=column(data,1);
  size=column(data,2);
  Linf=1.1*max(size);  // set Linf to 1.1 times the longest observed length
  a=0.7;
PROCEDURE_SECTION
  pred_size=Linf*(1.-exp(-K*(age-t0)));
  f=robust_regression(size,pred_size,a);
\endexample
The main modification to the model involves the addition of the 
parameter {\tt a}, which is used to 
estimate the amount of contamination by outliers.  
This parameter is declared in the
\PS.
\beginexample
  init_bounded_number a(0.0,0.7,2)
\endexample
\X{putting bounds on initial parameters}
\X{init\_bounded\_number}
\noindent This introduces two concepts, putting bounds on the
values which initial parameters can take on and carrying out 
the minimization in a number of stages. The value of {\tt a}
should be restricted to lie between 0.0 and 0.7 
(See the discussion on robust regression in the AUTODIF user's
manual if you want to know where the 0.0 and 0.7 come from).
This is accomplished by declaring {\tt a} to be of type
{\tt init\_bounded\_number}.
\X{multi-phase minimization}
In general it is not possible to estimate the parameter {\tt a}
determining the amount of contamination by outliers until the other
parameters of the model have been ``almost'' estimated, that is,
until we have done a preliminary fit of the model. This is a common
situation in nonlinear modeling and is discussed further in some
later examples. So
we want to carry out the minimization in two phases. During the
first phase {\tt a} should be held constant.  In general
for any initial parameter the last number in its declaration,
if present, determines the number of the phase in which that
parameter becomes active. If no number is given the parameter
becomes active in phase 1. (Note: For an {\tt init\_bounded\_number}
the upper and lower bounds must be given so the declaration
\beginexample
  init_bounded_number a(0.0,0.7)
\endexample
\noindent would use the default phase 1.
The {\tt 2} in the
declaration for {\tt a} causes {\tt a} to be constant until the second
phase of the minimization.  The second change to the model
involves the default initial value {\tt a}. The default value for
a bounded number is the average of the upper and lower
bounds. For {\tt a} this would be
$0.35$ which is too small. We want to use the upper bound of $0.7$.
This is done by adding the line
\beginexample
  a=0.7;
\endexample
\noindent in the {\tt PRELIMINARY\_CALCS\_SECTION}. Finally we modify
the statement in the {\tt PROCEDURE\_SECTION} 
including the {\tt regression} function to 
\beginexample
  f=robust_regression(size,pred_size,a);
\endexample
\noindent to invoke the robust regression function. That's all there is to
it! These three changes will convert any AD Model builder template
from a nonlinear regression model to a robust nonlinear regression model. 
\bigbreak
\vbox{
\medskip
\quad\hbox{
\beginpicture
  \setcoordinatesystem units <.18in,.04in>
  \setplotarea x from 0 to 16.5, y from 0 to 50 
  \axis left label {Size} ticks
    numbered from 0 to 50 by 10 
  /
  \axis bottom label {Age} ticks
    numbered from 0 to 16 by 2 
  /
 \multiput {\hbox{$\bullet$}} at "vvbad.dat" 
 \put {Figure 3} at 4 45
 \plot  "vvrbad.rep" 
\endpicture
\hfill
}}
\medskip
\quad Robust Nonlinear regression with bad data set
\medskip
{\openup 1pt
\beginexample
 index         value      std dev       1       2       3       4   
    1   Linf  5.6184e+01 3.6796e+00  1.0000
    2   K     1.6818e-01 3.4527e-02 -0.9173  1.0000
    3   t0    6.5129e-04 4.5620e-01 -0.5483  0.7724  1.0000
    4   a     3.6144e-01 1.0721e-01 -0.1946  0.0367 -0.2095  1.0000
\endexample
}
\X{confidence limits}
The results for the robust regression fit to the bad data set are shown in
figure~3. The estimate for $L_\infty$ is $56.18$ with a standard
deviation of $3.68$ to give a 95\% confidence interval of about
$56.18\pm 7.36$. Both the
parameter estimates and the confidence limits are much less affected 
by the outliers for the robust regression  estimates than
they are for the least squares estimates. The
parameter {\tt a} is estimated to be equal to $0.36$ which indicates
that the robust procedure has detected some moderately large outliers.

The results for the robust regression fit to the good data set are shown in
figure~4. The estimates are almost identical to the least-square 
estimates for the same data. This is a property of the robust estimates.
They do almost as well as the least-square estimates when the
assumption of normally distributed errors in the statistical model 
is satisfied exactly, and they do much better than least square estimates
in the presence of moderate or large outliers. You can lose only a little
and you stand to gain a lot by using these estimators. 

\vbox{
\medskip
\quad\hbox{
\beginpicture
  \setcoordinatesystem units <.18in,.04in>
  \setplotarea x from 0 to 16.5, y from 0 to 50 
  \axis left label {Size} ticks
    numbered from 0 to 50 by 10 
  /
  \axis bottom label {Age} ticks
    numbered from 0 to 16 by 2 
  /
 \multiput {\hbox{$\bullet$}} at "vonbg.dat" 
 \put {Figure 4} at 4 45
 \plot  "vvrgood.rep" 
\endpicture
\hfill
}}
\medskip
 Robust Nonlinear regression with good data set
\medskip
{\openup 1pt
\beginexample
 index         value      std dev       1       2       3       4   
    1   Linf  5.5707e+01 1.9178e+00  1.0000
    2   K     1.7896e-01 1.9697e-02 -0.9148  1.0000
    3   t0    2.1490e-01 2.0931e-01 -0.5604  0.7680  1.0000
    4   a     7.0000e-01 3.2246e-05 -0.0001  0.0000 -0.0000  1.0000
\endexample
}
\goodbreak
\X{chemical engineering}
\X{chemical kinetics}
\htmlnewfile
\mysection{Chemical engineering -- a chemical kinetics problem}

\def\mynarrower{\advance\leftskip by 20truept\advance\rightskip by 20truept}
\def\ls{least squares }
\def\ei{\epsilon_i}
\def\mm{least median of squared residuals }
This example may strike you as being fairly complicated. If so, you
should compare it with the original solution using the so-called
sensitivity equations.
The reference is Bard, {\it Nonlinear Parameter Estimation},
chapter 8. We consider the chemical kinetics problem introduced on
page 233.  This is a model defined by a first order system of two ordinary
differential equations.
\htmlbeginignore
\def\guts{\exp\big(-\theta_2/T\big)\big(s_1-e^{-1000/T}s_2^2\big)/
       \big(1+\theta_3\exp(-\theta_4/T)s_1\big)^2}

\htmlendignore
%$$\eqalign{
 \begin{eqnarray}
ds_1/dt&=-\theta_1\guts \cr
           ds_2/dt&=2\theta_1\guts \cr % \eqno(1)
 \end{eqnarray}
%}$$
       
\htmlbeginignore
\def\gutst{\exp\big(-\theta_2/T\big)\big(s_1(t_n)-e^{-1000/T}s_2(t_n)^2\big)/
       \big(1+\theta_3\exp(-\theta_4/T)s_1(t_n)\big)^2}
\htmlendignore
\noindent The differential equations describe the evolution over time
of the
concentrations of the two reactants, $s_1$, and $s_2$.
There are ten initial parameters in the model,
$\theta_1,\ldots,\theta_{10}$.
$T$ is the temperature at which the reaction takes place.
To integrate the system of differential equations we require the
initial concentrations of the reactants, $s_1(0)$ and $s_2(0)$
 at time~0.

The reaction was carried out three times at temperatures of 200, 400,
and~600 degrees. For the first run there were initially equal concentrations
of the two reactants. The second run initially consisted
of only the first reactant, and
the third run initially consisted of only the second 
reactant. The initial concentrations of
the reactants are known only approximately. They  are

\htmlbegintex
{ \tabskip=30pt plus 50pt minus 20pt
\halign to \hsize{\qquad#\hfil&#\hfil&#\hfil\qquad\cr
 Run 1&$s_1(0)=\theta_5=1\pm0.05$&$s_2(0)=\theta_6=1\pm0.05$\cr
 Run 2&$s_1(0)=\theta_7=1\pm0.05$&$s_2(0)=0$\cr
 Run 3&$s_1(0)=0$&$s_2(0)=\theta_8=1\pm0.05$\cr}}
\htmlendtex

\noindent The unknown initial concentrations are treated as parameters to be
estimated with Bayesian prior distributions on them reflecting the
level of certainty of their true values which we have. 
The concentrations of the reactants were not measured 
directly. Rather the mixture was analyzed by a ``densitometer''
whose response to the concentrations of the reactants is
$$y=1+\theta_9s_1+\theta_{10}s_2$$
where $\theta_9=1\pm0.05$ and $\theta_{10}=2\pm0.05$. The differences 
between the predicted and 
observed responses of the densitometer are assumed to be normally distributed
so that least squares is used to fit the model. 
Bard employs an ``explicit'' method for integrating these differential 
equations, that is, the equations are approximated by a finite difference 
scheme like
%$$\eqalign{
 \begin{eqnarray}
s_1(t_{n+1})&=&s_1(t_n)\cr
    &-h&\theta_1\gutst\cr
           s_2(t_{n+1})&=&s_2(t_n)\cr
    +&2h&\theta_1\gutst \cr \label{chp1:xx3}
 \end{eqnarray}
%$$
over the time period $t_n$ to $t_{n+1}$ of length $h$.
Equations 2 are called explicit because the values of $s_1$
and $s_2$ at time $t_{n+1}$ are given explicitly in terms
of the values of $s_1$ and $s_2$ at time $t_n$.

The advantage of using an explicit scheme for integrating the 
model differential equations is that the derivatives of the
model functions with respect to the model parameters also satisfy
differential equations -- called sensitivity equations
(Bard pg 227-229).
It is possible to integrate these equations as well as the model equations
to get values for the derivatives. However this involves generating
a lot of extra code as well as carrying out a lot of
extra calculations. Since with \ADM\ it is not necessary to
produce any code for derivative calculations  it is possible
to employ alternate schemes for integrating the differential
equations. 

Let $A=\theta_1\exp(-\theta_2/T)$, $B=\exp(-1000/T)$, and  
$C=(1+\theta_3\exp(-\theta_4/T)s_1)^2$
In terms of $A$ and $C$ we can replace  explicit finite difference 
scheme by the semi-implicit scheme
%$$\eqalign{
 \begin{eqnarray}
   s_1(t_{n+1})&=&
    s_1(t_n)-hA\big(s_1(t_{n+1})-B  s_2^2(t_{n+1})\big)/C \cr
           s_2(t_{n+1})&=&
   s_2(t_n)+2hA\big(s_1(t_{n+1})-Bs_2(t_n) s_2(t_{n+1})\big)/C\cr
 \end{eqnarray}
%} \eqno(3)$$
Now let $D=hA/C$ and solve equations (3) for 
$s_1(t_{n+1})$ and $s_2(t_{n+1})$ to obtain
%$$\eqalign{
 \begin{eqnarray}
s_1(t_{n+1})&=&\big(s_1(t_n)+DB s_2(t_n)\big)/(1+D) \cr
           s_2(t_{n+1})&=&\big(s_2(t_n)+2Ds_1(t_n)\big)/
             \big(1+(2DBs_2(t_n))\big)\cr
 \end{eqnarray}
%} \eqno(4)$$
Implicit and semi-implicit schemes tend to be more stable than
explicit schemes over large time steps and large values of some of the
model parameters. This stability is especially important when
fitting nonlinear models because the algorithms for function
minimization will pick very large or ``bad'' values of the
parameters from time to time and the minimization procedure will
generally perform better when a more stable scheme is employed.
\XX{differential equations}{implicit methods}
%// This code does the parameter estimation for the chemical kinetics
%// problem discussed in Bard, Nonlinear parameter estimation, 1974,
%// Academic Press, chapter 8, section 8-7 pages 233--240.
%//
%// Note that when using AUTODIF it is not necessary to integrate
%// the sensitivity equations (page 227-229) to get the derivatives with
%// respect to the model parameters for model fitting. We have used an
%// implicit method for which the values of the function being integrated
%// are nonlinear functions.
%//
%// The observations are contained in a file named "admodel.dat".
%// The initial parameter estimates are in a file named "inpars"
%// but they are read in from the file "admodel.par and then
%// the final parameter estimates are written into a file "admodel.par".

\beginexample
DATA_SECTION
  init_matrix Data(1,10,1,3)  
  init_vector T(1,3) // the initial temperatures for the three runs
  init_vector stepsize(1,3)  // the stepsize to use for numerical integration
  matrix data(1,3,1,10)
  matrix sample_times(1,3,1,10) // times at which reaction was sampled
  vector x0(1,3)           // the beginning time for each of the three
                           // runs
  vector x1(1,3)           // the ending time for each of the three runs
  // for each of the three runs

PARAMETER_SECTION
  init_vector theta(1,10)   // the model parameters
  matrix init_conc(1,3,1,2) // the initial concentrations of the two
                            // reactants over three time periods
  vector instrument(1,2)    // determines the response of the densitometer
  matrix y_samples(1,10,1,2)// the predicted concentrations of the two
                            // reactants at the ten sampling periods
                            // obtained by integrating the differential
                            // equations
  vector diff(1,10)         // the difference between the observed and
                            // readings of the densitometer
  objective_function_value f  // the log_likelihood function
  number bayes_part         // the Bayesian contribution
  number y2
  number x_n
  vector y_n(1,2)
  vector y_n1(1,2)
  number A  // A B C D hold some common subexpressions
  number B
  number C
  number D
PRELIMINARY_CALCS_SECTION
  data=trans(Data);   // it is more convenient to work with the transformed
                   // matrix
PROCEDURE_SECTION

    // set up the begining and ending times for the three runs
    x0(1)=0;
    x1(1)=90;
    x0(2)=0;
    x1(2)=18;
    x0(3)=0;
    x1(3)=4.5;
    // set up the sample times for each of the three runs
    sample_times(1).fill_seqadd(0,10);  // fill with 0,10,20,...,90
    sample_times(2).fill_seqadd(0,2);   // fill with 0,2,4,...,18
    sample_times(3).fill_seqadd(0,0.5); // fill with 0,0.5,1.0,...,4.5

    // set up the initial concentrations of the two reactants for
    // each of the three runs
    init_conc(1,1)=theta(5);
    init_conc(1,2)=theta(6);
    init_conc(2,1)=theta(7);
    init_conc(2,2)=0.0;     // the initial concentrations is known to be 0
    init_conc(3,1)=0.0;     // the initial concentrations is known to be 0
    init_conc(3,2)=theta(8);

    // coefficients which determine the response of the densitometer
    instrument(1)=theta(9);
    instrument(2)=theta(10);
    f=0.0;
    for (int run=1;run<=3;run++)
    {
       // integrate the differential equations to get the predicted
       // values for the y_samples
      int nstep=(x1(run)-x0(run))/stepsize(run);
      nstep++;
      double h=(x1(run)-x0(run))/nstep; // h is the stepsize for integration

      int is=1;
      // get the initial conditions for this run
      x_n=x0(run);
      y_n=init_conc(run);
      for (int i=1;i<=nstep+1;i++)
      {
        // gather common subexpressions
        y2=y_n(2)*y_n(2);
        A=theta(1)*exp(-theta(2)/T(run));
        B=exp(-1000/T(run));
        C=(1+theta(3)*exp(-theta(4)/T(run))*y_n(1));
        C=C*C;
        D=h*A/C;
        // get the y vector for the next time step
        y_n1(1)=(y_n(1)+D*B*y2)/(1.+D);
        y_n1(2)=(y_n(2)+2.*D*y_n(1))/(1.+(2*D*B*y_n(2)));

        // if an observation occurred during this time period save
        // the predicted value
        if (is <=10)
        {
          if (x_n<=sample_times(run,is) && x_n+h >= sample_times(run,is))
          {
            y_samples(is++)=y_n;
          }
        }
        x_n+=h;  // increment the time step
        y_n=y_n1;  // update the y vector for the next step

      }
      diff=(1.0+y_samples*instrument)-data(run); //differences between the
                 // predicted and observed values of the densitometer 
      f+=diff*diff;  // sum of squared differences
    }
    // take the log of f and multiply by nobs/2 to get log-likelihood
    f=15.*log(f); // This is (number of obs)/2. It is wrong in Bard (pg 236).

    // Add the Bayesian stuff
    bayes_part=0.0;
    for  (int i=5;i<=9;i++)
    {
      bayes_part+=(theta(i)-1)*(theta(i)-1); 
    }
    bayes_part+=(theta(10)-2)*(theta(10)-2);
    f+=1./(2.*.05*.05)*bayes_part;

\endexample
\ADM\ produces  a report containing values, standard deviations,
and correlation matrix of the parameter estimates. As discussed below 
any parameter or group of parameters can easily be included in this
report. For models with a large number of parameters this report
can be a bit unwieldly so options are provided to exclude parameters
from the report if desired. 
\htmlbegintex
{ \openup -2pt
\obeylines\obeyspaces\tteight
 index          value   std dev     1     2     3     4     5     6     7     8     9     10   
    1   theta  1.37e+00 2.09e-01     1
    2   theta  1.12e+03 7.70e+01  0.95     1
    3   theta  1.80e+00 7.95e-01   0.9  0.98     1
    4   theta  3.58e+02 1.94e+02  0.91  0.98  0.99     1
    5   theta  1.00e+00 4.49e-02  0.20  0.28  0.12  0.17     1
    6   theta  9.94e-01 2.99e-02 -0.42 -0.35 -0.25 -0.22 -0.58     1
    7   theta  9.86e-01 2.59e-02  0.01  0.22  0.22  0.28  0.26  0.42     1
    8   theta  1.02e+00 1.69e-02 -0.38 -0.34 -0.36 -0.30  0.09  0.63  0.34     1
    9   theta  1.00e+00 2.59e-02 -0.02 -0.23 -0.23 -0.30 -0.28 -0.43 -0.98 -0.37     1
   10   theta  1.97e+00 3.23e-02  0.44  0.37  0.40  0.32 -0.09 -0.65 -0.37  -0.93  0.40    1
}
\htmlendtex

\htmlbeginignore
\def\pr{{\bf[P]}}
\def\HW{{Hilborn and Walters}}
\def\hw{{\bf[HW]}}
\htmlendignore

\htmlnewfile
\mysection{Financial Modelling -- A Generalized Autoregressive Conditional
\hbox{Heteroskedasticity} or GARCH model} 
\X{Financial modelling}
\XX{time series}{GARCH Model}
\X{GARCH Model}
Time series models are often used in financial
modeling. For these models the parameters are often extremely
badly determined. With the stable numerical environment
produced by \ADM\ it is a simple matter to fit
such models.

Consider a time series of returns $r_t$  where $t=0,\ldots,T$,
which are available from some type of financial instrument.
The model assumptions are
    $$ r_t = \mu + \epsilon_t\quad
        h_t = a_0 + a_1\epsilon_{t-1}^2 + a_2h_{t-1}\quad 
\hbox{\rm for }\ 1\le t\le T,\ \ a_0\ge 0,\ \ a_1\ge 0,\ \ a_2\ge 0$$
where the $\epsilon_t$ are independent normally distributed random variables
with mean 0 and variance $h_t$.
We assume $\epsilon_0=0$ and $h_0 = \sum_{i=0}^T(r_i-\bar r)^2/(T+1)$ 
There are four initial parameters to be estimated for this model, 
\hbox{$\mu$, $a_0$, $a_1$, and $a_2$}.
The log-likelihood function for the vector $r_t$ is equal 
to a constant plus
$$-.5\sum_{t=1}^T\big(\,\log(h_t)+(r_t-\mu)^2/h_t\,\big)$$
\beginexample
DATA_SECTION
  init_int T
  init_vector r(0,T)
  vector sub_r(1,T)
  number h0
INITIALIZATION_SECTION
  a0 .1
  a1 .1
  a2 .1
PARAMETER_SECTION
  init_bounded_number a0(0.0,1.0)
  init_bounded_number a1(0.0,1.0,2)
  init_bounded_number a2(0.0,1.0,3)
  init_number Mean
  vector eps2(1,T)
  vector h(1,T)
  objective_function_value log_likelihood
PRELIMINARY_CALCS_SECTION
  h0=square(std_dev(r));   // square forms the element-wise square 
  sub_r=r(1,T);    // form a subvector so we can use vector operations
  Mean=mean(r);    // calculate the mean of the vector r 
PROCEDURE_SECTION
  eps2=square(sub_r-Mean);   
  h(1)=a0+a2*h0;
  for (int t=2;t<=T;t++)
  {
    h(t)=a0+a1*eps2(t-1)+a2*h(t-1);
  }
  // calculate minus the log-likelihood function
  log_likelihood=.5*sum(log(h)+elem_div(eps2,h));  // elem_div performs  
                                // element-wise division of vectors
RUNTIME_SECTION
  convergence_criteria .1, .1, .001
  maximum_function_evaluations 20, 20, 1000
\endexample
\X{vector operations}
\X{element-wise operations}
We have used vector operations such as {\tt elem\_div} and {\tt sum}
to simplify the code. Of course the code could also have employed
loops and element-wise operations. The parameter values
and standard deviation report for this model appears below. 
\X{standard deviation report}
\beginexample
 index         value      std dev       1       2       3       4   
    1   a0    1.6034e-04 2.3652e-05  1.0000
    2   a1    9.3980e-02 2.0287e-02  0.1415  1.0000
    3   a2    3.7263e-01 8.2333e-02 -0.9640 -0.3309  1.0000
    4   Mean -1.7807e-04 3.0308e-04  0.0216 -0.1626  0.0144  1.0000
\endexample
This example employs bounded initial parameters.
Often it is necessary to put bounds on parameters in nonlinear
modeling to ensure that the minimization is stable.
In this example {\tt a0} is constrained to lie between $0.0$ and $1.0$
\X{putting bounds on initial parameters}
\X{init\_bounded\_number}
\beginexample
  init_bounded_number a0(0.0,1.0)
  init_bounded_number a1(0.0,1.0,2)
  init_bounded_number a2(0.0,1.0,3)
\endexample
\mysection{Carrying out the minimization in a number of phases}
For linear models one can simply estimate all the model parameters
simultaneously. For nonlinear models often this simple approach does 
not work very well. It may be necessary to keep some of the
parameters fixed during the initial part of the minimization 
process and carry out the minimization over a subset of the parameters.
The other parameters are included into the minimization
process in a number of phases until all of the parameters have
been included. \ADM\ provides support for this multi-phase
approach. In the declaration of any initial parameter the last number,
if present, determines the phase of the minimization 
during which this parameter is included (becomes active). If
no number is present the initial parameter 
becomes active in phase~1.  
In this case {\tt a0} has no phase number and so becomes
active in phase~1. {\tt a1} becomes active in phase~2, and
{\tt a2} becomes active in phase~3. In this example
phase~3 is the last phase of the optimization.

It is often convenient to modify aspects of the code depending on which 
phase of the minimization procedure is the current phase or on whether
a particular initial parameter is active. The function
\X{current\_phase() function}
\X{last\_phase() function}
\X{active() function}
\X{sd\_phase() function}
\beginexample
current_phase()
\endexample
\noindent returns an integer (object of type {\tt int}) which is the
value of the current phase.
The function
\beginexample 
last_phase()
\endexample
\noindent returns the value ``true'' ($\ne 0$) if the current phase is the
last phase and false ($=0$) otherwise. 
If {\tt xxx} is the name of any initial parameter the function
\beginexample 
active(xxx) 
\endexample
\noindent returns the value ``true'' if {\tt xxx} is active
during the  current phase and false otherwise.

After the minimization of the objective function has been completed
\ADM\ calculates the estimated covariance matrix for the
initial parameters as well as any other desired parameters which
have been declared to be of {\tt sd\_report} type. Often 
these additional parameters may involve
considerable additional computational overhead. If the values of 
these parameters are not used in calculations proper, it is possible to
only calculate them during the standard deviations report phase.
\beginexample 
sd_phase()
\endexample
\noindent The {\tt sd\_phase} function returns the value ``true'' if
we are in the standard deviations report phase and ``false''
otherwise. It can be used in a conditional statement to determine
whether to perform calculations associated with some 
{\tt sd\_report} object. 
\X{multi-phase minimization}
\X{default number of function evaluations}
When  estimating the parameters of a model by a multi-phase
minimization procedure the default behavior of \ADM\ is to
carry out the default number of function evaluations until 
convergence is achieved in each stage. If we are only interested
in the parameter estimates obtained after the last stage of
the minimization it is often not necessary to carry out the 
full minimization in each stage. Sometimes considerable time can be
saved by relaxing the convergence criterion in the
initial stages of the optimization.
The {\tt RUNTIME\_SECTION} allows the user to modify
the default behavior of the function minimizer during the 
phases of the estimation process.
\X{convergence criterion}
\XX{function minimizer}{default behaviour}
\XX{function minimizer}{modifying default behaviour}
\X{RUNTIME\_SECTION}
\XX{SECTIONS}{\tt RUNTIME\_SECTION}
\beginexample
RUNTIME\_SECTION
  convergence_criteria .1, .1, .001
  maximum_function_evaluations 20, 20, 1000
\endexample
\noindent The {\tt convergence\_criteria} affects the criterion used by
the function minimizer to decide when the optimization process has 
occurred.  The function minimizer compares the maximum value of the
vector of derivatives of the objective function with respect to the
independent variables to the numbers after the
{\tt convergence\_criteria} keyword. The first number is used in the
first phase of the optimization, the second number in the second phase
and so on. If there are more phases to the optimization than
there are numbers the last number is used for the rest of the
phases of the optimization.
The numbers must be separated by commas. The spaces are optional.
\X{{\tt maximum\_function\_evaluations}} 
The {\tt maximum\_function\_evaluations} keyword controls the 
maximum number of evaluations of the
objective function which will be performed by the function minimizer
in each stage of the minimization procedure.
\XX{optimization}{phases}
\htmlnewfile
\mysection{Natural resource management -- the Schaeffer-Pella-Tomlinson Model}
It is typical of many models in
natural resource management that the model tends to be
rather unstable numerically and in addition some of the model
parameters are often poorly determined. Notwithstanding these
difficulties it is often necessary to make decisions
about resource management based on the analysis provided by
these models. The example
provides a good opportunity for presenting some more advanced
features of \ADM\ which are designed to
overcome these difficulties.

\X{Schaeffer -- Pella-Tomlinson model}
\X{fisheries management}
The Schaeffer -- Pella-Tomlinson model is employed in 
fisheries management. The model assumes that the total
biomass of an exploited fish stock satisfies an
ordinary differential equation of the form
\begin{equation}
\myeq{{\displaystyle dB\over\displaystyle dt}=rB\bigg(\,1-\Big(\myfrac{B}{k}\Big)^{m-1}\,\bigg) 
           - FB  \qquad \hbox{\rm where}\quad m>1}
\label{chp3:xx3}
\end{equation} 
(Hilborn and Walters page 303)  where
$B$ is the biomass, $F$ is the instantaneous fishing mortality rate,
 $r$ is a parameter often 
referred to an the intrinsic rate of increase, $k$ is the
unfished  equilibrium stock size,   
\begin{equation}
\myeq{C=FB}
\label{chp3:xx4}
\end{equation} 
is the catch rate, and $m$ is a parameter which determines
where the maximum productivity of the stock occurs. 
If the value of $m$ is fixed at 2 the model is referred to as the
Schaeffer model.
The explicit form of the difference equation corresponding to
this differential equation is
\XX{difference equation}{explicit form}
\XX{difference equation}{semi-implicit form}
\XX{difference equation}{numerical stability}
\begin{equation}
\myeq{B_{t+\delta}=B_t+rB_t\delta-rB_t\Big(\myfrac{B_t}{k}\Big)^{m-1}\delta 
     - F_tB_t\delta}
  \label{chp3:xx5}
\end{equation} 
To get a semi-implicit form of this difference equation which has
better numerical stability than the explicit version
we replace some of the terms $B_t$ on the
right hand side  of \ref{chp3:xx5} by $B_{t+\delta}$ to get
\begin{equation}
\myeq{B_{t+\delta}
   =B_t+rB_t\delta-rB_{t+\delta}\Big(\myfrac{B_t}{k}\Big)^{m-1}\delta 
     - F_tB_{t+\delta}\delta}
 \label{chp3:xx6}
\end{equation} 
and solve for $B_{t+\delta}$ to give
\begin{equation}
\myeq{B_{t+\delta}
   =\myfrac{B_t(1+r\delta)}{1+\big(\,r(B_t/k)^{m-1}+F_t\,\big)\delta}}
  \label{chp3:xx7}
\end{equation} 
The catch $C_{t+\delta}$ over the period $(t,t+\delta)$
is given by
\begin{equation}
 \myeq{C_{t+\delta} =F_tB_{t+\delta}\delta }
  \label{chp3:xx8}
\end{equation} 
\XX{Schaeffer -- Pella-Tomlinson model}{Bayesian considerations}
\XX{Schaeffer -- Pella-Tomlinson model}{unfished equilibrium biomass level}
\XX{Schaeffer -- Pella-Tomlinson model}{optimal productivity}
\mysection{Bayesian considerations in the Pella--Tomlinson model}
The parameter $k$ is referred to as the carrying capacity or
the unfished equilibrium biomass level because it is the value
that the biomass of the population will eventually assume if there is no
fishing.  For a given value of $k$ the parameter $m$ determines
the level of maximum productivity, that is the level of biomass~$\bmax$
for which the removals from fishing can be the greatest. 
$$\bmax=\myfrac{k}{\root {m-1} \of m}$$

For $m=2$ maximum productivity is obtained by that level of fishing 
pressure which reduces the stock to 50\% of the carrying capacity. 
For  the data available in many real fisheries problems the parameter
$m$ is very poorly determined. It is common practice therefore to
simply assume that $m=2$. Similarly, it is commonly assumed that
the carrying capacity $k$ does not change over time even though 
changes such as habitat degradation may well lead to changes 
in~$k$. 

We want to construct a statistical model where the carrying capacity 
can be varying slowly over time if there appears to be any information
in the fisheries data supporting this hypothesis. What is meant by slowly?
The answer to this question will depend on the particular situation.
For our purposes slowly means slowly enough so that the model
has some chance of supplying a useful analysis of the situation
at hand. We refer to this as the assumption of manageability.
The point is that since we are going to use this model anyway to try and
mange a resource we may as well assume that the model's
assumptions are satisfied at least well enough so that we have some
hope of success. 
This may seem extremely arbitrary, and it is. However it is not
as arbitrary as assuming that the carrying capacity is constant. 

We assume that $k_{i+1}=k_i \exp(\kappa_i)$ where the $\kappa_i$
are independent normally distributed random variables with mean~$0$.
and that $\log(m-1)$ is normally distributed with mean~0.
The parameters $\log(k)$  are assumed to have the structure of a random walk
which is the simplest type of time series. This Bayesian approach
is a very simple method for including time series structure into
the parameters of a nonlinear model.

\def\cobs{C_i^{\rmsmall obs}}
We don't know the true catches $C_i$ in each year. What we have are
estimates $\cobs$ of the catch. We assume that the
quantities $\log(\cobs/C_i)$ are normally distributed with mean~0.

Finally we must deal with the fishing mortality $F$.  Estimates of
$F$ are not obtained directly. Instead what is observed is an index
of fishing mortality, in this case fishing effort. We assume that for each
year we have an estimate $E_i$ of fishing effort and that
the fishing mortality rate $F_i$ in year $i$ Satisfies the relationship
$F_i=qE_i\exp(\eta_i)$ where $q$ is a parameter referred to as the 
catchability and  the $\eta_i$ are normally distributed random
variables with mean 0.
 
We assume that the variance of the $\eta_i$ is 10 times the 
variance in the observed catch errors and that the
variance of the $\kappa_i$ is 0.1 times the variance in the
observed catch errors. We assume that the variance in
$\log(m-1)$ is 0.25. Then given the data,
the Bayesian posterior distribution for the model parameters is
proportional to 
\begin{equation}
 \myeq{-(3n-1)/2\log\Big(\,\sum_{i=1}^n \big(\log(\cobs)-\log(C_i)\,\big)^2
      +.1\sum_{i=1}^n\eta_i^2  +10\sum_{i=2}^n\kappa_i^2\Big)
      -2.\log(m-1)^2 }    
  \label{ch1:xx7}
\end{equation} 
     
The number of initial parameters in the model (that is the number of
independent variables  in the function to be minimized)  is
$2n+4$. For the halibut data there are $56$ years of data
which gives $116$ parameters.  As estimates of the model
parameters we use the mode of the posterior distribution
which can by found by minimizing -1 times expression
(\number\mychapno.8). The covariance matrix of the model
parameters are estimated by computing the
inverse of the hessian of expression  (\number\mychapno.8)
at the minimum. The template for the model follows. 
To improve the readability the entire template has been
included. The various sections are discussed below.
\X{init\_bounded\_vector}
\X{init\_bounded\_dev\_vector}
\X{sdreport\_number}
\X{sdreport\_vector}
\X{sdreport\_matrix}
\X{active parameters}
\beginexample
DATA_SECTION
  init_int nobs;
  init_matrix data(1,nobs,1,3)
  vector obs_catch(1,nobs);
  vector cpue(1,nobs);
  vector effort(1,nobs);
  number avg_effort
INITIALIZATION_SECTION
  m 2.
  beta 1.
  r 1.
PARAMETER_SECTION
  init_bounded_number q(0.,1.)
  init_bounded_number beta(0.,5.)
  init_bounded_number r(0.,5,2)
  init_number log_binit(2)
  init_bounded_dev_vector effort_devs(1,nobs,-5.,5.,3)
  init_bounded_number m(1,10.,4)
  init_bounded_vector k_devs(2,nobs,-5.,5.,4)
  number binit
  vector pred_catch(1,nobs)
  vector biomass(1,nobs)
  vector f(1,nobs)
  vector k(1,nobs)
  vector k_trend(1,nobs)
  sdreport_number k_1
  sdreport_number k_last
  sdreport_number k_change
  sdreport_number k_ratio
  sdreport_number B_projected
  number tmp_mort;
  number bio_tmp;
  number c_tmp;
  objective_function_value ff;
PRELIMINARY_CALCS_SECTION
  // get the data out of the data matrix into 
  obs_catch=column(data,2);  
  cpue=column(data,3);  
  // divide the catch by the cpue to get the effort
  effort=elem_div(obs_catch,cpue);
  // normalize the effort and save the average
  double avg_effort=mean(effort);
  effort/=avg_effort;
  cout << " beta" << beta << endl;
PROCEDURE_SECTION
  // calculate the fishing mortality
  calculate_fishing_mortality();
  // calculate the biomass and predicted catch
  calculate_biomass_and_predicted_catch();
  // calculate the objective function
  calculate_the_objective_function();

FUNCTION calculate_fishing_mortality
  // calculate the fishing mortality
  f=q*effort;
  if (active(effort_devs)) f=elem_prod(f,exp(effort_devs));

FUNCTION calculate_biomass_and_predicted_catch
  // calculate the biomass and predicted catch
  if (!active(log_binit))
  {
    log_binit=log(obs_catch(1)/(q*effort(1)));
  }
  binit=exp(log_binit);
  biomass[1]=binit;
  if (active(k_devs))
  {
    k(1)=binit/beta;
    for (int i=2;i<=nobs;i++)
    {
      k(i)=k(i-1)*exp(k_devs(i));
    }
  }
  else
  {
    // set the whole vector equal to the constant k value
    k=binit/beta;
  }
  // only calculate these for the standard deviation report
  if (sd_phase)
  {
    k_1=k(1);
    k_last=k(nobs);
    k_ratio=k(nobs)/k(1);
    k_change=k(nobs)-k(1);
  }
  // two time steps per year desired
  int nsteps=2;
  double delta=1./nsteps;
  // Integrate the logistic dynamics over n time steps per year
  for (int i=1; i<=nobs; i++)
  {
    bio_tmp=1.e-20+biomass[i];
    c_tmp=0.;
    for (int j=1; j<=nsteps; j++)
    {
      //This is the new biomass after time step delta
      bio_tmp=bio_tmp*(1.+r*delta)/
        (1.+ (r*pow(bio_tmp/k(i),m-1.)+f(i))*delta );
      // This is the catch over time step delta
      c_tmp+=f(i)*delta*bio_tmp;
    }
    pred_catch[i]=c_tmp;        // This is the catch for this year
    if (i<nobs)
    {
      biomass[i+1]=bio_tmp;// This is the biomass at the begining of the next
    }                      // year
    else
    {
      B_projected=bio_tmp; // get the projected biomass for std dev report
    }
  }

FUNCTION calculate_the_objective_function
  if (!active(effort_devs))
  {
    ff=nobs/2.*log(norm2(log(obs_catch)-log(1.e-10+pred_catch)));
  }
  else if(!active(k_devs))
  {
    ff= .5*(size_count(obs_catch)+size_count(effort_devs))*
      log(norm2(log(obs_catch)-log(1.e-10+pred_catch))
      +0.1*norm2(effort_devs));
  }
  else 
  {
    ff= .5*( size_count(obs_catch)+size_count(effort_devs)
      +size_count(k_devs) )*
      log(norm2(log(obs_catch)-log(pred_catch))
      + 0.1*norm2(effort_devs)+10.*norm2(k_devs));
  }
  // Bayesian contribution for Pella Tomlinson m
  ff+=2.*square(log(m-1.));
  if (current_phase()<3)
  {
    ff+=1000.*square(log(mean(f)/.4));
  }

\endexample
The data are contained in three
columns, with the catch and catch per unit effort data contained
in the second and third columns. The matrix {\tt data} is
defined in order to read the data. The second and third columns 
of {\tt data} which we are interested in will then be put into
the vectors {\tt obs\_catch} and {\tt cpue}. (Later we get the fishing
effort by dividing the {\tt obs\_catch} by the {\tt cpue}.)
\beginexample
DATA_SECTION
  init_int nobs
  init_matrix data(1,nobs,1,3)
  vector obs_catch(1,nobs)
  vector cpue(1,nobs)
  vector effort(1,nobs)
  number avg_effort
\endexample
The \IS\ is used to define default values for
some model parameters if the standard default provided by 
\ADM\ is not acceptable. 
If the model finds the parameter file (whose default name
is {\tt admodel.par}) it will read in the initial values for
the parameters from there. Otherwise the default values will be used
unless the parameters appear in the \IS\ in which case those 
values will be used. 
\beginexample
INITIALIZATION_SECTION
  m 2.
  beta 1.
  r 1.
\endexample
\X{init\_bounded\_number}
The \PS\ for this model  introduces  several new features of \ADM.
The statement \hbox{\tt init\_bounded\_number r(0.,5.,2)}
declares an initial parameter whose value will be constrained to lie
between 0.0 and 5.0. It is often necessary to put bounds on
the initial parameters in nonlinear models to 
get stable model performance. This is accomplished in
\ADM\ simply by declaring the initial parameter to
be bounded and providing the desired bounds. 
The default initial value for a bounded object is
the average of the lower and upper bounds.

 The third number 2 in the declaration determines that
this initial parameter will not be made active until the second
phase of the minimization. This introduces the
concept of phases in the minimization process. 

\XX{optimization}{phases}
\X{multi-phase minimization}
\XX{minimization}{phases}
As soon as nonlinear statistical models become a bit
complicated one often finds that simply attempting to
estimate all the parameters simultaneously does not
work very well. In short ``you can't get there from here". 
A better strategy is to keep some of the
parameters fixed and to first minimize the function 
with respect to the other parameters. More 
parameters are added in a stepwise relaxation process.
In \ADM\ each step of this relaxation process is termed a phase.
The parameter {\tt r} is not allowed to vary until
the second phase. Initial parameters which are allowed to
vary will be termed active. In the first phase
the active parameter are {\tt beta} and
{\tt q}. The default phase for an initial parameter
is phase 1 if no phase number is included in its declaration. 
The phase
number for an initial parameter is the last number in the
declaration for that parameter. The general order for the arguments
int the definition of
any initial parameter is the size data for
a vector or matrix object if needed, the bounds 
for a bounded object if needed, followed by
the phase number if desired. 

It is often a difficult problem to decide what the order of
relaxation for the initial parameters should be. This must
sometimes be done by trial and error. However\ADM\
 makes the process a lot simpler. One only needs to
change the phase numbers of the initial parameters
int the \PS\ and recompile the program.

Often in statistical modeling it is useful to regard a vector
of quantities {$x_i$} as consisting of an overall mean, $\mu$,
and a set of deviations from that mean, $\delta_i$, so that
$$x_i=\mu+\delta_i \qquad\hbox{\rm where}\quad\sum_i \delta_i=0 $$ 
\ADM\ provides support for this modeling construction with
the\break 
 {\tt init\_bounded\_dev\_vector} declaration. The components of
an object created by this declaration will automatically sum to 0
without any user intervention. 
The line 
\beginexample
  init_bounded_dev_vector effort_devs(1,nobs,-5.,5.,3)
\endexample
\noindent declares  effort\_devs to be this kind of object.
The bounds {\tt -5.,5.}
control the range of the deviations. Putting reasonable bounds
on such deviations often improves the stability of the
estimation procedure.

\ADM\ has {\tt sdreport\_number}, {\tt sdreport\_vector},
and {\tt sdreport\_matrix} declarations in the \PS.
 These objects behave the same as 
{\tt number}, {\tt vector}, and {\tt matrix} objects
with the additional property that they
are included in the report of the estimated standard
deviations and correlation matrix.

For example merely by including the
statement {\tt sdreport\_number B\_projected}
one can obtain the estimated standard deviation of
the biomass projection for the next year. 
(Of course you must also set {\tt B\_projected}
equal to the projected biomass. This is done in the \PROS.)
\beginexample
PARAMETER_SECTION
  init_bounded_number q(0.,1.)
  init_bounded_number beta(0.,5.)
  init_bounded_number r(0.,5,2)
  init_number log_binit(2)
  init_bounded_dev_vector effort_devs(1,nobs,-5.,5.,3)
  init_bounded_number m(1,10.,4)
  init_bounded_vector k_devs(2,nobs,-5.,5.,4)
  number binit
  vector pred_catch(1,nobs)
  vector biomass(1,nobs)
  vector f(1,nobs)
  vector k(1,nobs)
  vector k_trend(1,nobs)
  sdreport_number k_1
  sdreport_number k_last
  sdreport_number k_change
  sdreport_number k_ratio
  sdreport_number B_projected
  number tmp_mort;
  number bio_tmp;
  number c_tmp;
  objective_function_value ff;
\endexample

The \PCS\ carries out a few simple operations on the data.
The model expects to have catch and effort data, but the input
file contained catch and cpue (catch/effort) data.
We divide the catch data by the cpue data to get the effort
data. The \AD\ operation {\tt elem\_div} which performs
element-wise divisions of vector objects is used.
As usual the same thing could have been accomplished
by employing a loop and writing element-wise code. 
The effort data are then normalized, that is, they are divided by
their average so that their average becomes 1. This is done so that
we have a good idea what the catchability parameter {\tt q}
should be to give reasonable values for the fishing mortality
rate (since $F=qE$). 

Notice that the \PCS\ section is \cplus\ code so that statements 
must be ended with a {\tt ;}.
\X{column}{extract a column from a matrix}
\beginexample
PRELIMINARY_CALCS_SECTION
  // get the data out of the data matrix into 
  obs_catch=column(data,2);  
  cpue=column(data,3);  
  // divide the catch by the cpue to get the effort
  effort=elem_div(obs_catch,cpue);
  // normalize the effort and save the average
  double avg_effort=mean(effort);
  effort/=avg_effort;
\endexample
The \PROS\ contains several new \ADM\ features. Some have to do with
the notion of carrying out the minimization in a number of steps
or phases. The line
\X{multi-phase minimization}
\XX{minimization}{phases}
\beginexample
  if (active(effort_devs)) f=elem_prod(f,exp(effort_devs));
\endexample
\X{active() function}
introduces the {\tt active} function. This function can be used on
any initial parameter and will return a value ``true'' if
that parameter is active in the current phase. The idea here is that
if the initial parameters {\tt effort\_devs} are not active then
since their value is 0 carrying out the calculations will have no
effect and we can save time by avoiding the calculations. 
The {\tt active} function is also used in the statement
\beginexample
  if (!active(log_binit))
  {
    log_binit=log(obs_catch(1)/(q*effort(1)));
  }
\endexample
The idea is that if the {\tt log\_binit} initial parameter 
(this is the logarithm of the biomass  at the beginning of the first year)
is not active then we set it equal to the value which produces the
observed catch (using the relationship $C=qEB$ so that 
$B=C/(qE)$. 
The {\tt active} function is also used in the calculations of
the objective function so that unnecessary calculations are
avoided.

The following code helps to deal with convergence problems in this
type of nonlinear model. The problem is that the starting
parameter values are often so bad that the optimization
procedure will try to make the population very large
and the exploitation rate very small because this is
the best local solution near the starting
 parameter values. To circumvent this 
problem we include a penalty function to keep the average
value of the fishing mortality rate {\tt f} close to 0.2
during the first two phases of the minimization. 
In the final phase the size of 
the penalty term is reduced to a very small value. The function
\hbox{\tt current\_phase()} returns the value of the current phase
of the minimization. 
\X{current\_phase() function}
\beginexample
 if (current_phase()<3)
  {
    ff+=1000.*square(log(mean(f)/.4));
  }
\endexample
\X{SUBROUTINE}
\X{FUNCTION}
\mysection{Using FUNCTIONS to improve code organization}
Subroutines or functions are used to improve the organization of the code. 
The code for the main part of the
\PROS\ which invokes the {\tt FUNCTIONS} should be placed at the 
top of the \PROS. 
\beginexample
PROCEDURE_SECTION
  // calculate the fishing mortality
  calculate_fishing_mortality();
  // calculate the biomass and predicted catch
  calculate_biomass_and_predicted_catch();
  // calculate the objective function
  calculate_the_objective_function();
\endexample
\X{PROCEDURE\_SECTION}
There are three user-defined functions called at the beginning of the
{\tt PROCEDURE\_SECTION}
The code to define the {\tt FUNCTIONS} comes next.
To define a function whose name is {\tt name} the template directive 
{\tt FUNCTION name} is used. Notice that no parentheses {\tt ()} 
are used in the
definition of the function, but to call the function the 
statement takes the form \hbox{\tt name();}

\htmlnewfile
\X{FUNCTION keyword} 
\X{Fisheries catch-at-age model}
\XX{Bayesian inference}{profile likelihood}
\XX{command line arguments}{-lprof profile likelihood}
\mysection{A fisheries catch-at-age model}
This section describes a simple catch-at age model. The data input to this
model include estimates of the numbers at age caught by the fishery 
each year and  estimates the fishing effort each year. This example introduces
\ADM's ability to automatically calculate profile likelihoods
for carrying out Bayesian inference. To cause the profile likelihood calculations
to be carried out use the -lprof command line argument.

Let $i$ index fishing years $1\le i\le n$ and $j$ index age classes
with $1\le j\le r$.
The instantaneous fishing mortality rate is assumed to have the 
form $F_{ij}=qE_is_j\exp(\delta_i)$ where $q$ is called the catchability,
$E_i$ is the observed fishing effort, $s_j$ is an age-dependent effect
termed the selectivity, and the $\delta_i$ are deviations from the
expected relationship between the observed fishing effort and
the resulting fishing mortality. The $\delta_i$ are assumed to
be normally distributed with mean 0. The instantaneous
natural mortality rate $M$ is assumed to be independent of
year and age class. It is not estimated in this version of the model.
The instantaneous total
mortality rate is given by $Z_{ij}=F_{ij}+M$. The survival rate is given
by $S_{ij}=\exp(-Z_{ij})$. The number of age class $j$ fish
in the population in year $i$ is denoted by $N_{ij}$.
The relationship $N_{i+1,j+1}=N_{ij}S_{ij}$ is assumed to hold.
Note that using this relationship if one knows $S_{ij}$ then all
the $N_{ij}$ can be calculated from knowledge of the 
initial population in year $1$, $N_{11}, N_{12},\ldots, N_{1r}$
and knowledge of the recruitment in each year $N_{21}, N_{31},\ldots
N_{n1}$.

The purpose of the model is to estimate quantities of interest to
managers such as the population size and exploitation rates
and to make projections about the population. In particular we can get 
an estimate of the numbers of fish in the population in year $n+1$
for age classes 2 or greater from the relationship
$N_{n+1,j+1}=N_{nj}S_{nj}$. If we have estimates $m_j$ for the mean weight
at age $j$, then the projected biomass level $B_{n+1}$ of age class $2+$
fish for year $n+1$ can be computed from the relationship
$B_{n+1}=\sum_{j=2}^r m_jN_{n+1,j}$.

Besides getting a point estimate for quantities of interest like
$B_{n+1}$ we also want to get an idea of how well determined the
estimate is. \ADM\ has completely automated the process of deriving
good confidence limits for these parameters in a Bayesian context.
One simply needs to declare the parameter to be of type
{\tt likeprof\_number}. The  results are given in the section
on Bayesian inference.

The code for the catch-at-age model is: 
\beginexample
DATA_SECTION
  // the number of years of data
  init_int nyrs  
  // the number of age classes in the population
  init_int nages
  // the catch-at-age data
  init_matrix obs_catch_at_age(1,nyrs,1,nages)
  //estimates of fishing effort
  init_vector effort(1,nyrs)
  // natural mortality rate
  init_number M
  // need to have relative weight at age to calculate biomass of 2+
  vector relwt(2,nages)
INITIALIZATION_SECTION
  log_q -1 
  log_P 5
PARAMETER_SECTION
  init_number log_q(1)   // log of the catchability
  init_number log_P(1)  // overall population scaling parameter
  init_bounded_dev_vector log_sel_coff(1,nages-1,-15.,15.,2)
  init_bounded_dev_vector log_relpop(1,nyrs+nages-1,-15.,15.,2)
  init_bounded_dev_vector effort_devs(1,nyrs,-5.,5.,3)
  vector log_sel(1,nages)
  vector log_initpop(1,nyrs+nages-1);
  matrix F(1,nyrs,1,nages)   // the instantaneous fishing mortality
  matrix Z(1,nyrs,1,nages)   // the instantaneous total mortality
  matrix S(1,nyrs,1,nages)   // the survival rate
  matrix N(1,nyrs,1,nages)   // the predicted numbers at age
  matrix C(1,nyrs,1,nages)   // the predicted catch at age 
  objective_function_value f
  sdreport_number avg_F
  sdreport_vector predicted_N(2,nages)
  sdreport_vector ratio_N(2,nages)
  likeprof_number pred_B
PRELIMINARY_CALCS_SECTION
  // this is just to invent some relative average
  // weight numbers
  relwt.fill_seqadd(1.,1.);
  relwt=pow(relwt,.5);
  relwt/=max(relwt);
PROCEDURE_SECTION
  // example of using FUNCTION to structure the procedure section
  get_mortality_and_survival_rates();

  get_numbers_at_age();

  get_catch_at_age();

  evaluate_the_objective_function();

FUNCTION get_mortality_and_survival_rates
  // calculate the selectivity from the sel_coffs
  for (int j=1;j<nages;j++)
  {
    log_sel(j)=log_sel_coff(j);
  }
  // the selectivity is the same for the last two age classes
  log_sel(nages)=log_sel_coff(nages-1);

  // This is the same as F(i,j)=exp(log_q)*effort(i)*exp(log_sel(j));
  F=outer_prod(mfexp(log_q)*effort,mfexp(log_sel));
  if (active(effort_devs))
  {
    for (int i=1;i<=nyrs;i++)
    {
      F(i)=F(i)*exp(effort_devs(i));
    }
  }
  // get the total mortality
  Z=F+M;
  // get the survival rate
  S=mfexp(-1.0*Z);

FUNCTION get_numbers_at_age
  log_initpop=log_relpop+log_P;
  for (int i=1;i<=nyrs;i++)
  {
    N(i,1)=mfexp(log_initpop(i));
  }
  for (int j=2;j<=nages;j++)
  {
    N(1,j)=mfexp(log_initpop(nyrs+j-1));
  }
  for (i=1;i<nyrs;i++)
  {
    for (j=1;j<nages;j++)
    {
      N(i+1,j+1)=N(i,j)*S(i,j);
    }
  }
  // calculated predicted numbers at age for next year
  for (j=1;j<nages;j++)
  {
    predicted_N(j+1)=N(nyrs,j)*S(nyrs,j);
    ratio_N(j+1)=predicted_N(j+1)/N(1,j+1);
  }
  // calculate predicted biomass for profile
  // likelihood report
  pred_B=predicted_N *relwt;

FUNCTION get_catch_at_age
  C=elem_prod(elem_div(F,Z),elem_prod(1.-S,N));

FUNCTION evaluate_the_objective_function
  // penalty functions to ``regularize '' the solution
  f+=.01*norm2(log_relpop);
  avg_F=sum(F)/double(size_count(F));
  if (last_phase())
  {
    // a very small penalty on the average fishing mortality
    f+= .001*square(log(avg_F/.2));
  }
  else
  {
    // use a large penalty during the initial phases to keep the
    // fishing mortality high
    f+= 1000.*square(log(avg_F/.2));
  }
  // errors in variables type objective function with errors in
  // the catch at age and errors in the effort fishing mortality
  // relationship
  if (active(effort_devs)
  {
    // only include the effort_devs in the objective function if
    // they are active parameters
    f+=0.5*double(size_count(C)+size_count(effort_devs))
      * log( sum(elem_div(square(C-obs_catch_at_age),.01+C))
      + 0.1*norm2(effort_devs));
  }
  else
  {
    // objective function without the effort_devs
    f+=0.5*double(size_count(C))
      * log( sum(elem_div(square(C-obs_catch_at_age),.01+C)));
  }
REPORT_SECTION
  report << "Estimated numbers of fish " << endl;
  report << N << endl; 
  report << "Estimated numbers in catch " << endl;
  report << C << endl; 
  report << "Observed numbers in catch " << endl;
  report << obs_catch_at_age << endl; 
  report << "Estimated fishing mortality " << endl;
  report << F << endl; 
\endexample
\XX{REPORT\_SECTION}{example of}
\XX{SECTIONS}{\tt REPORT\_SECTION}
This model employs several instances of the
{\tt init\_bounded\_dev\_vector} type.
This type consists of a vector of numbers which
sum to 0, that is they are deviations from a common mean,
and are bounded. For example the quantities {\tt log\_relpop}
are used to parameterize the logarithm of the variations in
year class strength of the fish population. Putting bounds on
the magnitude of the deviations helps to improve the stability of the model.
The bounds are from -15.0 to 15.0 which gives the estimates of
relative year class strength a dynamic range of $\exp(30.0)$. 

\X{FUNCTION}
The {\tt FUNCTION} keyword has been employed a number of times
in the 
\PS\ to help structure the code. 
A function is defined simply by using the {\tt FUNCTION } keyword
followed by the name of the function.
\beginexample
FUNCTION get_mortality_and_survival_rates
\endexample
\noindent Don't include the parentheses or semicolon here.
To use the function simply write its name in the procedure
section. 
\beginexample
  get_mortality_and_survival_rates();
\endexample
\noindent You must include the parentheses and the semicolon here.

The {\tt REPORT\_SECTION} shows how to generate a report
for an \ADM\ program.  The default report generating machinery
utilizes the C++ stream formalism. You don't need to know much
about streams to make a report, but a few comments are in order. 
The stream formalism associates stream object with
a file. In this case the stream object associated with
the \ADM\ report file is {\tt report}. To write an object
{\tt xxx} into the report  file you insert the line 
\beginexample
  report << xxx;
\endexample
\noindent into the {\tt REPORT\_SECTION}.
If you want to skip to a new line after writing the object you can 
include the stream manipulator {\tt endl} as in 
\beginexample
  report << "Estimated numbers of fish " << endl;
\endexample
\noindent Notice that the stream operations know about
common C objects such as strings, so that it is a simple matter
to put comments or labels into the report file.
\X{endl stream manipulator} 
\X{profile likelihood}
\XX{profile likelihood}{confidence limits}
\X{confidence limits}
\X{likeprof\_number}
\XX{confidence limits}{use of profile likelihood}
\mysection{Bayesian inference and the \apl}

\ADM\ enables one to quickly build models with large numbers of
parameters -- this is especially useful for employing
Bayesian analysis. Traditionally however it has been difficult to
interpret the results of analysis using such models.
In a Bayesian context the results are represented
by the posterior probability distribution for the
model parameters. To get exact results from
the posterior distribution it is necessary to evaluate
integrals over large dimensional spaces and this can be computationally
intractable.  \ADM\ provides an 
approximations to these
integrals in the form of the profile likelihood. The
profile likelihood can be used to 
estimates for extreme values 
(such as estimating a value $\beta$ so that for a 
parameter $b$ the probability that $b<\beta \approx 0.10$
or the probability that $b>\beta \approx 0.10$) for
any model parameter.
To use this facility simply declare the parameter of interest
to be of type {\tt likeprof\_number} in the \PS\ and assign
the correct value to the parameter in the \PROS.

The code for the catch at age model estimates
the \apl\ for the projected biomass
of age class 2+ fish. (Age class 2+ has been used to avoid the
extra problem of dealing with the uncertainty of the
recruitment of age class 1 fish). As a typical application
of the method, the user of the model can
estimate the probability that the biomass 
of fish for next year will be larger or smaller than a certain
value. Estimates like these are
obviously of great interest to managers of natural resources.  

The profile likelihood report for a variable is in a file
with the same name as the variable (truncated to eight letters,
if necessary, with the suffix {\tt .PLT} appended). For this example
the report is in the file {\tt PRED\_B.PLT}. Part of the file is shown here.  
\beginexample
pred_B:
Profile likelihood
 -1411.23 1.1604e-09
 -1250.5 1.71005e-09
 -1154.06 2.22411e-09
  ...................      // skip some here
  ...................
 278.258 2.79633e-05
 324.632 5.28205e-05
 388.923 6.89413e-05
 453.214 8.84641e-05
 517.505 0.0001116
 581.796 0.000138412
  ...................
  ...................
 1289 0.000482459
 1353.29 0.000494449
 1417.58 0.000503261
 1481.87 0.000508715
 1546.16 0.0005107
 1610.45 0.000509175
 1674.74 0.000504171
 1739.03 0.000490788
 1803.32 0.000476089
 1867.61 0.000460214
 1931.91 0.000443313
 1996.2 0.000425539
 2060.49 0.000407049
 2124.78 0.000388
 2189.07 0.00036855
  ...................
  ...................
 4503.55 2.27712e-05
 4599.98 2.00312e-05
 4760.71 1.48842e-05
 4921.44 1.07058e-05
 5082.16 7.45383e-06
  ...................
  ...................
 6528.71 6.82689e-07
 6689.44 6.91085e-07
 6850.17 7.3193e-07
Minimum width confidence limits:
        significance level  lower bound  upper bound
               0.90             572.537     3153.43
               0.95             453.214     3467.07
               0.975            347.024     3667.76

One sided confidence limits for the profile likelihood:

The probability is     0.9 that pred_B is greater than 943.214
The probability is    0.95 that pred_B is greater than 750.503
The probability is   0.975 that pred_B is greater than 602.507

The probability is     0.9 that pred_B is less than 3173.97
The probability is    0.95 that pred_B is less than 3682.75
The probability is   0.975 that pred_B is less than 4199.03
\endexample

The file contains the probability density function and the 
approximate confidence limits for the the profile
likelihood and the normal approximation. Since the format is the same
for both, we only discuss the \apl\ here.
The first part of the report contains pairs of numbers 
$(x_i,y_i)$ which consist of values of the parameter in the
report (in this case {\tt PRED\_B} and the estimated value for
the probability density associated with that parameter at the point.
The probability that the parameter lies in the interval
$x_r \le x\le x_s)$ where $x_r<x_s$ can be estimated from the sum
$$\sum_{i=r}^s (x_{i+1}-x_i)y_i.$$
The reports of the one and two sided confidence limits for the parameter
were produced this way.
Also a plot of $y_i$ verses $x_i$ gives the user an indication of what
the probability distribution of the parameter looks like. 
\vbox{
\medskip
\quad\hbox{
\beginpicture
  \setplotsymbol ({\eightrm .})
  \setcoordinatesystem units <.7in,5in>
  \setplotarea x from -1.400 to 5.600, y from 0 to .55 
  \axis left 
  /
  \axis bottom label {Predicted Biomass of (2+) Fish
    $\times 10^3$} ticks
    numbered from -.0100 to 5.50 by .50 
  /
\setdashpattern <2pt,2pt,2pt,3pt>
% \plot  "eprl.tex" 
 \plot  "st.2" 
 \put {\hbox{Prof. like.}} at 3.0 .45
 \putrule from 4.2 .45 to 5.5 .45
\setdashpattern <1pt,3pt,4pt,3pt>
% \plot  "esno.tex" 
 \plot  "st.3" 
 \put {\hbox{Normal approx.}} at 3.0 .40
 \putrule from 4.2 .40 to 5.5 .40
\endpicture
\hfill
}}


The the profile likelihood
indicates the fact that the biomass can not be less than zero.
The normal approximation is not very useful for calculating the probability 
that the biomass is very low -- a question of great interest 
to managers who are probably not going to be impressed by the
knowledge that there is an estimated probability of 0.975 that
the biomass is greater than -52.660.
\beginexample
One sided confidence limits for the normal approximation

The probability is     0.9 that pred_B is greater than  551.235
The probability is    0.95 that pred_B is greater than  202.374
The probability is   0.975 that pred_B is greater than  -52.660
\endexample
\mysection{Saving the output from \apl to use as starting values
for MCMC analysis}
If the profile likelihood calculations are carried out with the
{\tt -prsave} option the values of the independent variables 
for each point on the profile are saved in a file named
{\tt xxx.pvl} where {\tt xxx} is the name of the variable
being profiled. 
\beginexample
#Step -8
#num sigmas -27
 -2.96325 6.98069 -2.96893 -1.15811 0.417864 1.5352 1.50556 
 0.668417 1.29106 2.04238 1.85167 1.02342 1.03264 1.35247 
 1.5832 1.87033 1.67212 0.984254 -0.58013 -8.10159 0.757686 
 0.958038 0.414446 -1.48443 -2.57572 -4.09184 -0.869426 -0.545055 
 -0.333125 -0.350978 -0.487261 -0.123192 -0.158569 -0.434328 
 -0.609651 -0.684244 -0.405214 5.00104
#Step -7
#num sigmas -22
  // .............................
#Step 7
#num sigmas 22
 -5.94034 9.29211 -2.6122 0.0773101 1.54853 1.91895 0.578923 
-1.51152 0.0124827 0.712157 0.520084 -0.202059 -0.0505122 
0.284112 0.469956 0.731273 0.664325 0.642344 0.691073 
-1.10233 -0.362781 0.034522 0.0127999 -0.538117 -0.575466 
-1.94386 -0.544077 -0.0349702 0.349352 0.355073 0.237236 
0.335559 0.177427 -0.0507647 -0.167382 -0.303103 -0.249956 -0.104393
#Step 8
#num sigmas 27
 -6.09524 9.43103 -2.59874 0.0930842 1.55938 
1.91285 0.561478 -1.52804 -0.0139936 0.687758 0.502089 
-0.212203 -0.0519722 0.287149 0.474422 0.739316 0.678415 
0.663857 0.71933 -1.07637 -0.387684 0.0146463 0.00647923 
-0.530625 -0.566471 -1.93414 -0.521944 -0.0111346 0.372352 
0.372706 0.247599 0.333505 0.171122 -0.0585298 -0.177735 
-0.319115 -0.273111 -0.135715
\endexample
To use the values as a starting point for the MCMC analysis
use a text editor to put the desired starting values in a file
by themselves. Suppose that the file name is {\tt mcmc.dat}.
Run the MCMC analysis with the option {\tt -mcpin mcmc.dat}
and it will begin the MCMC analysis from that file. 
\XX{command line arguments}{-prsave}
\XX{command line arguments}{-mcpin}

\mysection{The profile likelihood Calculations}
\X{profile likelihood}
\XX{profile likelihood}{form of calculations}
We have been told that the profile likelihood as calculated in \ADM\\
for dependent variables may differ from that calculated by other authors.
This section will clarify what we mean by the term and motivate our
calculation.  

Let $(x_1,\ldots,x_n)$ be $n$ independent variables, $f(x_1,\ldots,x_n)$
be a probability density function and
$g$ denote a dependent variable, that is, a real valued function of 
$(x_1,\ldots,x_n)$. The profile likelihood calculation for $g$
is intended to produce an approximation to the probability density
function for g.

Consider first the case where $g$ is equal to one of the independent
variables, say $g=x_1$. In this simple case the marginal distribution of
$x_1$ is give by the integral
\begin{equation}
  \myeq{
\int f(x_1,..., x_n)\, dx_2dx_3\ldots dx_n
}
\end{equation}
The use of the profile likelihood in this case is based on the assumption 
(or hope) that
there exists a constant $\lambda$ independent of $x_1$ such that
$\lambda \displaystyle{\max_{x_2,\ldots,x_n}} \{f(x_1,..., x_n)\}$ is a good approximation
to this integral.

This approach should be useful for a lot of applications  based on the fact that
the central limit theorem implies that for a lot of observations the
posterior proability distribution  is more or less well approximated by
a multivariate normal distribution and for such distributions the
assumptions holds exactly.
So the profile likelihood is calculated by calculating the conditional maximum of the 
likelihood function and then normalizing it so that it integrates to $1$.


For an arbitrary dependent variable the situation is a bit more
complicated. A good approximation to a probability distribution should
have the propertly of parameter invariance, that is 
$Pr \{a\le x\le b\}=Pr \{h(a)\le h(x)\le h(b)\}$ for any montonically 
increasing function $h$.
To achiveve the property of parameter invariance we modify the
definition of profile likelihood for dependent varialbes.


Fix a value $g_0$ for g and consider the integral
$$\int_{\{x:g_0-\epsilon/2\le g(x)\le g_0+\epsilon/2\}} f(x_1,..., x_n)
  \,dx_1dx_2\ldots dx_n$$
which is the probability that $g(x)$ has a value between 
$g_0-\epsilon/2$ and $g_0+\epsilon/2$. This probability depends 
on two quantities,
the value of $f(x)$ and the thickness of the region being integrated over.
We approximate $f(x)$ by its maximum value 
$\hat x(g)=\displaystyle{\max_{\{x:g(x)=g_0\}}}\{f(x)\}$. For the thickness we have
$g(\hat x+h)\approx g(\hat x)+<\nabla g(\hat x),h>=\epsilon/2$ where
$h$ is a vector perpendicular to the level set of $g$ at $\hat x$.
However  
$\nabla g$ is also perpendicular to the level set so 
$<\nabla g(\hat x),h>=\|\nabla g(\hat x)\| \|h\|$ so that 
$ \|h\|=\epsilon/(2\|g(\hat x)\|)$. Thus the integral is approximated by
 $\epsilon f(\hat x)/\|\nabla g(\hat x)\|$ and taking the derivative
with respect to $\epsilon$ yields 
$f(\hat x)/\|\nabla g(\hat x)\|$ which is the profile likelihood expression
for a dependent variable.
For an independent variable $\|\nabla g(\hat x)\|=1$  so that our 
definition of 
the profile likelihood corresponds to the usual one in this case.

 
\mysection{Modifying the profile likelihood approximation procedure}
The functions  {\tt set\_stepnumber()} and 
{\tt set\_stepsize()} can be used to modify the number of
points used to approximate the \apl\ or to change
the stepsize between the points. This can be carried out in the
 \PCS. If {\tt u} has been declared to be of type {\tt likeprof\_number}
\X{PRELIMINARY\_CALCS\_SECTION}
\XX{SECTIONS}{PRELIMINARY\_CALCS\_SECTION}
\beginexample
PRELIMINARY_CALCS_SECTION
  u.set_stepnumber(10);   // default value is 8
  u.set_stepsize(0.2);    // default value is 0.5
\endexample
\X{set\_stepnumber}
\X{set\_stepsize}
\XX{profile likelihood} {set\_stepnumber option}
\XX{profile likelihood} {set\_stepsize option}
\noindent will set the number of steps
equal to 21 (from -10 to 10) and 
will set the step size equal to 0.2 times the estimated standard
deviation for the parameter {\tt u}.
\mysection{Changing the default file names for data and parameter input}
\X{\tt ad\_comm::change\_datafile\_name}
\X{\tt ad\_comm::change\_pinfile\_name}
\XX{input}{changing the default file names}
\XX{USER\_CODE} {adding a line of the users code}
\X{adding a line of the users code to the \DS}
The following code fragment illustrates how the files used for input 
of the data and parameter values can be changed. This code has been taken from
the example {\tt catage.tpl} and modified. In the \DS, the data are first read 
in from the file {\tt catch.dat}. Then the effort data are read in from the
file {\tt effort.dat}. The remainder of the data are read in from the
file {\tt catch.dat}. It is necessary to save the current file position in
an object of type {\tt streampos}. This object is used to 
position the file properly.
The escape sequence {\tt!!} can be used to include one line of the users's
code into the \DS\ or \PS. It is more compact than the {\tt LOCAL\_CALCS}
construction.
\X{LOCAL\_CALCS}
\beginexample
DATA_SECTION
 // will read data from file catchdat.dat
 !! ad_comm::change_datafile_name("catchdat.dat");
  init_int nyrs  
  init_int nages
  init_matrix obs_catch_at_age(1,nyrs,1,nages)
 // now read the effort data from the file effort.dat and save the current
 // file position in catchdat.dat in the object tmp
 !! streampos tmp = ad_comm::change_datafile_name("effort.dat");
  init_vector effort(1,nyrs)
 // now read the rest of the data from the file catchdat.dat 
 // including the ioption argument tmp will reset the file to that position
 !! ad_comm::change_datafile_name("catchdat.dat",tmp);
  init_number M

 // ....

PARAMETER_SECTION
 // will read parameters from file catch.par
 !! ad_comm::change_parfile_name("catch.par");
\endexample
\mysection{Using the subvector operation to avoid writing loops}
\XX{subvector operation}{avoiding loops with}
\XX{subvector operation}{examples of}
If {\tt v} is a vector object then for integers {\tt l} and {\tt u}
the expression {\tt v(l,u)} is a vector object of the same type
with minimum valid index {\tt l} and maximum valid index {\tt u}
(Of course {\tt l} and {\tt u} must be within the valid index range for
{\tt v} and {\tt l} must be less than or equal to~{\tt u}.
The subvector formed by this operation ican be used on both sides of
the equals sign in an arithmetic expression. The  number of loops which must be
written can be significantly reduced in this manner. We shall use the subvector
operator to remove some of the loops in the catch-at-age model code.
\beginexample
  // calculate the selectivity from the sel_coffs
  for (int j=1;j<nages;j++)
  {
    log_sel(j)=log_sel_coff(j);
  }
  // the selectivity is the same for the last two age classes
  log_sel(nages)=log_sel_coff(nages-1);

  // same code using the subvector operation
  log_sel(1,nage-1)=log_sel_coff;
  // the selectivity is the same for the last two age classes
  log_sel(nages)=log_sel_coff(nages-1);
\endexample
\noindent Notice that {\tt log\_sel(1,nage-1)} is not a distinct vector
from  {\tt log\_sel}. This means that an assignment to {\tt log\_sel(1,nage-1)}
is an assignment to a part of {\tt log\_sel}.
The next example is a bit more complicated. It involves taking a row of a matrix,
to form a vector, forming a subvector, and changing the valid index range
for the vector.
\bestbreak
\beginexample
  // loop form of the code
  for (i=1;i<nyrs;i++)
  {
    for (j=1;j<nages;j++)
    {
      N(i+1,j+1)=N(i,j)*S(i,j);
    }
  }

  // can only eliminate the inside loop
  for (i=1;i<nyrs;i++)
  {
    // ++ increments the index bounds by 1
    N(i+1)(2,nyrs)=++elem_prod(N(i)(1,nage-1),S(i)(1,nage-1));
  }
\endexample
\noindent Notice that {\tt N(i+1)} is a vector object so that
{\tt N(i+1)(2,nyrs)} is a subvector of {\tt N(i)}. Another
point is that {\tt elem\_prod(N(i)(1,nage-1),S(i)(1,nage-1))}
is a vector object with minimum valid index {\tt 1} and maximum valid
index {\tt nyrs-1}. The operator {\tt ++} applied to a subvector
increments the valid index range by 1 so that it has the same
range of valid index values as {\tt N(i+1)(2,nyrs)}.
The operator {\tt --} would decrement the valid index range by 1.
\XX{operator ++}{used to increment the valid index bounds}
\XX{operator --}{used to decrement the valid index bounds}
\mysection{The use of higher dimensional arrays}
\XX{arrays}{three dimensional}
\XX{arrays}{four dimensional}
\X{three dimensional arrays}
\X{four dimensional arrays}
\X{five dimensional arrays}
\X{six dimensional arrays}
\X{seven dimensional arrays}
The example contained in the file
{\tt FOURD.TPL} illustrates some aspects of the use of three and four
dimensional arrays. There are now examples of the use of arrays
up to dimension~7 in the documentation\footnote{See the chapter on
regime switching models for an example of the use of higher dimensional arrays.}.
\beginexample 
DATA_SECTION
  init_4darray d4(1,2,1,2,1,3,1,3)
  init_3darray d3(1,2,1,3,1,3)
PARAMETER_SECTION
  init_matrix M(1,3,1,3) 
  4darray p4(1,2,1,2,1,3,2,3)
  objective_function_value f
PRELIMINARY_CALCS_SECTION
 for (int i=1;i<=3;i++)
 {
   M(i,i)=1;   // set M equal to the identity matrix to start
 } 
PROCEDURE_SECTION
 for (int i=1;i<=2;i++)
 {
   for (int j=1;j<=2;j++)
   {
     // d4(i,j) is a 3x3 matrix -- d3(i) is a 3x3 matrix 
     // d4(i,j)*M is matrix multiplication -- inv(M) is matrix inverse
     f+= norm2( d4(i,j)*M + d3(i)+ inv(M) );
   }
 }
REPORT_SECTION
  report << "Printout of a 4 dimensional array" << endl << endl;
  report << d4 << endl << endl;
  report << "Printout of a 3 dimensional array" << endl << endl;
  report << d3 << endl << endl;
\endexample
\noindent In the \DS\ you can use {\tt 3darrays}, {\tt 4darrays},
up to {\tt 7darrays} and
{\tt init\_3darrays}, {\tt init\_4darrays} up to {init\_7darrays}.
In the \PS\ you can use {\tt 3darrays}, {\tt 4darrays}, up to {\tt 7darrays}
and  {\tt init\_3darrays}, {\tt init\_4darrays} up to {\tt init\_5darrays}
at the time of writing. 

If {\tt d4} is a {\tt 4darray} then 
{\tt d4(i)} is a three dimensional array and {\tt d4(i,j)} is a {\tt matrix}
object so that {\tt d4(i,j)*M} is matrix multiplication. Similarly
if {\tt d3} is a {\tt 3darray} then {\tt d3(i)} is a matrix object
so that {\tt d4(i,j)*M + d3(i) +inv(M)} combines matrix multiplication,
matrix inversion, and matrix addition.

\htmlnewfile
\X{TOP\_OF\_MAIN section}
\mysection{The TOP\_OF\_MAIN section}
The TOP\_OF\_MAIN section is intended to allow the programmer to insert any desired
\cplus\ code at the top of the main() function in the program. The code is copied
literally from the template to the program.  This section can be used to set
the AUTODIF global variables (see the AUTODIF manual chapter on
AUTODIF global variables.)  The following code fragment will set these
variables.
\beginexample
TOP_OF_MAIN_SECTION
  arrmblsize = 200000; // use instead of 
                   // gradient_structure::set_ARRAY_MEMBLOCK_SIZE
  gradient_structure::set_GRADSTACK_BUFFER_SIZE(100000); // this may be incorrect in 
                                       // the AUTODIF manual.
  gradient_structure::set_CMPDIF_BUFFER_SIZE(50000); 
  gradient_structure::set_MAX_NVAR_OFFSET(500); // can have up to 500 
                                             // independent variables
  gradient_structure::set_MAX_NUM_DEPENDENT_VARIABLES(500); // can have up to 
                                          // 500 dependent variables
\endexample
\X{arrmblsize}
\X{gradient\_structure::set\_ARRAY\_MEMBLOCK\_SIZE}
\X{gradient\_structure::set\_GRADSTACK\_BUFFER\_SIZE}
\X{gradient\_structure::set\_CMPDIF\_BUFFER\_SIZE} 
\X{gradient\_structure::set\_MAX\_NVAR\_OFFSET}
\X{gradient\_structure::set\_MAX\_NUM\_DEPENDENT\_VARIABLES}

Note that within \ADM\ one doesn't use the function
{\tt gradient\_structure::set\_ARRAY\_MEMBLOCK\_SIZE} to set the amount of memory available
for variable arrays. Instead use the line of code {\tt arrmblsize = nnn;}
where {\tt nnn} is the amount of memory desired.

\X{GLOBALS\_SECTION}
\XX{SECTIONS}{\tt GLOBALS\_SECTION}
\mysection{The GLOBALS\_SECTION}
The GLOBALS\_SECTION is intended to allow the programmer to insert any desired
\cplus\ code before the  main() function in the program. The code is copied
literally from the template to the program. This enables the programmer to define
global objects and to include include header files and user-defined functions into
the generated \cplus\ code.

\XX{SECTIONS}{BETWEEN\_PHASES\_SECTION}
\X{BETWEEN\_PHASES\_SECTION}
\mysection{The BETWEEN\_PHASES\_SECTION}
\X{current\_phase() function}
Code  in the between phases section is executed before each phase of the
minimization. It is possible to carry out different actions which depend
on which phase of the minimization is to begin by using a {\tt swtich}
statement (you can read about this in a book on C or \cplus) 
together with the {\tt current\_phase()} function. 
\beginexample
switch (current_phase()
{
case 1:
  // some action
  cout << "Before phase 1 minimization " << endl;
  break;
case 2: i
  // some action
  cout << "Before phase 2 minimization " << endl;
  break;
// ....
}
\endexample
\endchapter
\htmlnewfile
%\def\chapno{2}
\mychapter{Markov Chain Simulation}
\mysection{Introduction to the Markov Chain Monte Carlo approach in Bayesian Anaylsis}
\XX{command line arguments}{-mcmc N Markov chain Monte Carlo}
\XX{command line arguments}{-mcscale}
\XX{command line arguments}{-mcr}
\XX{command line arguments}{-mcdiag}
\XX{command line arguments}{-mcmult}
\XX{command line arguments}{-mceval}
\XX{command line arguments}{-mcsave}
\XX{command line arguments}{-nosdmcmc}
The reference for this chapter is  
{\it Bayesian Data Analysis} (chapter 11) by Gelman {\it et al}.  

The Markov chain Monte Carlo method (MCMC) is a method for approximating the
posterior distribution for parameters of interest in the Bayesian framework.
This option is invoked by using the command line option {\tt -mcmc N}
where {\tt N} is the number of simulations performed. You will proabably also
want to include the option {\tt -mcscale} which dynamically scales the
covariance matrix until a reasonable acceptance rate is observed.
You may also want to use the {\tt -mcmult n} option which scales the initial
covariances matrix if the initial values are so large that arithmetic
errors occur.
One advantage of \ADM\ over some other implementations of MCMC is that
the mode of the posterior distribution  together with the hessian at
the mode is available to use for the MCMC routine. This information
is used to implement a version of the Hastings-Metropolis algorithm.
Another advantage is that
with \ADM\ it is possible to calculate the profile likelihood for a parameter
of interest and compare the distribution to the MCMC distribution for that
parameter. A large discrepancy may indicate that one or both estimates
are inadequate. If you wish to do more simulations (and to carry on from where
the last one ended use the {\tt -mcr} option. 
The following figure compares the profile likelihood for the projected
biomass
to the estimates produced by the MCMC method for different sample sizes 
(25,000 and 2,500,000 samples) for the {\tt catage} example.

\htmlbeginignore
\vbox{
\medskip
\quad\hbox{
\htmlendignore
\beginpicture
  \setplotsymbol ({\eightrm .})
  \setcoordinatesystem units <.23in,8in>
  \setplotarea x from -1.00 to 18.00, y from 0 to .37 
  \axis left 
  /
  \axis bottom label {Predicted Biomass of (2+) Fish
    $\times 10^3$} ticks
    numbered from 0.0 to 16 by 2.0 
  /
  \setplotsymbol ({\rmmedium .})
\setdashpattern <2pt,2pt,2pt,3pt>
 \putrule from 4.2 .35 to 6.5 .35
 \plot  "prof.1" 
 \put {\hbox{Prof. like.}} at 8.2 .35
  \setplotsymbol ({\eightrm .})
\setdashpattern <1pt,3pt,4pt,3pt>
 \putrule from 4.2 .32 to 6.5 .32
 \put {\hbox{MCMC 25,000}} at 8.7 .32
 \plot  "mcmc.2" 
\setdashpattern <1pt,1pt,1pt,1pt>
  \setplotsymbol ({\tenrm .})
 \putrule from 4.2 .29 to 6.5 .29
 \put {\hbox{MCMC 2,500,000}} at 9.0 .29
 \plot  "mcmc.4" 
  \setplotsymbol ({\rmmedium .})
%\setdashpattern <2pt,1pt,1pt,1pt>
% \putrule from 4.2 .26 to 6.5 .26
% \put {\hbox{MCMC 2,000,000}} at 9.0 .26
% \plot  "mcmc.4" 
% \put {\hbox{Prof. like.}} at .15 .45
% \putrule from 4.2 .45 to 5.5 .45
\setdashpattern <1pt,3pt,4pt,3pt>
% \plot  "esno.tex" 
% \plot  "st.3" 
% \put {\hbox{Normal approx.}} at 3.0 .40
% \putrule from 4.2 .40 to 5.5 .40
\endpicture
\htmlbeginignore
\hfill
}}
\htmlendignore

\X{sdreport\_number}
\X{sdreport\_vector}
\X{sdreport\_matrix}
\X{mceval\_phase()}
A report containing the observed distributions is produced in the file
{\tt root.hst}. All objects of type {\tt sdreport} i.e {\tt number}, 
{\tt vector} or {\tt matrix} are included. 
\XX{command line arguments}{-mcsave}
\XX{command line arguments}{-nosdmcmc}
It is possible to save the results of every {\tt n'th} simulation by
using the {\tt -mcsave n} option.  Afterwords these values can be used by
running the model with the {\tt -mceval} option which will evaluate the 
{\tt userfunction} once for every saved simulation value. At this time
the function {\tt mceval\_phase()} will return the value true and can be
used as a switch to perform desired calculations. The results are saved in a
binary file {\tt root.psv}. If you want to convert this file into
ASCII see the next section. If you have a large number of variables of
type {\tt sdreport} calculating the values of them for the mcmc can
appreciably slow down the calculations. To turn off these
calculations during the {\tt -mcsave} phase use the option
{\tt -nosdmcmc}. {\bf Note: If you use thus option and restart the
mcmc calculations with the {\tt -mcr} option you must use the
{\tt -nosdmcmc} as well. Otherwise the program will try to read in the
non-existent histogram data.}


\ADM\  uses the hessian to produce an (almost) multivariate normal
distribution for the Metropolis-Hastings algorithm. It is not
exacly multivariate normal because the random vectors produced are
modified to satisfy any bounds on the parameters.

There is also an option for using a fatter tailed distribution.
This distribution is a mixture of the multivariate normal and
a fat-tailed distribution. It is invoked with the {\tt -mcgrope n}
option where {\tt n} is the amount of fat-tailed distribution in
the mixture. Proabably a value of {\tt n}
between 0.05 and 0.10 is best.


\XX{binary files}{uistream}
\XX{binary files}{uostream}
\XX{binary files}{how to read them}
\mysection{Reading \ADM\ binary files}
Often the data which \ADM\ needs to save are saved in the form
of a binary file using the {\tt uistream} and {\tt uostream} classes. 
If these data consist of a series of vectors
all of which have the same dimension they are often saved in this
form where the dimension is saved at the top of the file
ad the vectors are saved afterword. It may be useful to
convert thes numbers into binary form so that they can be
put into other programs such as spreadsheets.
the following code will read the contents of these binary files.
You should call the program readbin.cpp. It should be
a simple matter to modify this program for other uses.
\beginexample
#include <fvar.hpp>
/*  program to read a binary file (using ADMB's uistream and
    uostream stream classes)  of vectors of length n.
    It is assumed that the size n is stored at the top of
    the file. there is no information about any many vectors
    are stored so we must check for an eof after each read
    To use the program you type:

                 readbin filename
*/
void produce_comma_delimited_output(dvector& v)
{
  int i1=v.indexmin();
  int i2=v.indexmax();
  for (int i=i1;i<=i2;i++)
  {
    cout << v(i) << ",";
  }
  cout << endl;
}

main(int argc, char * argv[])
{
  if (argc < 2)
  {
    cerr << " Usage:   progname inputfilename" << endl;
    exit(1);
  }
  uistream uis = uistream(argv[1]);
  if (!uis)
  {  
    cerr << " Error trying to open binary input file " 
         <<  argv[1] << endl;
    exit(1);
  }
  int ndim;
  uis >> ndim;
  if (!uis)
  {  
    cerr << " Error trying to read dimension of the vector"
            " from the top of the file " 
         <<  argv[1] << endl;
    exit(1);
  }
  if (ndim <=0)
  {  
    cerr << " Read invalid dimension for the vector"
            " from the top of the file " 
         <<  argv[1] << " the number was " << ndim << endl;
    exit(1);
  }
  
  int nswitch;
  cout << " 1 to see all records" << endl
       << " 2 then after the prompts  n1 and  n2 to see all" << endl
       << " records between n1 and n2 inclusive" <<  endl
       << " 3 to see the dimension of the vector" << endl
       << " 4 to see how many vectors there are" << endl;
  cin >> nswitch;
  dvector rec(1,ndim);
  int n1=0;
  int n2=0;
  int ii=0;
  switch(nswitch)
  {
  case 2:
    cout << " Put in the number for the first record you want to see"
         << endl;
    cin >> n1;
    cout << " Put in the number for the second record you want to see"
         << endl;
    cin >> n2;
  case 1:
    do 
    {
      uis >> rec;
      if  (uis.eof()) break;
      if (!uis) 
      {
        cerr << " Error trying to read vector number " << ii
             << " from file " <<  argv[1] << endl;
        exit(1);
      }
      ii++;
      if (!n1)
      {
        // comment out the one you don't want
        //cout << rec << endl;
        produce_comma_delimited_output(rec);
      }
      else
      {
        if (n1<=ii && ii<=n2)
        {
          // comment out the one you don't want
          //cout << rec << endl;
          produce_comma_delimited_output(rec);
        }
      }
    }
    while (1);
    break; 
  case 4:
    do 
    {
      uis >> rec;
      if  (uis.eof()) break;
      if (!uis) 
      {
        cerr << " Error trying to read vector number " << ii
             << " from file " <<  argv[1] << endl;
        exit(1);
      }
      ii++;
    }
    while (1);
    cout << " There are " << ii << " vectors" << endl;
    break; 
  case 3:
    cout << " Dimension = "  << ndim  << endl;
  default:
    ;
  }
}
\endexample
\mysection{Convergence diagnostics for MCMC analysis}
A major difficulty with MCMC analysis is determining  whether the
chain has converged to the underlying distribution. In general it is
never possible to prove that this convergence has occurred. In this
section we concentrate on methods which hopefully will detect
situations when convergence has not occurred.

The default MCMC method employed in \ADM\ takes advantage of the
fact that \ADM\ can find the mode of the posterior distribution and
compute the Hessian at the mode. If the posterior distribution is well
approximated by a multivariate normal centered at the mode
with covariance matrix equal to the inverse of the Hessian 
this method can be extremely efficient for many parameter problems,
expecially when compared to simpler methods such as the Gibbs sampler.
The price one pays for this increased efficiency is that the method
is not as robust as the Gibbs sampler and for some problems it
will perform much more poorly than the Gibbs sampler.

As an example of this poor performance we consider a simple three
parameter model developed by Vivian Haist to analyze Bowhead whale
data.

The data for the model consist of total catches between 1848 and 1993
together with an estimate of the biomass in 1988 and an estimate of the
change in relative biomass between 1978 and 1988.
 
\beginexample
DATA_SECTION
  init_vector cat(1848,1993)

PARAMETER_SECTION
  init_bounded_number k(5000,40000,1)
  init_bounded_number r(0,0.10,1)
  init_bounded_number p(0.5,1,2)
  number delta;
  vector bio(1848,1994);
  likeprof_number fink
 !! fink.set_stepsize(.003);
 !! fink.set_stepnumber(20);
  sdreport_number finr
  sdreport_number finp
  objective_function_value f
PROCEDURE_SECTION
  if (initial_params::mc_phase)
  {
    cout << k << endl;
    cout << r << endl;
    cout << p << endl;
  }
  bio(1848)=k*p;
 
  for (int iy=1848; iy<=1993; iy++)
  {
     dvariable fpen1=0.0;
     bio(iy+1)=posfun(bio(iy)+r*bio(iy)*(1.-(bio(iy)/k)),100.0,fpen1); 
     dvariable sr=1.- cat(iy)/bio(iy);
     dvariable kcat=cat(iy);
      f+=1000*fpen1;
     if(sr< 0.05)
     {
       dvariable fpen=0.;
       kcat=bio(iy)*posfun(sr,0.05,fpen);
       f+=10000*fpen;
//     cout << " kludge "<<iy <<" "<<kcat<<" "<<cat(iy)<<" "<<fpen<<endl;
     }
     bio(iy+1)-=kcat;
  }
  finr=r;
  fink=k;
  finp=p;
  delta=(bio(1988)-bio(1978))/bio(1978);
  f+=log(sqrt(2.*3.1415927)*500)+square(bio(1988)-7635.)/(2.*square(500));
  f+=log(sqrt(2.*3.1415927)*.03)+square(delta-0.15)/(2.*square(.03));
\endexample
This is a biomass dynamic model where the biomass is assumed to
satisfy the difference equation
\begin{equation}
B_{i+1}=B_i+r*B_i(1-B_i/k)-C_i 
\end{equation}
For this formulation there is no guarantee that the biomass will
remain positive so the {\tt posfun} function has been used in 
the program to ensure that this condition will hold.
This is a very ``data poor'' design. 

The model was fit to the data and the standard MCMC analysis was performed
for it. The results were compared to an MCMC analysis performed with
the Gibbs sampler. It was found that the Gibbs sampler performed better.

It is not difficult to determine why the MCMC performed so poorly.
the estimated covariance matrix for the parameters is shown below.
To four signficant figures the correlation between $r$ and $k$
is $-1.0000$ Thus the hessian matrix is almost singular.
\beginexample
   index   name    value      std dev     1      2       3   
     1   k     1.0404e+04 8.8390e+05   1.0000
     2   r     4.8838e-02 9.0337e+00  -1.0000  1.0000
     3   p     5.7293e-01 3.6946e+00   0.9998 -0.9998  1.0000
\endexample
If the posterior distribution were exactly normally distributed then
the hessian would be constant ie not depend on the
point at which is is calculated and its use would 
produce the most efficient MCMC
procedure. However in nonlinear models the posterior distribution is
not normally distributed so that the Hessian changes as we move away
from the mode and the use of an almost singular
Hessian can perform very badly as in the present case.

To deal with almost singular Hessians we have added the {\tt -mcrb N}
option.
\XX{command line arguments}{-mcrb}
This option reduces the amount of correlation in the Hessian while leaving
the standard deviaions fixed. The number {\tt N} should be between
1 and 9. the smaller the number the more the correlation is reduced.
For this example a value of 3 seemed to perform well.
\vfill
\hrule
\pdfximage  height 3.5in width 6.5in depth 0in {mcrb3-50K.png}
\vfill


% !! fink.set_stepsize(.003);
% !! fink.set_stepnumber(20);
\endchapter
\htmlnewfile

%\def\chapno{3}
\mychapter{A forestry model -- estimating the size distribution of
 wildfires}
\XX{ad\_begin\_funnel}{reducing the amount of temporary storage}
\X{numerical integration}
\mysection{Model description}
This examples highlights two features of \ADM, the use of a numerical
integration routine within a statistical parameter estimation model
and the use of the {\tt ad\_begin\_funnel} mechanism to reduce the size
of temporary file storage required. It also provides a performance comparison between
\ADM\ and Splus.

This problem investigates a model which predicts a relationship
between the size and frequency of wildfires.
It is assumed that the probability of observing a wildfire
in size category $i$ is given by $P_i$, where
$$\log(P_i)=\ln\big(S_i-S_{i+1}\big)-\ln\big(S(1)\big).$$
If $f_i$ is the number of widfires observed to lie in
size category $i$ the log-likelihood function for the problem is given by

\begin{equation}
  \myeq{
l(\tau,\nu,\beta,\sigma) = 
   \sum_i f_i\,\big[\ln\big(S_i-S_{i+1}\big)-\ln\big(S(1)\big)\big]}
\label{chp3:xx9}
\end{equation} 
where $S_i$ is defined by the integral
\begin{equation}
  \myeq{
S_i=\int_{-\infty}^\infty 
   \exp\Big\{-z^2/2 + 
  \tau\Big(-1+\exp\big(-\nu a_i^\beta\exp(\sigma z)\big)\Big) \Big\}dz}
\label{chp3:xx10}
\end{equation} 
 
The parameters $\tau$, $\nu$, $\beta$, and $\sigma$ are 
functions of the parameters of the original 
model, and don't have a simple interpretation. 
Fitting the model to 
data involves maximizing the above log-likelihood (\number\mychapno.1).  While the gradient can be 
calculated (in integral form), coding it is cumbersome.  Numerically maximizing 
the log-likelihood without specifying the gradient is preferable.  

The parameter $\beta$ is related to the fractal dimension of the perimeter
of the fire.  One hypothesis of interest is that $\beta =2/3$ which is related
to hypotheses about the nature of the mechanism by which fires spread. 
The \ADM\ code for the model follows. \par
\beginexample
DATA_SECTION
 int time0 
 init_int nsteps 
 init_int k
 init_vector a(1,k+1)
 init_vector freq(1,k)
 int a_index;
 number sum_freq
!! sum_freq=sum(freq);
PARAMETER_SECTION
  init_number log_tau
  init_number log_nu
  init_number log_beta(2)
  init_number log_sigma
  sdreport_number tau
  sdreport_number nu
  sdreport_number sigma
  sdreport_number beta
  vector S(1,k+1)
  objective_function_value f
INITIALIZATION_SECTION
  log_tau 0  
  log_beta -.405465 
  log_nu 0  
  log_sigma -2
PROCEDURE_SECTION
  tau=exp(log_tau);
  nu=exp(log_nu);
  sigma=exp(log_sigma);
  beta=exp(log_beta);
   funnel_dvariable Integral;
   int i;
   for (i=1;i<=k+1;i++)
   {
     a_index=i;
     ad_begin_funnel();
     Integral=adromb(&model_parameters::h,-3.0,3.0,nsteps);
     S(i)=Integral;
   }
   f=0.0;
   for (i=1;i<=k;i++)
   {
     dvariable ff=0.0;
     // make the model stable for case when S(i)<=S(i+1)
     // we have to subrtract s(i+1) from S(i) first or roundoff will
     // do away with the 1.e-50.
     f-=freq(i)*log(1.e-50+(S(i)-S(i+1)));
     f+=ff;
   }
   f+=sum_freq*log(1.e-50+S(1));
FUNCTION dvariable h(const dvariable& z)
  dvariable tmp;
  tmp=exp(-.5*z*z + tau*(-1.+exp(-nu*pow(a(a_index),beta)*exp(sigma*z))) );  
  return tmp;
REPORT_SECTION
  int * pt=NULL;
  report << " elapsed time = "  << time(pt)-time0 << " seconds" << endl;
  report << "nsteps = " << setprecision(10) <<  nsteps << endl;
  report << "f = " << setprecision(10) <<  f << endl;
  report << "a" << endl << a << endl;
  report << "freq" << endl << freq << endl;
  report << "S" << endl << S << endl;
  report << "S/S(1)" << endl << setfixed << setprecision(6) << S/S(1) << endl;
  report << "tau "  << tau << endl; 
  report << "nu "  << nu << endl; 
  report << "beta "  << beta << endl; 
  report << "sigma "  << sigma << endl; 
\endexample
\mysection{The numerical integration routine} 
The statement 
\beginexample
   Integral=adromb(&model_parameters::h,-3.0,3.0,nsteps);
\endexample
\noindent invokes the numerical integration routine for the user-defined
function {\tt h}. The function must be defined in a {\tt FUNCTION}
subsection. It can have any name, must be defined to take a 
{\tt const dvariable\&} argument, and must return a {\tt dvariable}.
The values -3.0, 3.0 are the limits of integration (effectively
$-\infty$, $\infty$ for this example). The integer argument nsteps
determines how accurate the integration will be. Higher values of 
nsteps will be more accurate but greatly increase the amount of
time necessary to fit the model. The basic strategy is to use a moderate
value for nteps, such as 6, and then to increase this value to see
if the parameter estimates change much.
\beginexample
FUNCTION dvariable h(const dvariable& z)
\endexample
\mysection{Using the ad\_begin\_funnel routine to reduce the amount of temporary
storage required} 
Numerical integration routines can be very computationally intensive,
especially when they must be computed to great accuracy.
Such computations will require a lot of temporary storage in \ADM.
Fortunately the output from such a routine is just one number, the
value of the integral. In automatic differentiation terminology
a long set of computations which produce just one number is known
as a funnel. It is possbile to exploit the properties of such a
funnel to greatly reduce the amount of temporary storage required.
All that is necessary is to declare an object of type {\tt funnel\_dvariable}
\XX{\tt funnel\_dvariable}{use with {\tt ad\_begin\_funnel}}
\X{\tt ad\_begin\_funnel}
and to assign the results of the computation to it.  At the beginning
of the funnel a call to the function {\tt ad\_begin\_funnel} is
made. There is quite a bit of overhead associated with the funnel
construction so it should not be used for very small calculations.
However it is possible to put it in and test the program to
see whether it runs more quickly or not. The following  modifed
code will produce exactly the same results, but without the funnel
construction.
\beginexample
  dvariable Integral;   // change the definition of Integral
  int i;
  for (i=1;i<=k+1;i++)
  {
    a_index=i;
    // ad_begin_funnel();  // commment out this line
    Integral=adromb(&model_parameters::h,-3.0,3.0,nsteps);
    S(i)=Integral;
  }
\endexample
If the funnel construction is used on a portion of code which is
not a funnel, incorrect derivative  values will be obtained.
If this is suspected the funnel should be removed as in the above
example and the model run again.

\mysection{Effect of the accuracy switch on the running time for
numerical integration}
The following report shows the amount of time required to run the model
with a fxied value of $\beta$ for different values of the
parameter {\tt nsteps}. For practical perposes a vlaue of nsteps=8
gives enough accuracy so that the model could be fit in about 6 seconds.
\beginexample
elapsed time = 2 seconds nsteps = 6 f = 629.9846518
tau 9.851110 nu 8.913479 beta 0.666667 sigma 1.885570

elapsed time = 2 seconds nsteps = 7 f = 629.9851092
tau 9.850213 nu 8.835066 beta 0.666667 sigma 1.882967

elapsed time = 6 seconds nsteps = 8 f = 629.9851223
tau 9.850227 nu 8.836769 beta 0.666667 sigma 1.883024

elapsed time = 6 seconds nsteps = 9 f = 629.9851222
tau 9.850226 nu 8.836769 beta 0.666667 sigma 1.883024

elapsed time = 14 seconds nsteps = 10 f = 629.9851222
tau 9.850226 nu 8.836769 beta 0.666667 sigma 1.883024
\endexample

The corresponding times when beta was estimated in an extra
phase of the minimization are given here. 
It as apparent tat the model parameters become unstable when
beta is being estimated.  Twice the log-likelihood difference
is $2(629.98-627.31)=5.34$ which is significant

\beginexample

elapsed time = 3 seconds nsteps = 6 f = 627.2919906
tau 20.729183 nu 427.816375 beta 0.180225 sigma 2.499445

elapsed time = 6 seconds nsteps = 7 f = 627.2952716
tau 21.868971 nu 80914.970724 beta 0.170392 sigma 4.232237

elapsed time = 17 seconds nsteps = 8 f = 627.297021
tau 22.858629 nu 2326271883.421848 beta 0.164749 sigma 7.653068

elapsed time = 62 seconds nsteps = 9 f = 627.2993787
tau 23.771061 nu 1652877622661391616.000000 beta 0.161073 sigma 14.451510

elapsed time = 123 seconds nsteps = 10 f = 627.3106333
tau 23.116097 nu 49753858778.636856 beta 0.159364 sigma 8.663666

elapsed time = 244 seconds nsteps = 11 f = 627.310624
tau 23.115275 nu 49009470510.133156 beta 0.159369 sigma 8.658643

\endexample

\mysection{A comparison with Splus for the forestry model}

The Splus minimizing routine nlminb was used to fit the model.
Fitting the three parameter model with Splus required
approximately 280 seconds compared to 6 seconds with \ADM,
so that \ADM\ was approximately 45 times faster for this simple problem.

For the four parameter problem with beta estimated,
the SPLUS routine exited after
fourteen minutes and 30 seconds, reporting false convergence
with a function value of 627.338.

The data for the example is
\beginexample
a
 0.04 0.1 0.2 0.4 0.8 1.6 3.2 6.4 12.8 25.6 51.2 102.4 204.8
freq
 167 84 61 29 19 17 4 4 1 0 1 1
\endexample
\noindent where the first line contains the bounds for the size
catagories and the second line contains the number of observations
in each size category.
The Splus code with fixed beta for the example is
\beginexample
obj.20<-
function(xvec)
{
#Objective for maxn in NLMINB   NB vector argument
 - llik.20(xvec[1], xvec[2], xvec[3])
}
llik.20<-
function(logtau, lognu, logsigma)
{
        tau<-exp(logtau)
        nu<-exp(lognu)
        sigma<-exp(logsigma)
	print(tau)
	print(nu)
	print(sigma)
        llik <- 0
        for(i in 1:(length(freq)+1)) {
           Int[i]<-S.20(xa[i], tau, nu, sigma)
	}      
        print(llik)
        for(i in 1:length(freq)) {
           llik <- llik + (freq[i] * (log(1.e-50+(Int[i]-Int[i+1])) 
               -log(1.e-50+Int[1])))
        }
        llik
}
S.20<-
function(da, tau, nu, sigma)
{
        results <- integrate(intgnd.20, -3, 3, TAU = tau, NU = nu, SIGMA =
                sigma, A = da)
        if(results\$message != "normal termination")
                ans <- results\$message
        else ans <- results\$integral
        ans
}
intgnd.20<-
function(z, A, TAU, NU, SIGMA)
{
$$  \exp\big(-0.5 * z^2 + TAU * (-1 +\exp( - NU * A^2/3 * \exp(SIGMA * z)))\big)$$
}
\endexample
To run the example in Splus with the same initial values 
use the following values 
\beginexample
  logtau 0  lognu 0  logsigma -2
\endexample
The vector xa should contain the 13 {\tt a} values while
the vector freq should contain the 12 observed frequencies.

\endchapter
\htmlnewfile

%\def\chapno{4}
\mychapter{Economic Models -- regime switching}

An active field in macroeconomic modeling is the area of ``regime switching''.
This is discussed in greater generality in
Hamilton (1994, chapter 22)\footnote{\rmfoot 
Hamilton, James D. 1994. 
{\it Time Series Analysis}, 
Princeton, N.J.: Princeton University Press.}.
The code for the following example
is based on the
domain switching model taken from 
Hamilton (1989)\footnote{\rmfoot A new approach to the 
%Hamilton (1989)\footnote{${}^{\dagger\dagger}$}{\rmfoot A new approach to the 
economic analysis of nonstationary time series and the 
business cycle, {\it Econometrica}, {\bf 57(2)}:357-384.}.
This example is not ideal for exploiting \ADM's greatest advantage,
the ability to estimate parameters in models with a large number of
independent variables. However it does illustrate the efficacy of the use 
of higher (up to seven dimensional) arrays in \ADM.
\mysection{Anaylsis of economic data from Hamilton's 1989 paper} 
For this model
The observed quantities are the $Y_t$ where
\begin{equation}
 \myeq{
Y_t=a_0+a_1s_{ti}+Z_t}
\label{chp4:xx1}
\end{equation} 
and the state variables $Z_t$ satisfy the fourth order 
autoregressive relationship
\begin{equation}
  \myeq{
Z_t=f_1Z_{t-1}+f_2Z_{t-2}+f_3Z_{t-3}+f_4Z_{t-4}+
  \epsilon_t }
\label{chp4:xx2}
\end{equation} 
where the $\epsilon_t$ are independent, normally distributed random variables
with mean~$0$  and standard deviation~$\sigma$.
These equations correspond to Hamilton's equations 4.3.
The state variable $s_{ti}$ is the realized value of a Markov process,~$S_t$,
whose evolution is described below. This coefficient 
takes on the value $i$ 
when the system is in state~$i$. In the current example
there are two states so that $s_t$ takes on one of the two values
$0$ or $1$. We can solve $\number\mychapno.1$ for the values of $Z_t$ conditioned on the
unknown value of the state at time $t$.
Let $z_{ti}$ be defined by
%$$\eqalign{
 \begin{eqnarray}
z_{i0}&=Y_t-a_0\cr
z_{t1}&=Y_t-a_0-a_1\cr
 \end{eqnarray}
%}\eqno(\number\mychapno.3)$$

Let $(i,j,k,l,m)$ be a quintuplet of state values
for the states at time $t,t-1,\ldots,t-4.$ 
Define~$e(t,i,j,k,l,m)$, the realized values of the
random variables~$\epsilon_t$ by  
$$e(t,i,j,k,l,m)=
  Y_{ti}-f_1z_{t-1,j}-f_2z_{t-2,k}
   -f_3z_{t-3,l}-f_4z_{t-4,m}$$
Notice that we due to the lags we can only begin to calculate values for
the $e(t,i,j,k,l,m)$ in time period~$5$.
It is assumed that the states transitions are given by a Markov process
with transition matrix 
$P=(p_{ij})$\footnote{${}^{\dagger\dagger\dagger}${\rmfoot 
Hamilton seems to  index his matrices with
the column index first in some cases. 
We use the row index first. Thus Hamilton's
$p_{ij}$ may correspond to our $p_{ji}$.}}
If we are in state $j$ at time $t$ the probability of being in state
$i$ at time $t+1$ is $p_{ij}$.

If we consider the quintuple of the last 5 states to be 
the states of a new markov process then we can define the transition matrix
for this process by 
$$(i,j,k,l,m) \Rightarrow (0,i,j,k,l) 
 \hbox{\rm \quad with probability\ } p_{0i}$$
and
$$(i,j,k,l,m) \Rightarrow (1,i,j,k,l) 
 \hbox{\rm \quad with probability\ } p_{1i}$$
 If $q(t-1,j,k,l,m,n)$ is the probability of
being in state $(j,k,l,m,n)$  at period $t-1$
the probability of being in state  $q(t,i,j,k,l,m)$ 
at time period $t$ is given by
$$q(t,i,j,l,,m)=\sum_{n}
  P_{ij}q(t-1,j,k,l,m,n)$$
In particular if $$q_b(t,i,j,k,l,m)$$ is the 
probability of being in
the state $(i,j,l,m,n)$ before observing $Y_t$
and 
\hbox{$q_a(t-1,j,k,l,m,n)$} is the probability of being in
the state $(j,k,l,m,n)$ after observing 
$Y_{t-1}$ then
\begin{equation}
 \myeq{
q_b(t,i,j,k,l,m)=\sum_{n}
  P_{ij}q_a(t-1j,k,l,m,n)}
\label{chp4:xx3}
\end{equation} 
Let $Q(Y_t|(i,j,k,l,m),Y_{t-1},Y_{t-2},Y_{t-3},Y_{t-4})$ 
be the conditional probability (or probability density) for
$Y_t$ given $S_{t}=i,S_{t-1}=j,S_{t-2}=k,S_{t-3}=l,S_{t-4}=m,
Y_{t-1},Y_{t-2},Y_{t-3},Y_{t-4}$. Then, ignoring a constant term
which is irrelevant for the calculations,
\begin{equation}
\myeq{
Q(Y_t|(i,j,k,l,m),Y_{t-1},Y_{t-2},Y_{t-3},Y_{t-4})=
  \exp\big(-e(i,j,k,l,m)^2/2\sigma^2\big)/\sigma }
\label{chp4:xx4}
\end{equation} 
Define $u(Y_t,i,j,k,l,m)$ by
\begin{equation}
 u(Y_t,i,j,k,l,m)=
\myeq{
  Q(Y_t|(i,j,k,l,m),Y_{t-1},\ldots,Y_{t-4})
    q_b(t,i,j,k,l,m)}
\label{chp4:xx5}
\end{equation} 
Then $q_a(t,i_{t},j,k,l,m)$ can be calculated 
from the relationship
\begin{equation}
\myeq{
q_a(t,i_{t},j,k,l,m)= u(Y_t,i,j,k,l,m)/
   \sum_{i,j,k,l,m} u(Y_t,i,j,k,l,m)}
\label{chp4:xx6}
\end{equation} 

The log-likelihood function for the parameters can be calculated
from the $u(Y_t,i,j,k,l,m)$. It is equal to
\begin{equation}
\myeq{
\sum_t \log\big(\sum_{i,j,k,l,m} 
        u(Y_t,i,j,k,l,m)\big) }
\label{chp4:xx7}
\end{equation} 
The sums needed for the calculations in $\number\mychapno.9$ 
can be saved from the calculations for $\number\mychapno.8$).

\mysection{The code for Hamilton's fourth order autoregressive model}
The complete \ADM\ template (TPL) code is in the file {\tt ham4.tpl}
The \cplus\  (CPP) code produced from this is in the file {\tt ham4.cpp}
Here is the TPL code split up with comments. 
\beginexample
DATA_SECTION
  init_number a1init   // read in the initial value of a1 with the data
  init_int nperiods1   // the number of observations
  int nperiods  // nperiods-1 after differencing
 !! nperiods=nperiods1-1;
  init_vector yraw(1,nperiods1)  //read in the observations
  vector y(1,nperiods)   // the differenced observations
 !! y=100.*(--log(yraw(2,nperiods1)) - log(yraw(1,nperiods))); 
  int order 
  int op1  
 !! order=4; //order of the autoregressive process
 !! op1=order+1;
  int nstates  // the number of states (expansion and contraction)
 !! nstates=2;
\endexample
The \DS\ contains constant quantities or ``data''. This is in contrast
to quantities which depend on parameters being estimated which go into the
\PS. All quantities in the \PS 
with the {\tt init\_} prefix are initial data
which must be read in from somewhere. By default they are read in from the
file {\tt ROOT.dat} (DAT file) where {\tt ROOT} is the root part of the name 
of the program being run (in this case {\tt ham4.exe}), so {\tt ham4.dat}.

The first quantity is a number, {\tt a1init} which will be used for initializing the
value of {\tt a1} in the program. This is a simple way to try 
different initial values for {\tt a1} simply by modifying the input data file.
Such procedures are often valuable to ensure that the correct global
value of the objective function has been found.
The  second quantity {\tt nperiods1} is the number of data points in the
file. Notice that as soon as a quantity has been defined it is available to use for
defining other quantities. The quantity {\tt nperiod} does not have an
{\tt init\_} before it so it will not be read in an must be calulated
in terms of other quantities at some point. Since we want it now it is 
calculated immediately.
\beginexample
 !! nperiods=nperiods1-1;
\endexample
\XX{LOCAL\_CALCS}{!!}
\XX{!!}{LOCAL\_CALCS}
\noindent The {\tt !!} are used to insert any valid \cplus\ code into the \DS\
or \PS\ (see {\tt LOCAL\_CALCS}). 
This code will be executed verbatim (after the {\tt !!} have been stripped off of
course) at the appropriate time.
The {\tt init\_vector yraw} is defined  and give a size with indices going from
{\tt 1} to {\tt nperiods1}. The {\tt nperiods1} data points will be read into
{\tt yraw} from the DAT file. 
The data are immediately transformed and the resulting
{\tt nperiods} data point are put into {\tt y}.
\beginexample
PARAMETER_SECTION
  init_vector f(1,order,1)  // coefficients for the atuoregressive
                            // process
  init_bounded_matrix Pcoff(0,nstates-1,0,nstates-1,.01,.99,2)  
        // determines the transition matrix for the markov process
  init_number a0(5)  // equation 4.3 in Hamilton (1989)
  init_bounded_number a1(0.0,10.0,4);  
 !! if (a0==0.0) a1=a1init;  // set initial value for a1 as specified
                     // in the top of the file nham4.dat
  init_bounded_number smult(0.01,1,3)  // used in computing sigma
  matrix z(1,nperiods,0,1)  // computed via equation 4.3 in 
                          // Hamilton (1989)
  matrix qbefore(op1,nperiods,0,1);  // prob. of being in state before
  matrix qafter(op1,nperiods,0,1); // and after observing y(t)
  number sigma // variance of epsilon(t) in equation 4.3
  number var  // square of sigma
  sdreport_matrix P(0,nstates-1,0,nstates-1);
  number ff1;
  vector qb1(0,1); 
  matrix qb2(0,1,0,1); 
  3darray qb3(0,1,0,1,0,1);
  4darray qb4(0,1,0,1,0,1,0,1);
  6darray qb(op1,nperiods,0,1,0,1,0,1,0,1,0,1); 
  6darray qa(op1,nperiods,0,1,0,1,0,1,0,1,0,1);
  6darray eps(op1,nperiods,0,1,0,1,0,1,0,1,0,1);
  6darray eps2(op1,nperiods,0,1,0,1,0,1,0,1,0,1);
  6darray prob(op1,nperiods,0,1,0,1,0,1,0,1,0,1);
  objective_function_value ff;
\endexample
The \PS\ describes the parameters of the model, that is, the quantities
to be estimated. Quantities which which have the prefix {\tt init\_}
are akin to the independent variables from which the log-likelihood
function (or more generally any objective function) can be calculated.
Other objects are dependent variables which must be calculated from the
independent variables. The default behaviour of \ADM\ is to read in
initial parameter values for the parameters from a {\tt PAR} file if
it finds one. Otherwise they are given default values consistent with their
type.
The quantity {\tt f} is a vector of four coefficents for the autoregressive
process. {\tt Pcoff} is a $2\times 2$ matrix which is used to parameterize
The transition matrix {\tt P} for the Markov process. Its values are restricted to lie
between $.01$ and $0.99$. {\tt smult} is a number used to parameterize 
{\tt sigma} and {\tt var} (which is the variance) as a multiple of the
mean squared residuals. This reparameterization undimensionalizes the
calculation and is a good  technique to employ for nonlinear modeling
in general. The transition matrix {\tt P} is defined to be of 
type {\tt sdreport\_matrix} so that the standard deviation estimates
for its members will be included in the standard deviation report contained
in the {\tt} STD file. To date \ADM\ suports up to seven
dimensional arrays. For historical reasons one and two dimensional
arrays are referred to as {\tt vector} and  {\tt matrix}. This becomes a
bit difficult for higer dimensional arrays so they are simply referred
to as {\tt 3darray},{\tt 4darray},$\ldots$,{\tt 7darray}.
\XX{arrays}{vector}
\XX{arrays}{matrix}
\XX{arrays}{3darray}
\XX{arrays}{4darray}
\XX{arrays}{5darray}
\XX{arrays}{6darray}
\XX{arrays}{7darray}
\X{3darray}
\X{4darray}
\X{5darray}
\X{6darray}
\X{7darray}
\X{three dimensional arrays}
\X{four dimensional arrays}
\X{six dimensional arrays}
\beginexample
PROCEDURE_SECTION
  P=Pcoff;
  dvar_vector ssum=colsum(P);  // form a vector whose elements are the
                           // sums of the columns of P
  ff+=norm2(log(ssum)); // this is a penalty so that the hessian will
                        // not be singular and the coefficients of P 
                        // will be well defined
  // normalize the transition matrix P so its columns sum to 1
  int j;
  for (j=0;j<=nstates-1;j++)
  {
    for (int i=0;i<=nstates-1;i++)
    {
      P(i,j)/=ssum(j);
    }
  }  

  // get z into a useful format
  dvar_matrix ztrans(0,1,1,nperiods);
  ztrans(0)=y-a0;
  ztrans(1)=y-a0-a1;
  z=trans(ztrans);
  int t,i,k,l,m,n;
  
  qb1(0)=(1.0-P(1,1))/(2.0-P(0,0)-P(1,1)); // unconditional distribution
  qb1(1)=1.0-qb1(0);
  
  // for periods 2 through 4 there are no observations to condition
  // the state distributions on so we use the unconditional distributions
  // obtained by multiplying by the transition matrix P.
  for (i=0;i<=1;i++) {
    for (j=0;j<=1;j++) qb2(i,j)=P(i,j)*qb1(j);  
  }
  
  for (i=0;i<=1;i++) {
    for (j=0;j<=1;j++) {
      for (k=0;k<=1;k++) qb3(i,j,k)=P(i,j)*qb2(j,k); 
    }  
  }
  
  for (i=0;i<=1;i++) {
    for (j=0;j<=1;j++) {
      for (k=0;k<=1;k++) {
        for (l=0;l<=1;l++) qb4(i,j,k,l)=P(i,j)*qb3(j,k,l); 
      }
    }  
  }
  
  // qb(5) is the probabilibility of being in one of 32
  // states (32=2x2x2x2x2) in periods 5,4,3,2,1 before observing
  // y(5)
  for (i=0;i<=1;i++) {
    for (j=0;j<=1;j++) {
      for (k=0;k<=1;k++) {
        for (l=0;l<=1;l++) {
          for (m=0;m<=1;m++) qb(op1,i,j,k,l,m)=P(i,j)*qb4(j,k,l,m); 
        }  
      }
    }  
  }
  // now calculate the realized values for epsilon for all 
  // possible combinations of states
  for (t=op1;t<=nperiods;t++) {
    for (i=0;i<=1;i++) {
      for (j=0;j<=1;j++) {
        for (k=0;k<=1;k++) {
          for (l=0;l<=1;l++) {
            for (m=0;m<=1;m++) {
              eps(t,i,j,k,l,m)=z(t,i)-phi(z(t-1,j),
                z(t-2,k),z(t-3,l),z(t-4,m),f);
              eps2(t,i,j,k,l,m)=square(eps(t,i,j,k,l,m));
            }  
          }
        }
      }                            
    }  
  }  
  // calculate the mean squared "residuals" for use in 
  // "undimensionalized" parameterization of sigma
  dvariable eps2sum=sum(eps2);
  var=smult*eps2sum/(32.0*(nperiods-4)); 
  sigma=sqrt(var);
  
  for (t=op1;t<=nperiods;t++) {
    for (i=0;i<=1;i++) {
      for (j=0;j<=1;j++) {
        for (k=0;k<=1;k++) 
          prob(t,i,j,k)=exp(eps2(t,i,j,k)/(-2.*var))/sigma;
      }                            
    }  
  }  
  
  for (i=0;i<=1;i++) {
    for (j=0;j<=1;j++) {
      for (k=0;k<=1;k++) {
        for (l=0;l<=1;l++) {
          for (m=0;m<=1;m++) qa(op1,i,j,k,l,m)= qb(op1,i,j,k,l,m)*
            prob(op1,i,j,k,l,m);
        }
      }
    }                            
  }  
  ff1=0.0;
  qbefore(op1,0)=sum(qb(op1,0));
  qbefore(op1,1)=sum(qb(op1,1));
  qafter(op1,0)=sum(qa(op1,0));
  qafter(op1,1)=sum(qa(op1,1));
  dvariable sumqa=sum(qafter(op1));
  qa(op1)/=sumqa;
  qafter(op1,0)/=sumqa;
  qafter(op1,1)/=sumqa;
  ff1-=log(1.e-50+sumqa);
  for (t=op1+1;t<=nperiods;t++) { // notice that the t loop includes 2 
    for (i=0;i<=1;i++) {      // i,j,k,l,m blocks
      for (j=0;j<=1;j++) {
        for (k=0;k<=1;k++) {
          for (l=0;l<=1;l++) {
            for (m=0;m<=1;m++) {
              qb(t,i,j,k,l,m).initialize(); 
              // here is where having 6 dimensional arrays makes the
              // formula for moving the state distributions form period
              // t-1 to period t easy to program and understand.
              // Throw away  n and accumulate its two values into next
              // time period after multiplying by transition matrix P
              for (n=0;n<=1;n++) qb(t,i,j,k,l,m)+=P(i,j)*qa(t-1,j,k,l,m,n); 
            }
          }
        }
      }                            
    }  
    for (i=0;i<=1;i++) {
      for (j=0;j<=1;j++) {
        for (k=0;k<=1;k++) {
          for (l=0;l<=1;l++) {
            for (m=0;m<=1;m++) qa(t,i,j,k,l,m)=qb(t,i,j,k,l,m)*
                  prob(t,i,j,k,l,m);
          }
        }
      }                            
    }  
    qbefore(t,0)=sum(qb(t,0));
    qbefore(t,1)=sum(qb(t,1));
    qafter(t,0)=sum(qa(t,0));
    qafter(t,1)=sum(qa(t,1));
    dvariable sumqa=sum(qafter(t));
    qa(t)/=sumqa;
    qafter(t,0)/=sumqa;
    qafter(t,1)/=sumqa;
    ff1-=log(1.e-50+sumqa); // add small constant to avoid log(0)
  }  
  ff+=ff1; //ff1 is minus the log-likelihood
  ff+=.1*norm2(f); // add small penalty to stabilize estimation
\endexample
The \PROS\ is where the calculation of the objective function are 
carried out.  First the transition matrix {\tt P} is calculated from
the {\tt Pcoff}. The function {\tt colsum} forms a {\tt vector} whose
elements are the column sums of the {\tt matrix}. This is used to normalize
{\tt P
} so that its columns sum to~$1$. A penalty is added to
the objective function for the colum sums so that the hessian matrix 
with respect to the independent variables will not be singular. This does not affect
 the ``statistical'' properties of the parameters of interest.
The matrix {\tt z} is calculated using a transformed matrix because
\ADM\ deals with vector rows better than columns.
The probability distribution for the states in period~1, {\tt qb1}
is set equal to the uncondtional distribution for a Markov process
in terms of its transition matrix, {\tt P}, as discussed in Hamilton (1994).
The transition matrix is used to compute the probability distribution of the
states in  periods $(2,1)$, $(3,2,1)$, $(4,3,2,1)$, and finally
$(5,4,3,2,1)$. For the last quintuplet this is the probability  
distribution before observing {\tt y(5)}. The quantities {\tt eps}
in the code correspond to the possible 
realized values of the random variable $\epsilon$. The quantities
{\tt qa} and  {\tt qb} correspond to $q_a$ and $q_b$ in the documentation.
The {\tt sum} function is defined for arrays of any dimension and simply
forms the sum of all the components. In \ADM\ if {\tt xx} is an n dimensional
array then {\tt x(i)} is an n-1 dimensional array. So the statement
\beginexample
    qbefore(t,0)=sum(qb(t,0));
\endexample
\noindent takes the sum of the probabilities for the 
sixteen quintuples of states at time period {\tt t} through {\tt t-4}
for which the state at time period {\tt t} is $0$. These are used in the
\RS\ to write out a report of the estimated state probabilities at time
period {\tt t} before and after observing {\tt y(t)}.
\beginexample
REPORT_SECTION
  dvar_matrix out(1,2,op1,nperiods);
  dvar_matrix out1(1,1,op1,nperiods);
  out(1)=trans(qbefore)(1);
  out(2)=trans(qafter)(1);
  {
    ofstream ofs("qbefore.rep");
    out1(1)=trans(qbefore)(0);
    ofs << trans(out1)<< endl;
  }  
  {
    ofstream ofs("qafter.rep");
    out1(1)=trans(qafter)(0);
    ofs << trans(out1) << endl;
  }  
  report << "#qbefore    qafter" <<  endl;
  report << setfixed << setprecision(3) << setw(7) << trans(out) << endl;
\endexample
\XX{REPORT\_SECTION}{example of}
The \RS\ is used to report any result in a manner not already
carried out by the models default behaviour. The probabilites of
being in state $0$ before and after observing {\tt y(t)}
are printed into the files {\tt qbefore.rep} and {\tt qafter.rep}.
These vectors were stored in files so that they could be easily imported into 
graphing programs. The results are very similar to figure 1 in Hamilton
(1989) as one might hope.
\beginexample
RUNTIME_SECTION
  maximum_function_evaluations 20000 
  convergence_criteria 1.e-6 
\endexample
\X{\tt convergence\_criteria} 
\X{{\tt maximum\_function\_evaluations}} 
The  {\tt maximum\_function\_evaluations 20000} will simply let the program
run a long time by setting the maximum number of function evaluations
in the function minimizer equal to 20,000 (nowhere near this many are actually
needed.) 
The   convergence\_criteria 1.e-6 
was needed becuase the default value of {\tt 1.e-4} caused the program to 
exit from the miniization before convergence had been achieved.
\X{\tt arrmblsize}
\X{TOP\_OF\_MAIN section}
\XX{gradient\_structure::set\_ARRAY\_MEMBLOCK\_SIZE}{the correct way to set}
\X{gradient\_structure::set\_GRADSTACK\_BUFFER\_SIZE}
\beginexample
TOP_OF_MAIN_SECTION
  arrmblsize=500000;
  gradient_structure::set_GRADSTACK_BUFFER_SIZE(200000);
  gradient_structure::set_CMPDIF_BUFFER_SIZE(2100000);
\endexample
The  {\tt TOP\_OF\_MAIN\_SECTION} is for including code which
will be included at the top of the {\tt main()} function in the
\cplus\ program. Any desired legal code may be included. There are
a number of common statements which are used to control aspects
of \ADM's performance. 
The statement {\tt arrmblsize=500000;} reserves 500,000 bytes of memory
for variable objects. If it is not large enough a message will be printed
out at run time. See the index for references to more
discussion of this matter.
The statements {\tt gradient\_structure::set\_GRADSTACK\_BUFFER\_SIZE(200000);}
and {\tt gradient\_structure::set\_CMPDIF\_BUFFER\_SIZE(2100000);} 
set the amount of memory that \ADM\ reserves for variable objects.
Setting these is a matter of tuning for optimum performance. If you have 
a lot of memory available making them larger may improve performance.
However models will run without including these statements
as long as there is enough memory for \ADM's temporary files. 
\X{The GLOBALS\_SECTION}
\beginexample
GLOBALS_SECTION  
  #include <admodel.h>
  
  dvariable phi(const dvariable& a1,const dvariable& a2,const dvariable& a3,
    const dvariable& a4,const dvar_vector& f)
  {
    return   a1*f(1)+a2*f(2)+a3*f(3)+a4*f(4);
  }  
\endexample
The {\tt GLOBALS\_SECTION} is used to include statements at the top of the
file containing the {\tt CPP} program. This is generally where global
declarations are made in \cplus, hence its name. However it may be
used for any legal statments such as including header files for
the users data structures etc. In this case it has been used to
define the function {\tt phi} which is used to simplify  the code
for the model's calculations. The header file {\tt admodel.hpp} is
included to define the {\tt AUTODIF} structures used in the
definition of the function. This header is automatically included
near the top of the file,
but this would be too late as {\tt GLOBALS\_SECTION} material
is included first.
\mysection{Results of the analysis} 
The parameter estimates for the initial parameters are written into a file
{\tt HAM4.PAR}. This is an ASCII file wich can be easily read. (The results
are also stored in a binary file {\tt HAM4.BAR} which can be used to
restart the model with more accurate parameters estimates.)
\beginexample
# Objective function value = 60.8934
# f:
 0.0139989 -0.0569580 -0.246292 -0.212250
# Pcoff:
 0.754133 0.0955834
 0.245118 0.900333
# a0:
-0.357964
# a1:
1.52138
# smult:
0.281342
\endexample
The estimates are almost identical to those reported in 
Hamilton (1989)\footnote{\rmfoot Our method for parameterizing the 
intial state probability distribution {\ttfoot qb1} is slightly
different from Hamilton's which would explain the small discrepancy.}
The first line reports the value of the log-likelihood function. This value can
be used in hypothesis (likelihood-ratio) tests.  
the file {\tt ham5.par}a for the fifth order autoregressive model fit to
the data in Hamilton (1989) is shown below. there is one more parameter in
this model. Twice the difference in the log-likelihood functions is 
$2(60.89-59.60)=2.58$. For one extra parameter the  $95\%$
significance level is $3.84$, the improvement in fit is not significant. 

\beginexample
# Objective function value = 59.6039
# f:
 -0.0474771 -0.113829 -0.241966 -0.225535 -0.192585
# Pcoff:
 0.779245 0.0951739
 0.219775 0.900719
# a0:
-0.271318
# a1:
1.46301
# smult:
0.259541
\endexample
\bigbreak
The plot of {\tt qa} and {\tt qb} demonstrates the extra information 
about the probability distribution of the current state contained in
in the current value of {\tt y(t)}. 

\medskip
\vbox{
\hbox to\textwidth{\hfill
\beginpicture
  \setplotsymbol ({\eightrm .})
  \setcoordinatesystem units <3.2in,2.5in>
  \setplotarea x from -0.02 to 1.3, y from -.02 to 1.02 
  \axis left 
  /
  \axis bottom label {Apriori and aposteriori
     Probabilitites of Being in State 0 in Period t}
  /
  \axis left 
    ticks
    numbered from .0 to 1.0 by .2 
  /
\linethickness .8 truept
\setdashpattern <1pt,1pt,1pt,1pt>
{
   \setcolor\cmykRed  % turn on color 
 %  \setcolor\cmykSeaGreen  % turn on color 
 \plot  "qbefore4.tex" 
   \setcolor\cmykMidnightBlue  % turn on color 
\setdashpattern <1pt,3pt,4pt,3pt>
 \plot  "qafter4.tex" 
   \setcolor\cmykRed  % turn on color 
 \put {\hbox{\tt qb }} at .43 0.9
 \putrule from .48 .9 to .64 0.9
   \setcolor\cmykMidnightBlue  % turn on color 
 \put {\hbox{\tt qa }} at .43 .8
 \putrule from .48 .8 to .64 .8
   \setcolor\cmykBlack  % turn on color 
}
% \put {\hbox{\tt qb }} at .43 0.9
% \putrule from .48 .9 to .64 0.9
%\setdashpattern <1pt,3pt,4pt,3pt>
%{
  % \setcolor\cmykMidnightBlue  % turn on color 
 %\plot  "qafter4.tex" 
%}
 %\put {\hbox{\tt qa }} at .43 .8
% \putrule from .48 .8 to .64 .8
\endpicture
\hfill
}}
\medskip
% \vbox{
% \hbox{
% \beginpicture
%   \setplotsymbol ({\eightrm .})
%   \setcoordinatesystem units <3.2in,2.0in>
%   \setplotarea x from -0.02 to 1.3, y from -.02 to 1.02 
%   \axis left 
%   /
%   \axis bottom label {Aposteriori 
%      Probabilitites of Being in State 0 in Period t}
%   /
%   \axis left 
%     ticks
%     numbered from .0 to 1.0 by .2 
%   /
% \linethickness .8 truept
% \setdashpattern <1pt,1pt,1pt,1pt>
% % \plot  "qbefore4.tex" 
%  \put {\hbox{\tt qb }} at .43 0.9
%  \putrule from .48 .9 to .64 0.9
% \setdashpattern <1pt,3pt,4pt,3pt>
%  \plot  "qafter4.tex" 
%  \put {\hbox{\tt qa }} at .43 .8
%  \putrule from .48 .8 to .64 .8
% \endpicture
% \hfill
% }}
\bigskip
\vbox{
The standard deviation and correlation report for the model are 
in the file {\tt ham4.cor} reproduced below.
\htmlbegintex
{ \openup -2pt
\obeylines\obeyspaces\ttsix
 index name  value    std dev    1     2     3     4     5     6     7     8     9    10    11    12    13    14   15   
   1  f      1.39e-02 1.20e-01  1.00
   2  f     -5.69e-02 1.37e-01  0.33  1.00
   3  f     -2.46e-01 1.06e-01  0.33  0.29  1.00
   4  f     -2.12e-01 1.10e-01  0.43  0.26  0.17  1.00
   5  Pcoff  7.54e-01 5.39e-01  0.00  0.04  0.01  0.00  1.00
   6  Pcoff  9.55e-02 7.58e-02  0.04  0.05  0.02  0.03 -0.04  1.00
   7  Pcoff  2.45e-01 1.97e-01 -0.01 -0.11 -0.03 -0.01  0.77  0.04  1.00
   8  Pcoff  9.00e-01 6.20e-01 -0.00 -0.00 -0.00 -0.00  0.00  0.83 -0.00  1.00
   9  a0    -3.57e-01 2.65e-01  0.27  0.56  0.25  0.21  0.08  0.07 -0.23 -0.00  1.00
  10  a1     1.52e+00 2.63e-01 -0.31 -0.57 -0.29 -0.25 -0.07 -0.04  0.21  0.00 -0.96  1.00
  11  smult  2.81e-01 1.25e-01  0.54  0.69  0.48  0.45  0.06  0.05 -0.17 -0.00  0.82 -0.84  1.00
  12  P      7.54e-01 9.65e-02  0.02  0.24  0.07  0.03  0.17 -0.08 -0.48  0.00  0.47 -0.44  0.36  1.00
  13  P      9.59e-02 3.77e-02  0.09  0.10  0.04  0.06 -0.02  0.49  0.08 -0.05  0.14 -0.09  0.11 -0.16  1.00
  14  P      2.45e-01 9.65e-02 -0.02 -0.24 -0.07 -0.03 -0.17  0.08  0.48 -0.00 -0.47  0.44 -0.36 -1.00  0.16  1.00
  15  P      9.04e-01 3.77e-02 -0.09 -0.10 -0.04 -0.06  0.02 -0.49 -0.08  0.05 -0.14  0.09 -0.11  0.16 -1.00 -0.16 1.00
}
} %\vbox
\htmlendtex
\bigbreak
\mysection{Extending Hamilton's model to a fifth order autoregressive process} 
Hamilton (1989, page 372) remarks that investigating higher order
autoregressive processes might be a fruitful area of research.
The form of the model is. The first extension of the model is a fifth
order autoregressive process.  
\begin{equation}
\myeq{
Y_t=a_0+a_1s_{ti}+Z_t }
\label{chp5:yy1}
\end{equation} 
and the state variables $Z_t$ satisfy the fourth order 
autoregressive relationship
\begin{equation}
\myeq{
Z_t=f_1Z_{t-1}+f_2Z_{t-2}+f_3Z_{t-3}+f_4Z_{t-4}+f_5Z_{t-5}+
  \epsilon_t}
\label{chp5:yy2}
\end{equation} 
which extend equations \number\mychapno.1 and \number\mychapno.2.
The {\tt TPL} file {\tt ham5.tpl} for the fifth order autoregressive model
is reproduced here. By employing higher dimensional arrays the conversion of
the {\tt TPL} file from a fourth order autoregressive 
process to a fifth order one
is largely formal. An experienced \ADM\ user can carry out the
modifications in under 1 hour. Places where modifications were made were tagged
with the comment {tt //!!5}. 
\beginexample
DATA_SECTION
  init_number a1init   // read in the initial value of a1 with the data
  init_int nperiods1   // the number of observations
  int nperiods  // nperiods-1 after differencing
 !! nperiods=nperiods1-1;
  init_vector yraw(1,nperiods1)  //read in the observations
  vector y(1,nperiods)   // the differenced observations
 !! y=100.*(--log(yraw(2,nperiods1)) - log(yraw(1,nperiods))); 
  int order 
  int op1  
 !! order=5; // !!5 order of the autoregressive process
 !! op1=order+1;
  int nstates  // the number of states (expansion and contraction)
 !! nstates=2;
PARAMETER_SECTION
  init_vector f(1,order,1)  // coefficients for the atuoregressive
                            // process
  init_bounded_matrix Pcoff(0,nstates-1,0,nstates-1,.01,.99,2)  
        // determines the transition matrix for the markov process
  init_number a0(5)  // equation 4.3 in Hamilton (1989)
  init_bounded_number a1(0.0,10.0,4);  
 !! if (a0==0.0) a1=a1init;  // set initial value for a1 as specified
                     // in the top of the file nham4.dat
  init_bounded_number smult(0.01,1,3)  // used in computing sigma
  matrix z(1,nperiods,0,1)  // computed via equation 4.3 in 
                          // Hamilton (1989)
  matrix qbefore(op1,nperiods,0,1);  // prob. of being in state before
  matrix qafter(op1,nperiods,0,1); // and after observing y(t)
  number sigma // variance of epsilon(t) in equation 4.3
  number var  // square of sigma
  sdreport_matrix P(0,nstates-1,0,nstates-1);
  number ff1;
  vector qb1(0,1); 
  matrix qb2(0,1,0,1); 
  3darray qb3(0,1,0,1,0,1);
  4darray qb4(0,1,0,1,0,1,0,1);
  5darray qb5(0,1,0,1,0,1,0,1,0,1); // !!5
  7darray qb(op1,nperiods,0,1,0,1,0,1,0,1,0,1,0,1); 
  7darray qa(op1,nperiods,0,1,0,1,0,1,0,1,0,1,0,1);
  7darray eps(op1,nperiods,0,1,0,1,0,1,0,1,0,1,0,1);
  7darray eps2(op1,nperiods,0,1,0,1,0,1,0,1,0,1,0,1);
  7darray prob(op1,nperiods,0,1,0,1,0,1,0,1,0,1,0,1);
  objective_function_value ff;
PROCEDURE_SECTION
  P=Pcoff;
  dvar_vector ssum=colsum(P);  // forma a vector whose elements are the
                           // sums of the columns of P
  ff+=norm2(log(ssum)); // this is a penalty so that the hessian will
                        // not be singular and the coefficients of P 
                        // will be well defined
  // normalize the transition matrix P so its columns sum to 1
  int j;
  for (j=0;j<=nstates-1;j++)
  {
    for (int i=0;i<=nstates-1;i++)
    {
      P(i,j)/=ssum(j);
    }
  }  

  dvar_matrix ztrans(0,1,1,nperiods);
  ztrans(0)=y-a0;
  ztrans(1)=y-a0-a1;
  z=trans(ztrans);
  int t,i,k,l,m,n,p;
  
  qb1(0)=(1.0-P(1,1))/(2.0-P(0,0)-P(1,1)); // unconditional distribution
  qb1(1)=1.0-qb1(0);
  
  // for periods 2 through 4 there are no observations to condition
  // the state distributions on so we use the unconditional distributions
  // obtained by multiplying by the transition matrix P.
  for (i=0;i<=1;i++) {
    for (j=0;j<=1;j++) qb2(i,j)=P(i,j)*qb1(j);  
  }
  
  for (i=0;i<=1;i++) {
    for (j=0;j<=1;j++) {
      for (k=0;k<=1;k++) qb3(i,j,k)=P(i,j)*qb2(j,k); 
    }  
  }
  
  for (i=0;i<=1;i++) {
    for (j=0;j<=1;j++) {
      for (k=0;k<=1;k++) {
        for (l=0;l<=1;l++) qb4(i,j,k,l)=P(i,j)*qb3(j,k,l); 
      }
    }  
  }
  // !!5
  for (i=0;i<=1;i++) {
    for (j=0;j<=1;j++) {
      for (k=0;k<=1;k++) {
        for (l=0;l<=1;l++) {
          for (m=0;m<=1;m++) qb5(i,j,k,l,m)=P(i,j)*qb4(j,k,l,m); 
	}  
      }
    }  
  }
  // qb(6) is the probabilibility of being in one of 64
  // states (64=2x2x2x2x2x2) in periods 5,4,3,2,1 before observing
  // y(6)
  for (i=0;i<=1;i++) {
    for (j=0;j<=1;j++) {
      for (k=0;k<=1;k++) {
        for (l=0;l<=1;l++) {
          for (m=0;m<=1;m++) { // !!5
            for (n=0;n<=1;n++) qb(op1,i,j,k,l,m,n)=P(i,j)*qb5(j,k,l,m,n); 
          }	
	}  
      }
    }  
  }
  // now calculate the realized values for epsilon for all 
  // possible combinations of states
  for (t=op1;t<=nperiods;t++) {
    for (i=0;i<=1;i++) {
      for (j=0;j<=1;j++) {
        for (k=0;k<=1;k++) {
          for (l=0;l<=1;l++) {
            for (m=0;m<=1;m++) {
              for (n=0;n<=1;n++) { // !!5
	        eps(t,i,j,k,l,m,n)=z(t,i)-phi(z(t-1,j),
	          z(t-2,k),z(t-3,l),z(t-4,m),z(t-5,n),f);
                eps2(t,i,j,k,l,m,n)=square(eps(t,i,j,k,l,m,n));
              }		
	    }  
	  }
	}
      }	    		
    }  
  }  
  // calculate the mean squared "residuals" for use in 
  // "undimensionalized" parameterization of sigma
  dvariable eps2sum=sum(eps2);
  var=smult*eps2sum/(64.0*(nperiods-4));  //!!5
  sigma=sqrt(var);
  
  for (t=op1;t<=nperiods;t++) {
    for (i=0;i<=1;i++) {
      for (j=0;j<=1;j++) {
        for (k=0;k<=1;k++) {
          for (l=0;l<=1;l++)  //!!5
	    prob(t,i,j,k,l)=exp(eps2(t,i,j,k,l)/(-2.*var))/sigma;
        }	    
      }	    		
    }  
  }  
  
  for (i=0;i<=1;i++) {
    for (j=0;j<=1;j++) {
      for (k=0;k<=1;k++) {
        for (l=0;l<=1;l++) {
          for (m=0;m<=1;m++) {
            for (n=0;n<=1;n++) qa(op1,i,j,k,l,m,n)= qb(op1,i,j,k,l,m,n)*
	      prob(op1,i,j,k,l,m,n);
          }	      
        }
      }
    }	    		
  }  
  ff1=0.0;
  qbefore(op1,0)=sum(qb(op1,0));
  qbefore(op1,1)=sum(qb(op1,1));
  qafter(op1,0)=sum(qa(op1,0));
  qafter(op1,1)=sum(qa(op1,1));
  dvariable sumqa=sum(qafter(op1));
  qa(op1)/=sumqa;
  qafter(op1,0)/=sumqa;
  qafter(op1,1)/=sumqa;
  ff1-=log(1.e-50+sumqa);
  for (t=op1+1;t<=nperiods;t++) { // notice that the t loop includes 2 
    for (i=0;i<=1;i++) {      // i,j,k,l,m blocks
      for (j=0;j<=1;j++) {
        for (k=0;k<=1;k++) {
          for (l=0;l<=1;l++) {
            for (m=0;m<=1;m++) {
              for (n=0;n<=1;n++) { //!!5
                qb(t,i,j,k,l,m,n).initialize(); 
	        // here is where having 6 dimensional arrays makes the
	        // formula for moving the state distributions form period
	        // t-1 to period t easy to program and understand.
	        // Throw away  n and accumulate its two values into next
	        // time period after multiplying by transition matrix P
                for (p=0;p<=1;p++) qb(t,i,j,k,l,m,n)+=P(i,j)*
		  qa(t-1,j,k,l,m,n,p); 
              }		
            }
	  }
	}
      }	    		
    }  
    for (i=0;i<=1;i++) {
      for (j=0;j<=1;j++) {
        for (k=0;k<=1;k++) {
          for (l=0;l<=1;l++) {
            for (m=0;m<=1;m++) { // !!5
              for (n=0;n<=1;n++) qa(t,i,j,k,l,m,n)=qb(t,i,j,k,l,m,n)*
	          prob(t,i,j,k,l,m,n);
            }		  
	  }
	}
      }	    		
    }  
    qbefore(t,0)=sum(qb(t,0));
    qbefore(t,1)=sum(qb(t,1));
    qafter(t,0)=sum(qa(t,0));
    qafter(t,1)=sum(qa(t,1));
    dvariable sumqa=sum(qafter(t));
    qa(t)/=sumqa;
    qafter(t,0)/=sumqa;
    qafter(t,1)/=sumqa;
    ff1-=log(1.e-50+sumqa);
  }  
  ff+=ff1;
  ff+=.1*norm2(f);
REPORT_SECTION
  dvar_matrix out(1,2,op1,nperiods);
  out(1)=trans(qbefore)(1);
  out(2)=trans(qafter)(1);
  {
    ofstream ofs("qbefore4.tex");
    for (int t=5;t<=nperiods;t++)
    {
      ofs << (t-4)/100. << " " << qbefore(t,0) << endl;
    }
  }  
  {
    ofstream ofs("qafter4.tex");
    for (int t=5;t<=nperiods;t++)
    {
      ofs << (t-4)/100. << " " << qafter(t,0) << endl;
    }
  }  
  report << "#qbefore    qafter" <<  endl;
  report << setfixed << setprecision(3) << setw(7) << trans(out) << endl;
RUNTIME_SECTION
  maximum_function_evaluations 20000 
  convergence_criteria 1.e-6 
TOP_OF_MAIN_SECTION
  arrmblsize=500000;
  gradient_structure::set_GRADSTACK_BUFFER_SIZE(400000);
  gradient_structure::set_CMPDIF_BUFFER_SIZE(2100000);
  gradient_structure::set_MAX_NVAR_OFFSET(500);
GLOBALS_SECTION  
  #include <fvar.hpp>
   // !!5
  dvariable phi(const dvariable& a1,const dvariable& a2,const dvariable& a3,
    const dvariable& a4,const dvariable& a5,const dvar_vector& f)
  {
    return   a1*f(1)+a2*f(2)+a3*f(3)+a4*f(4)+a5*f(5);
  }  
\endexample
\endchapter
\htmlnewfile
\def\chapno{5}
\mychapter{Econometric Models -- simultaneous equations}
\input simequat.tex
\endchapter

\htmlnewfile
\mychapter{Truncated Regression}
\input truncreg 
\endchapter


\htmlnewfile
\mychapter{Multivariate GARCH}
\input magarch-april2000 
\endchapter

\htmlnewfile
\mychapter{The Kalman filter}
\input kalman.tex
\endchapter
\htmlnewfile
\mychapter{
Applying the Laplace approximation to the Generalized 
Kalman Filter -- with an application to Stochastic Volatility Models}
\input kal_lap.tex
\htmlnewfile
\mychapter{
Using Vectors of initial parameter types} 
\input advectors.tex
\endchapter
\htmlnewfile
\mychapter{Creating Dynamic Link Libraries with AD Model Builder}
\input admb-dll.tex 
\endchapter
\htmlnewfile
\mychapter{Command line options}
\input commandl.tex
\endchapter
\htmlnewfile
%\mychapter{Parallel Processing on clusters}
%\input parallel1.tex
%\endchapter

\htmlnewfile
\mychapter{Writing Adjoint Code}
\input adjoint.tex
\endchapter

%\htmlnewfile
%\mychapter{The Random Effects Module}
%\input randeff.tex
%\endchapter

% \htmlnewfile
% \mychapter{Nonlinear state space models -- beyond the Kalman Filter
% -- the Holistic approach}
% \input nlss2 
% \endchapter

\htmlnewfile
\mychapter{Truncated Regression}
\input truncreg 
\endchapter

\htmlnewfile
\mychapter{All the functions in AD Model Builder}
\input allfun.tex
\endchapter
\htmlnewfile
%\def\chapno{7}
\mychapter{Miscellaneous and Advanced Features of \ADM}
\X{adstring class}
\XX{strings}{reading from DAT file}
\mysection{Using strings and labels in the TPL file}
For purposes of this manual a label is a string that does not
have any blanks in it. Such strings can be read in from the
{\tt data} file using the {\tt init\_adstring} declaration
as in
\beginexample
  DATA_SECTION
    init_adstring s
\endexample
The {\tt DAT} file should contain something like
\beginexample
  # label to be read in
    my_model_data   
\endexample
\noindent When the program runs the {\tt adstring} object {\tt s}
should contain the string ``my\_model\_data''. White space at
the beginning is ignored and following white space terminates
the input of the object.

Discussions of the various
operations on adstring class members are found elsewhere in the
manual.

\mysection{Using other class libraries in \ADM\ programs}
\XX{Using other \cplus\ classes in \ADM}{!!CLASS}
A useful feature of \cplus\ is its open nature.  This means that the user
can combine several class libraries into one program. In general this simply
involves including the necessary header files in the program and then
declaring the appropriate class instances in the program.
Instances of external classes can be declared in \ADM\ program in several ways.
They can always be declared in the procedure or report section 
of the program as local objects. It is sometimes desired to include instances
of external classes in a more formal way into an \ADM\ program. 
This section describes how to include them into the \DS\ or \PS.
After that they can be referred to as though they were part of the 
\ADM\ code (except for the technicalities to be discussed below).
 
\ADM\ employs a strategy of late initialization of class members. The reason for
this is to allow time for the user too carry out any calculations which
may be necessary for determining parameter values etc. which are used
in the initializatin of the object. Because of the nature of constructors in
\cplus\ this means that every object declared in the \DS\ or the \PS
must have a default constructor which takes no arguments. The actual
allocation of the object is carried out by a class member function 
named {\tt allocate} which takes any desired arguments.
Since external classes will not generally satisfy these requirments
a different strategy is employed  for these classes. 
A pointer to the object is included in the 
appropriate \ADM\ class. This pointer has the prefix {\tt pad\_} inserted before
the name of the object. The pointer to {\tt myobj} would have
the form {\tt pad\_myobj}. 
\beginexample
 !!CLASSfooclass myobj( .... )
\endexample
The user can refer to the object in the code simply by using its name.

\mysection{Appendix 1 -- The regression function}
The {\tt robust\_regression} function calculates the log-likelihood function
for the standard statistical model of independent
normally distributed errors with mean 0 and equal variance.
The code is written in terms of \AD\ objects such as
{\tt dvariable} and {\tt dvar\_vector}. They are
described in the \AD\ User's Manual.

\beginexample
dvariable regression(const dvector& obs,const dvar_vector& pred)
{
  double nobs=double(size_count(obs));  // get the number of
                                        // observations
  dvariable vhat=norm2(obs-pred);  // sum of squared deviations
  vhat/=nobs;                      //mean of squared deviations
  return (.5*nobs*log(vhat));     //return log-likelihood value
}
\endexample

\mysection{Appendix 2 -- \ADM\ types}
The effect of a declaration depends on whether it occurs in the
\DS\ or in the \PS. Objects declared in the \DS\ are constant,
that is like data. Objects declared in the \PS\ are variable,
that is like the parameters of the model which are to be estimated.
Any objects which depend on variable objects must themselves be
variables objects, that is they are declared in the \PS\ and
not in the \DS. 

In the \DS\ the prefix {\tt init\_} indicates
that the object is to be read in from the data file. 
In the
\PS\ the prefix indicates that the object is an initial 
parameter whose value will be used to calculate the value of other
(non initial) parameters. In the \PS\ initial parameters
will either have their values read in from a parameter
file or will be initialized 
with their default initial values. The actual 
default values used can be modified in the {\tt INITIALIZATION\_SECTION}.
From a mathematical point of view objects declared with the
{\tt init\_} prefix are independent variables which are used to
calculate the objective function being minimized.
\X{INITIALIZATION\_SECTION}
\X{three dimensional arrays}

The prefixes {\tt bounded\_} and {\tt dev\_} can only be used
in the \PS. The prefix {\tt bounded\_} restricts the numerical values
which an object can take on to lie in a specified  bounded
interval. The prefix {\tt dev\_} can only be applied to
the declaration of vector objects. It has the
effect of restricting the sum of the individual components
of the vector object to sum to 0.
 
The prefix {\tt sdreport\_} can only be used in the \PS.
An object declared with this prefix will appear in the
covariance matrix report. This provides a convenient method for
obtaining estimates for the variance of any parameter
which may be of interest. Note that the prefixes
{\tt sdreport\_} and {\tt init\_} can not both be applied to the
same object. There is no need to do so since 
initial parameters are automatically included in the
standard deviations report.
\ADM\ also has three and four dimensional arrays. They are declared like
\XX{arrays}{three dimensional}
\XX{arrays}{four dimensional}
\X{three dimensional arrays}
\X{four dimensional arrays}
\beginexample
3darray dthree(1,10,2,20,3,10)
4darray df(1,10,2,20,3,10)
init_3darray dd(1,10,2,20,3,10)    // data section only
init_4darray dxx(1,10,2,20,3,10)   // data section only
\endexample
The following table contains a summary of declarations and the types of 
objects associatated with them in \ADM. The types {\tt dvariable,
dvector, dmatrix, d3\_array, dvar\_vector, dvar\_matrix, and dvar3\_array}
are are described in the AUTODIF Users's manual.
\htmlbegintex
{\stt
\halign to \hsize{#\hfil&\quad#\hfil&\quad#\hfil\cr
declaration&type of object&type of object\cr
&in \sDS&in \sPS\cr
\noalign{\medskip}
[init\_]int&int&int\cr
[init\_][bounded\_]number&double&dvariable\cr
[init\_][bounded\_][dev\_]vector&vector of doubles(dvector)&
vector of dvariables(dvar\_vector)\cr
[init\_][bounded\_]matrix&matrix of doubles(dmatrix)
 &matrix of dvariables(dvar\_matrix)\cr
[init\_]3darray&3 dimensional array of doubles
  &3 dimensional array of dvariables\cr
4darray&4 dimensional array of doubles
  &4 dimensional array of dvariables\cr
5darray&5 dimensional array of doubles
  &5 dimensional array of dvariables\cr
6darray&6 dimensional array of doubles
  &6 dimensional array of dvariables\cr
7darray&7 dimensional array of doubles
  &7 dimensional array of dvariables\cr
sdreport\_number&{\stt na}&dvariable\cr
likeprof\_number&{\stt na}&dvariable\cr
sdreport\_vector&{\stt na}&vector of dvariables(dvar\_vector)\cr
sdreport\_matrix&{\stt na}&matrix of dvariables(dvar\_matrix)\cr
}}
\htmlendtex

\mysection{Appendix 3 -- The profile likelihood}
\X{profile likelihood}
\XX{profile likelihood}{form of calculations}
We have been told that the profile likelihood as calculated in \ADM\\
for dependent variables may differ from that calculated by other authors.
This section will clarify what we mean by the term and motivate our
calculation.  

Let $(x_1,\ldots,x_n)$ be $n$ independent variables, $f(x_1,\ldots,x_n)$
be a probability distribution and
$g$ denote a dependent variable that is a real valued function of 
$(x_1,\ldots,x_n)$.
Fix a value $g_0$ for g and consider the integral
$$\int_{\{x:g_0-\epsilon/2\le g(x)\le g_0+\epsilon/2\}} f(x_1,..., x_n)$$
which is the probability that $g(x)$ has a value between 
$g_0-\epsilon/2$ and $g_0+\epsilon/2$. This probability depends 
on two quantities,
the value of $f(x)$ and the thickness of the region being integrated over.
We approximate $f(x)$ by its maximum value 
$\hat x(g)=\max_{x:g(x)=g_0}\{f(x)\}$. For the thickness we have
$g(\hat x+h)\approx g(\hat x)+<\nabla g(\hat x),h>=\epsilon/2$ where
$h$ is a vector perpendicular to the level set of $g$ at $\hat x$.
However  
$\nabla g$ is also perpendicular to the level set so 
$<\nabla g(\hat x),h>=\|\nabla g(\hat x)\| \|h\|$ so that 
$ \|h\|=\epsilon/(2\|g(\hat x)\|)$. Thus the integral is approximated by
 $\epsilon f(\hat x)/\|\nabla g(\hat x)\|$ and taking the derivative
with respect to $\epsilon$ yields 
$f(\hat x)/\|\nabla g(\hat x)\|$ which is the profile likelihood exprression
for a dependent variable.

\mysection{Appendix 4 -- Concentrated Likelihoods}
The log-likelihood function for a collection of $n$ observations $Y_i$
where the $Y_i$ are assumed to be normally distributed random variables with
mean $\mu$ and variance $\sigma^2$ has the  form
\begin{equation}
\myeq{-n\log(\sigma)-\sum_{i=1}^n \myfrac{(Y_i-\mu_i)^2}{2\sigma^2}}\label{ap:xx1}
\end{equation}
\X{concentrated likelihood}\XX{likelihood}{concentrated}
To find the maximum of this expression with respect to $\sigma$ take the derivative
of \ref{ap:xx1} with respect to $\sigma$ and set the resulting equation =0.
\begin{equation}
\myeq{-n/\sigma+\sum_{i=1}^n \myfrac{(Y_i-\mu_i)^2}{\sigma^3}=0}\label{ap:xx2}
\end{equation}
solving \ref{ap:xx2} for $\hat\sigma^2$ yields
\begin{equation}
\myeq{\hat\sigma^2=1/n\sum_{i=1}^n (Y_i-\mu_i)^2}\label{ap:xx3}
\end{equation}
and substituting this value into \ref{ap:xx1} yields
\begin{equation}
\myeq{-.5n\log(\sum_{i=1}^n (Y_i-\mu_i)^2)+\hbox{\rm const}}\label{ap:xx4}
\end{equation}
where const is a constant which can be ignored.
It follows that maximizing \ref{ap:xx1} is equivalent to maximizing
\begin{equation}
\myeq{-.5n\log(\sum_{i=1}^n (Y_i-\mu_i)^2)}\label{ap:xx5}
\end{equation}
Expression \ref{ap:xx5} is referred to as the concentrated log-likelihood.

See Harvey for more complicated examples of concentrated likelihoods.
\mysection{References}
\htmlbeginignore
\medskip
{
\rmsmall\parindent=0cm \parskip=\medskipamount
\everypar={\hangindent=.5cm \hangafter=1}
\htmlendignore

Bard, Yonathan. Nonlinear Parameter Estimation. 
Academic Press. N.Y. 1974

Gelman, Andrew., John B. Carlin, Hal S. Stern, and Donald B. Rubin.
Bayesian Data Analysis. Chapman and Hall. 

Harvey, Andrew C. Forecasting structural time series models and
the Kalman filter. Cambridge University Press. 1990.

Hilborn, Ray and Carl Walters. Quantitative Fisheries Stock
Assessment and Management: Choice, Dynamics, and Uncertainty.
1992.

\par
}

\mysection{How to order \ADM}
\ADM\ bundled with \AD\ is available for a wide variety of compilers on
Intel computers including Borland \cplus under WIN32, 
Visual \cplus (32 bit) and the ``GNU'' mingw32 \cplus\ compiler.  
Various flavours of Linux on Intel platforms are also supported.

Multi-user and site licenses
are available.
Contact 
\htmlbeginignore
\medskip
{\openup -4pt\obeylines
\htmlendignore
users@admb-project.org
http://www.admb-project.org/
\htmlbeginignore
}
\htmlendignore
\endchapter
%\input chadvanced.tex
\htmlnewfile
%\def\chapno{8}
{
%  \advance \vsize by -20pt
%\mychapter{Index}
%\readindexfile{i}
\pagestyle{xxx}

\def\hrefname#1#2{{
  \setbox0\hbox{#2 }
\myht=\ht0
\myhtt=\ht0
\mydp=\dp0
\mydpp=\dp0
\mywidth=\wd0
\advance \myht by 4pt
\advance \myhtt by 3pt
\advance \mydp by 4pt
\advance \mydpp by 3pt
\advance \mywidth by 2pt
\hbox{\vrule height\myht depth\mydp width 0pt
\pdfstartlink height \myhtt depth \mydpp attr {/C [0.9 0 0.0] /Border [0 0 1] } goto name {page.#1}
 \ #2

\pdfendlink}}}

%\endchapter
\htmlnewfile
%\def\chapno{2}
\twocolumn
\mychapter{Index}

\input all_ad.ind
\onecolumn
%\printindex
\vfill
}
%\endchapter
\end{document}
\bye
\bye
