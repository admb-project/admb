  %
  % $Id:   $
  %
  % $Rev   ::                       $: Revision of last commit
  % $Author::                       $: Author of last commit
  % $Date  ::                       $: Date of last commit
  %
  % Author: David Fournier
  % Copyright (c) 2008 Regents of the University of California
  %


\mysection{Brief description of the optimization routines}
At present there are three optimization routines include 
with \ADM. They are a quasi-Newton routine, a limited memory
quasi-Newton routine both of which use first derivatives 
and the derivative free simplex algorithm.  

The quasi-Newton  and the limited memory quasi-Newton
methods employ the history of minimizing steps and the
values of the gradient at each step to build up an 
approximation to the Hessian matrix or its inverse.
for the limited memory method only the last $m$ steps are
used where $m$ can be set by the user. The major advantage 
of the limited memory method is that the storage requirements
are $O(n)$ where $n$ is the number of independent variables.
for the quasi-Newton method the storage reuirements are
$O(n^2)$.
