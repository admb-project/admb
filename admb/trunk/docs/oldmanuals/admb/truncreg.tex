% $Id$
%
% Author: David Fournier
% Copyright (c) 2008 Regents of the University of California
%

%\magnification=1400
%\def\mysection#1{{\noindent\bf\medskip #1\medskip}}
\mysection{Truncated Linear Regression}

The linear regression model we consider here has the form
$$Y_i=\sum_{j=1}^m a_j x_{ij} +\epsilon_i$$
where the $Y_i$ for $i=1,\ldots,n$ are the $n$ observations
and the $a_j$ are $m$ parameters to be estimated.  
The $\epsilon_i$ are assumed to be normally distributed random 
variables with mean $0$ and variance $v$ 

Let $r_i=Y_i-\sum_{j=1}^m a_j x_{ij}$. The log-likelihood function 
for the standard regression model is give by
$$ -.5n\log(v) - \sum_{i=1}^n  {r_i^2 \over 2v}$$
Now assume that we only consider the $Y_i$ for $Y_i/ge 0$
i.e. the left truncated situation.
The probability that $Y_i\ge 0$ is equal to the probability that
$\epsilon_I>-\sum_{j=1}^m a_j x_{ij} $. This is equal to 
$1-\Phi(-\sum_{j=1}^m a_j x_{ij} /v)$ where
$$\Phi(u)={1\over\sqrt{2\pi}}\int_{-\infty}^u \exp(-t^2/2)\,dt$$
For this truncated regression the log-likelihood function
has the logarithm of this quantity subtracted from it
so it becomes 
$$ -.5n\log(v) - \sum_{i=1}^n  {r_i^2 \over 2v}
   -\log\big(1-\Phi(-\sum_{j=1}^m a_j x_{ij} /v)\big)$$
If instead we consider the right truncated case where only $Y_i<0$
are considered the log-likelihood function becomes
$$ -.5n\log(v) - \sum_{i=1}^n    {r_i^2 \over 2v}
   -\log\big(\Phi(-\sum_{j=1}^m a_j x_{ij} /v)\big)$$
 

To parameterize $v$ we introduce a new parameter $a$ satisfying
the condition $v=a\hat v$ where 
$\hat v={1\over n}\sum_{i=1}^n  r_i^2$ is the usual maximum liklihood
estimate for $v$. This leads to more
numerically stable behaviour. In terms of a the 
expression for the log-likelihood simplifies to
$$ -.5n\log(a)-.5n\log(\hat v) - {n\over 2a}
   -\log\big(1-\Phi(-\sum_{j=1}^m a_j x_{ij}/(a \hat v))\big)$$
\mysection{The AD Model Builder Truncated Regression Program}

Here are the contents of the file {\tt truncreg.tpl}.

\beginexample
DATA\_SECTION
  init\_int nobs
  init\_int m
  init\_int trunc\_flag
  init\_matrix data(1,nobs,1,m+1)
  vector Y(1,nobs)
  matrix X(1,nobs,1,m)
 LOC\_CALCS
  Y=column(data,1);
  for (int i=1;i<=nobs;i++)
  {
    X(i)=data(i)(2,m+1).shift(1);
  }
PARAMETER\_SECTION
  sdreport\_number sigma
  number vhat
  init\_bounded\_number log\_a(-5.0,5.0);
  sdreport\_number a
  init\_vector u(1,m)
  objective\_function\_value f
PROCEDURE\_SECTION
  a=exp(log\_a);
  dvar\_vector pred=X*u;
  dvar\_vector res=Y-pred;
  dvariable r2=norm2(res); 
  vhat=r2/nobs; 
  dvariable v=a*vhat;
  sigma=sqrt(v);

  dvar\_vector spred=pred/sigma;
  f=0.0;
  switch (trunc\_flag)
  {
  case -1:  // left\_truncated
    {
      for (int i=1;i<=nobs;i++)
      {
        f+=log(1.00001-cumd\_norm(-spred(i)));
      }
    }
    break;
  case 1:   // right truncated
    {
      for (int i=1;i<=nobs;i++)
      {
        f+=log(0.99999*cumd\_norm(-spred(i)));
      }
    }
    break;
  case 0:   // no truncation
    break;
  default:
    cerr << "Illegal value for truncation flag" << endl;
    ad\_exit(1);
  }
  f+=0.5*nobs*log(v)+0.5*r2/v;


REPORT\_SECTION
  report << "#u " << endl << u << endl;
  report << "#sigma " << endl << sigma << endl;
  report << "#a " << endl << a << endl;
  report << "#vhat " << endl << vhat << endl;
  report << "#shat " << endl << sqrt(vhat) << endl;
}
\endexample




