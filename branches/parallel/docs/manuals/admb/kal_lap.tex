% $Id$
%
% Author: David Fournier
% Copyright (c) 2008 Regents of the University of California
%

Let $y_i$ be an $N$-dimensional multivariate time series for
$i=1,\ldots,n$, where $y_i$ is a random vector with probability
density function $p(y_i | \alpha_i)$.  For each $i$,
the $\alpha_i$ are random vectors that satisfy the condition
\begin{equation}
  \alpha_i=T_i(\alpha_{i-1},y_{i-1})+\eta_i
\end{equation}
where $\mu_{\eta_i}=0$ and $\sigma^2_{\eta_i}=\sigma^2_\eta$.

Let $p(\alpha_1)$ be the probability density function for 
$\alpha_1$ before $y_1$ is observed. After observing 
$y_1$, we want to calculate the probability distribution of $\alpha_1$
given $y_1$. This is given by
\begin{equation}
  p(\alpha_1 | y_1)=p(y_1|\alpha_1)\,p(\alpha_1)\big/p(y_1)
\end{equation}
where
\begin{equation}\label{eq:xx6}
  p(y_1)=\int_{-\infty}^\infty p(y_1|\alpha_1)\, p(\alpha_1) \,\,\textrm{d}\alpha_1
\end{equation}
Let $\phi(y_1,\alpha_1)=\log(p(y_1|\alpha_1)p(\alpha_1))$ and
$\hat\alpha_1(y_1)=\max_{\alpha_1} \{\phi(y_1,\alpha_1)\}$.
Approximate $\phi$ by its second-order Taylor expansion in
$\alpha_1$ at $\hat\alpha_1$.
\begin{equation}
  \phi(y_1,\alpha_1)\approx \phi(y_1,\hat\alpha_1) 
  + D^2_{\alpha_1\alpha_1}\phi \big(y_1,\hat\alpha_1(y_1)\big) \big(\alpha_1-\hat\alpha_1(y),
  \alpha_1-\hat\alpha_1(y) \big)
\end{equation}
so that 
\begin{equation}\label{eq:xx5}
  p(y)\approx e^{\phi(y_1,\hat\alpha_1(y_1))} \kern-.5em 
  \int_{-\infty}^\infty \kern-.25em \exp\bigg\{- 
  \Big(-D^2_{\alpha_1\alpha_1}\phi 
      \big(y_1,\hat\alpha_1(y_1) \big)\,
      \big (\alpha_1-\hat\alpha_1(y),  \alpha_1-\hat\alpha_1(y) \big) \,
   \Big) \bigg\}\,\textrm{d}\alpha_1
\end{equation}
Making a change of variables and integrating, we obtain
\begin{equation}\label{eq:xx4}
  p(y_1)\approx e^{\phi (y_1,\hat\alpha_1(y_1) )}  (2\pi)^{n/2} 
  \Big | -D^2_{\alpha_1\alpha_1} \phi \big(y_1,\hat\alpha_1(y_1) \big) \Big |^{-1/2}
\end{equation}
\X{Laplace approximation}
\XX{Laplace approximation}{in Kalman filter}
\X{robust Kalman filter}
This is the Laplace approximation to the integral in equation~(\ref{eq:xx6}).

If the distribution of $\alpha_1$ is (multivariate) normal and
the distribution of $y_1|\alpha_1$ is multivariate normal,
then $\phi(y_1,\alpha_1)$ is a quadratic function of $\alpha_1$, so
the Laplace approximation is exact. The advantage of the
Laplace approximation is that it can be employed for non-normal distributions.

To illustrate this advantage, consider the simple 1-dimensional case where
$\alpha_1$ has a (univariate) normal distribution with mean~0 and
variance~$\sigma_\alpha^2$.
Assume that the distribution of $y_1|\alpha_1$ is a fat-tailed
distribution, which is a mixture of $0.95$  normal distribution and $0.05$ Cauchy
distribution. Then,
\begin{align}
\nonumber \phi(y_1,\alpha_1)=& \log \bigg[ 0.95\exp \Big(\kern-.1em -0.5 (y_1-\alpha_1)^2  \Big/  \sigma_y^2 \Big)
     + 0.05\sqrt{2/\pi} \Big/ \Big(1+ (y-\alpha_1)^2 \big/ \sigma_y^2 \Big) \bigg] \\
   & -0.5\alpha_1^2 \big/ \sigma_\alpha^2 + \textrm{const}
   \label{eq:xx9.7}
\end{align}
whereas if $y_1$ is assumed to have a normal distribution,
\begin{equation}
  \phi(y_1,\alpha_1)=-0.5(y_1-\alpha_1)^2 \big/  \sigma_y^2
    -0.5\alpha_1^2 \big/ \sigma_\alpha^2\ + \textrm{const}
    \label{eq:xx9.8}
\end{equation}
where ``$\textrm{const}$'' denotes some constant independent of~$\alpha_1$.
There are two drawbacks to the use of equation~(\ref{eq:xx9.8}). %fixed $5.b$
If the value of $y_1$ is an outlier from the point of the normal
model, then it will have too much influence on the mode of the estimate of
$p(\alpha_1 | y_1)$.  Also, since the variance %xx added a { to middle term:
\begin{equation}
  \sigma_{\alpha_1|y_1}^2=\big\{D^2_{\alpha_1\alpha_1}\phi(y_1,\beta_i)\big\}^{-1}=\Big[1/\sigma_y^2 + 1/\sigma_\alpha^2\Big]^{-1}
\end{equation}
is independent of the value of $y_1$ observed, 
$\sigma_{\alpha_1|y_1}^2$ will be underestimated.  This is incorrect behavior,
since if $y_1$ is an outlier, it contains (almost) no information
about the value of $p(\alpha_1|y_1)$.  So, 
$p(\alpha_1|y_1)$ should be almost equal to~$p(\alpha_1)$.
The likelihood function based on equation~(\ref{eq:xx9.7}) %fixed $5a$
has the desired behavior.

To calculate expression~(\ref{eq:xx4}), it is necessary to maximize $\phi(y_1,\alpha_1)$ with respect to $\alpha_1$,
and to calculate its Hessian matrix with respect to $\alpha_1$.

For the maximization, we employ the Newton-Raphson algorithm. 
Let $\beta_0=\mu_{\alpha_1}$
\begin{equation}
 \beta_{i+1}=\beta_i-\big\{D^2_{\alpha_1\alpha_1}\phi(y_1,\beta_i)\big\}^{-1}
  \big(D_{\alpha_1}\phi(y_1,\beta_i) \big)
\end{equation}
This operation is carried out a fixed number, $r$, times and then
$\hat\alpha_1(y_1)\approx\beta_r$.
For ``well behaved'' problems, the sequence $\beta_i$ converges
quadratically to $\hat\alpha_1(y_1)$.
We approximate $p(\alpha_1|y_1)$ by a multivariate normal with
\begin{align*}
  \mu_{\alpha_1|y_1}&=\beta_r\\[6pt]
  \sigma^2_{\alpha_1|y_1}&=
  \big\{-D^2_{\alpha_1\alpha_1}\phi(y_1,\beta_r)\big\}^{-1}\\
\end{align*}
and approximate $p(\alpha_2|y_1)$ by a multivariate normal with
\begin{align*}
  \mu_{\alpha_2|y_1}&=T(\beta_r,y_1)\cr 
  \sigma^2_{\alpha_2|y_1}&= 
  D_{\alpha_1}\, T_1(\beta_r,y_1)\sigma^2_{\alpha_1|y_1}
  D_{\alpha_1}\, T_1(\beta_r,y_1) ^{\prime}+
  \sigma^2_\eta\cr
\end{align*}
Now,
\begin{equation}\label{eq:xx3}
  p(y_2|y_1)=\int_{-\infty}^\infty p(y_2|\alpha_2)\, p(\alpha_2|y_1)
  \,\textrm{d}\alpha_2
\end{equation}
As above, we maximize the integrand of equation~(\ref{eq:xx3}) with respect to
$\alpha_2$ and use the Laplace approximation to the integral.
This produces the sequence of conditional probabilities $p(y_i|y_{i-1})$.
The log-likelihood function for the observed sequence ${y_i}$
is given by 
\begin{equation}
  \sum_{i=1}^n \log\Big(p(y_i|y_{i-1})\Big)
\end{equation}


\section{Parameter estimation}

Although we have not explicitly shown them, the conditional likelihood
functions $p(y_i|y_{i-1})$  depend on a number of 
parameters. These parameters include the specification of $T$, other
parameters in the probability density $p(y_i|\alpha_i)$, and parameters
that determine $\sigma^2_\eta$. If we denote these parameters by
$\theta$ and write $\big(p(y_i|y_{i-1},\theta)\big)$ to indicate this
dependence, the log-likelihood function becomes
\begin{equation}\label{eq:xx1}
  \sum_{i=1}^n \log\Big(p(y_i|y_{i-1},\theta)\Big)
\end{equation}
The maximum likelihood estimates for the parameter vector $\theta$ are
found by maximizing expression~(\ref{eq:xx1}) with respect to~$\theta$.


\X{stochastic volatility model}
\section{The stochastic volatility model}

The version of the stochastic volatility model presented here is from
\cite{RePEc:eee:empfin:v:5:y:1998:i:2:p:155-173}.

It is assumed that $y_i$ has a multivariate normal distribution with
$\mu_{y_i}=0$ and covariance matrix 
$\Omega_i(\alpha_i)=H_i(\alpha_i)RH_i(\alpha_i)$.
$H_i(\alpha_i)$ is an $m\times m$ diagonal matrix whose $j^\textrm{th}$ element
on the diagonal is given by $\exp(\alpha_{ij})/2$, where the
$\alpha_{ij}$ satisfy the relationship
\begin{equation}
  \alpha_i=w+\ep(\delta,\alpha_{i-1}) +\ep(\lambda_1,y_{i-1}) 
   + \ep(\lambda_2,|y_{i-1}|) + \eta_i
\end{equation}
where, in turn, $\eta_i$ is a multivariate normal random variable with
$\mu_{\eta_i}=0$ and $\sigma^2_{\eta_i}=\sigma^2_\eta$.
If $u$ and $v$ are two vectors with $j^\textrm{th}$ component $u_j$ and $v_j$,
$\ep(i,v)$ is the vector with $j^\textrm{th}$ component $u_jv_j$.
$R$ is an  $m\times m$ positive definite matrix satisfying $r_{jj}=1$,
that is, a correlation matrix.
Then,
\begin{equation}
  \log\big(p(y_i | \alpha_i)\big)=
     -0.5\log|\Omega_i(\alpha_i)|-0.5y^\prime_i\,\,\Omega_i(\alpha_i)^{-1}y_i
\end{equation}
and the distribution of $\alpha_i|y_{i-1}$ is multivariate normal, with
mean vector and covariance matrix given by
\begin{align}
  \mu_{\alpha_i|y_{i-1}}&=w+\ep(\delta,\mu_{\alpha_{i-1}|y_{i-1}})
     +\ep(\lambda,y_{i-1})\\[4pt]
  \sigma^2_{\alpha_i|y_{i-1}}&=i\,\diag(\delta)
   \sigma^2_{\alpha_{i-1}|y_{i-1}}\diag(\delta) 
  +\sigma^2_\eta 
\end{align}
$\diag(\delta)$ is the diagonal matrix whose diagonal is equal to the vector
$\delta$. 
\begin{align} \label{eq:xx2}
  \log\big(p(y_i|\alpha_i)\,p(\alpha_i|y_{i-1})\big)&=
     -0.5\log|\Omega_i(\alpha_i)|-0.5y_i^\prime\Omega_i(\alpha_i)^{-1}y_i
     -0.5\log|\sigma^2_{\alpha_i|y_{i-1}}|\\[6pt]
     &\quad   -0.5(\alpha_i-\mu_{\alpha_i|y_{i-1}})^\prime
          (\sigma^2_{\alpha_i|y_{i-1}})^{-1}
             (\alpha_i-\mu_{\alpha_i|y_{i-1}}) 
\end{align}
To perform the Newton-Raphson calculations, it is necessary to calculate
the first and second derivatives of expression~(\ref{eq:xx2}) with respect
to the parameter vector~$\alpha$. This is the most involved part of the
calculations and will depend on the particular form of the model. In the present
case, the calculations are simplified by the fact that $\Omega_i$ only depends on
$\alpha$ through the diagonal matrix $H(\alpha_i)$.

The probability density function $p(\alpha_1)$ is 
assumed to be multivariate normal with $\mu_{\alpha_1}=\theta_0$
and $\sigma^2_{\alpha_1}=0$.


\section{The data}

The data consist of the daily Mark/Dollar and Yen/dollar exchange rates
and the U.S. and Japanese stock index data. There are 1301
time periods, with some missing data. The missing data, which are denoted by the
impossibly large value of 10,000, were replaced with
the average from the period before and after. They can, however, easily
be estimated in the model, if desired.


\section{The results}

The model was fit with various combinations of the parameters,
and the log-likelihood was examined to investigate the improvement in
fit due to the addition of the parameters.  See Table~\ref{tab:parameters}.
\begin{table}[h]
\begin{center}
\begin{tabular}{@{\vrule height 16pt depth 6pt width0pt}@{\extracolsep{1em}}c c c}
\\
\hline 
\bf Parameters in model & \bf Number of parameters & \bf Log-likelihood \\[3pt]
\hline 
%\noalign{\medskip}
 $w,\delta,R,\sigma^2_\eta$ & 24 & 3774.7 \\
 $w,\delta,R,\sigma^2_\eta,\lambda_1$ & 28 & 3806.6 \\
 $w,\delta,R,\sigma^2_\eta,\lambda_1,\theta_0$ & 32 & 3808.6 \\
 $w,\delta,R,\sigma^2_\eta,\lambda_1,\theta_0,\lambda_2$ & 36 & 3811.2 \\[6pt]
\hline
\\
\end{tabular}
\end{center}
\emptycaption
\label{tab:parameters}
\end{table}


The parameters $\theta_0$ and $\lambda_2$ did not produce a significant
improvement to the fit.  $\lambda_2$ measures the asymmetry 
in the response of the variance to positive and negative shocks.

Here are the parameter estimates and their standard deviations for
the model with $w,\delta,R,\sigma^2_\eta$, and $\lambda_1$:
\begin{lstlisting}
 index   name    value      std.dev   
    1   w(1)       -1.3749e-001 4.9434e-002
    2   w(2)       -6.5649e-001 1.6161e-001
    3   w(3)        3.1693e-002 1.0574e-002
    4   w(4)       -1.2973e-002 1.5375e-002
    5   lambda1(1)  1.5564e-001 4.9688e-002
    6   lambda1(2)  1.8647e-001 6.9525e-002
    7   lambda1(3)  -6.9265e-002 1.4158e-002
    8   lambda1(4)  -1.6689e-001 3.1626e-002
    9   delta(1)    8.2229e-001 4.6074e-002
   10   delta(1)    5.0848e-001 1.0785e-001
   11   delta(1)    9.5763e-001 1.4602e-002
   12   delta(1)    9.3610e-001 1.8812e-002
   29   R(1,1)      1.0000e+000 0.0000e+000
   30   R(1,2)      5.3821e-001 2.2883e-002
   31   R(1,3)      -7.1704e-002 2.9477e-002
   32   R(1,4)      -3.8796e-002 2.9278e-002
   33   R(2,1)      5.3821e-001 2.2883e-002
   34   R(2,2)      1.0000e+000  0.0000e+000
   35   R(2,3)      -1.2932e-001 2.9111e-002
   36   R(2,3)      -4.1466e-002 2.9468e-002
   37   R(3,1)      -7.1704e-002 2.9477e-002
   38   R(3,2)      -1.2932e-001 2.9111e-002
   39   R(3,3)      1.0000e+000  0.0000e+000
   40   R(1,4)      8.8811e-002 2.9085e-002
   41   R(4,1)      -3.8796e-002 2.9278e-002
   42   R(4,2)      -4.1466e-002 2.9468e-002
   43   R(4,3)      8.8811e-002 2.9085e-002
   44   R(4,4)      1.0000e+000  0.0000e+000
   45   Omega(1,1)  6.5973e-001 6.3099e-002
   46   Omega(1,2)  1.9827e-001 1.6129e-002
   47   Omega(1,3)  -1.3395e-001 5.4982e-002
   48   Omega(1,4)  -3.5161e-002 2.6676e-002
   49   Omega(2,1)  1.9827e-001 1.6129e-002
   50   Omega(2,2)  2.0570e-001 2.3994e-002
   51   Omega(2,3)  -1.3489e-001 3.2608e-002
   52   Omega(2,4)  -2.0985e-002 1.5016e-002
   53   Omega(3,1)  -1.3395e-001 5.4982e-002
   54   Omega(3,2)  -1.3489e-001 3.2608e-002
   55   Omega(3,3)  5.2895e+000 5.7872e-001
   56   Omega(3,4)  2.2791e-001 7.9318e-002
   57   Omega(4,1)  -3.5161e-002 2.6676e-002
   58   Omega(4,2)  -2.0985e-002 1.5016e-002
   59   Omega(4,3)  2.2791e-001 7.9318e-002
   60   Omega(4,4)  1.2451e+000 1.7043e-001
   61   Z(1,1)      2.3967e-001 7.4268e-002
   62   Z(1,2)      2.0711e-001 5.5599e-002
   63   Z(1,3)      3.8832e-002 1.8505e-002
   64   Z(1,4)      2.4097e-002 2.0344e-002
   65   Z(2,1)      2.0711e-001 5.5599e-002
   66   Z(2,2)      4.6309e-001 1.1143e-001
   67   Z(2,3)      3.4298e-002 2.3017e-002
   68   Z(2,4)      9.6831e-003 2.9999e-002
   69   Z(3,1)      3.8832e-002 1.8505e-002
   70   Z(3,2)      3.4298e-002 2.3017e-002
   71   Z(3,3)      3.9101e-002 1.6885e-002
   72   Z(3,4)      2.4602e-002 1.1053e-002
   73   Z(4,1)      2.4097e-002 2.0344e-002
   74   Z(4,2)      9.6831e-003 2.9999e-002
   75   Z(4,3)      2.4602e-002 1.1053e-002
   76   Z(4,4)      9.6109e-002 3.4268e-002
\end{lstlisting}

The AD Model Builder \textsc{tpl} file for the model is given below:
\begin{lstlisting}
DATA_SECTION
  init_int ndim
  init_int nobs
  int ndim1
  int ndim2
 !! ndim1=ndim*(ndim+1)/2;
 !! ndim2=ndim*(ndim-1)/2;
  init_matrix Y(1,nobs,1,ndim)
 LOC_CALCS
  // replace missing values (10000) with the average of before and after.
  for (int i=2;i<nobs;i++)
    for (int j=1;j<=ndim;j++)
      if (Y(i,j)==10000)
      {
        int i2=i+1;
        do
        {
          if (Y(i2,j)==10000) 
            i2++;
          else
            break; 
        } 
        while(1);
        Y(i,j)=(Y(i-1,j)+Y(i2,j))/2.;
        if (Y(i,j)>100.0)     // did this work
          cerr << " Y(i,j) too big " << Y(i,j) << endl; 
      }      
 END_CALCS
 
PARAMETER_SECTION
  matrix h_mean(1,nobs,1,ndim)
  3darray h_var(1,nobs,1,ndim,1,ndim)
  number ldR;
  init_vector theta0(1,ndim,3);
  vector lmin(1,nobs)
  init_bounded_vector w(1,ndim,-10,10)
  vector w1(1,ndim)
  init_vector lambda(1,ndim,2)
  init_vector lambda2(1,ndim,-1)
  init_bounded_vector delta(1,ndim,0,.98)
  sdreport_matrix R(1,ndim,1,ndim)
  sdreport_matrix Omega(1,ndim,1,ndim)
  matrix ch_R(1,ndim,1,ndim)
  matrix Rinv(1,ndim,1,ndim)
  init_bounded_vector v_R(1,ndim2,-1.0,1.0)
  sdreport_matrix Z(1,ndim,1,ndim)
  matrix ch_Z(1,ndim,1,ndim)
  init_bounded_vector v_Z(1,ndim1,-1.0,1.0)
  matrix S(1,ndim,1,ndim);
  objective_function_value f
INITIALIZATION_SECTION
  delta 0.9
PROCEDURE_SECTION

  fill_the_matrices();
  int sgn;
  ldR=ln_det(R,sgn);
  Rinv=inv(R);
  dvar_vector tmp(1,ndim);
  dvar_matrix sh(1,ndim,1,ndim);
  h_mean(1)=theta0;
  h_var(1)=0; 
  for (int i=2;i<=nobs;i++)
  {
    dvar_vector tmean=update_the_means(w,h_mean(i-1),Y(i-1));
    dvar_matrix v=update_the_variances(h_var(i-1));
    tmp=tmean;
    dvar_vector h(1,ndim);
    dvar_vector gr(1,ndim);
    for (int ii=1;ii<=4;ii++)  // do the Newton-Raphson 4 times
    {
      xfp12(tmp, Y(i),tmean,v,gr,sh); // get 1st and 2nd derivatives
      h=-solve(sh,gr);  //sh is hessian and gr is the gradient
      tmp+=h;  // add new step h
    }
    double nh=norm2(value(h)); // check size of h for convergence
    if (nh>1.e-1) 
      cout << "No convergence in NR " << nh << endl;
    if (nh>1.e+02) 
    {
      f+=1.e+7;   // this ensures that the function minimizer will take a
      return;    // smaller step
    }
    h_mean(i)=tmp;
    h_var(i)=inv(sh);
    lmin(i)=fp(tmp,Y(i),tmean,v);
    int sgn;
    f+=lmin(i)+0.5*ln_det(sh,sgn);  // Laplace approximation
  }
  f-=0.5*nobs*ndim*log(2.*PI);
  Omega=S;

FUNCTION  dvar_vector update_the_means(dvar_vector& w,dvar_vector& m,dvector& e)
  dvar_vector tmp= w+elem_prod(delta,m)+elem_prod(lambda,e);
  if (active(lambda2))
    tmp+=elem_prod(lambda2,fabs(e)); 
  return tmp;
  
FUNCTION  dvar_matrix update_the_variances(dvar_matrix& v)
  dvar_matrix tmp(1,ndim,1,ndim);
  for (int i=1;i<=ndim;i++)
  {
    for (int j=1;j<=i;j++)
    {
      tmp(i,j)=delta(i)*delta(j)*v(i,j);
      if (i!=j) tmp(j,i)=tmp(i,j);
    }
  }
  tmp+=Z;
  return tmp;
  
FUNCTION dvariable fp(dvar_vector& h, dvector& y, dvar_vector& m,dvar_matrix& v)
  dvar_vector eh=exp(.5*h);
  for (int i=1;i<=ndim;i++)
  {
    for (int j=1;j<=i;j++)
    {
      S(i,j)= eh(i)*eh(j)*R(i,j);
      if (i!=j) S(j,i)=S(i,j);
    }
  }   

  dvariable lndet;
  dvariable sgn;
  dvar_vector u=solve(S,y,lndet,sgn);
  dvariable l;
  l=.5*lndet+.5*(y*u);
  dvar_vector hm=h-m;
  w1=solve(v,hm,lndet,sgn);
  l+=.5*lndet+.5*(w1*hm);
  return l;

FUNCTION void xfp12(dvar_vector& h, dvector& y,dvar_vector& m,dvar_matrix& v,
 dvar_vector gr,dvar_matrix& hess)
  dvar_vector ehinv=exp(-.5*h);
  dvariable lndet;
  dvariable sgn;
  dvar_vector ys=elem_prod(ehinv,y);
  dvar_vector u=Rinv*ys;
  gr=0.5;
  dvar_vector vv=elem_prod(ys,u);
  gr-=.5*vv;
  dvar_vector hm=h-m;
  dvar_vector w=solve(v,hm,lndet,sgn);
  gr+=w;
  for (int i=1;i<=ndim;i++)
  {
    for (int j=1;j<=i;j++)
    {
      hess(i,j)=0.25*ys(i)*ys(j)*Rinv(i,j);
      if (i!=j) hess(j,i)=hess(i,j);
    }
  }
  for (i=1;i<=ndim;i++)
  {
    hess(i,i)+=.25*vv(i);
  }
  hess+=inv(v);

FUNCTION  fill_the_matrices
  int ii=1;
  ch_Z.initialize();
  for (int i=1;i<=ndim;i++)
  {
    for (int j=1;j<=i;j++)
      ch_Z(i,j)=v_Z(ii++);  
    ch_Z(i,i)+=0.5;
  }
  Z=ch_Z*trans(ch_Z);
  ch_R.initialize();
  ii=1;
  for (i=1;i<=ndim;i++)
  {
    for (int j=1;j<i;j++)
      ch_R(i,j)=v_R(ii++);  
    ch_R(i,i)+=0.1;
    ch_R(i)/=norm(ch_R(i));
  }
  R=ch_R*trans(ch_R);

REPORT_SECTION
  report<<"observed"<<Y<<endl;
  for (int i=1;i<=nobs;i++)
  {
    report<< "mean" <<endl;
    report<< h_mean(i) <<endl;
    report<< "covariance" <<endl;
    report<<h_var(i)<<endl;
    report<<endl;
  }
  report<< "S(nobs) " << endl;
  report<< Omega << endl;
  report<< "Z " << endl;
  report<< Z << endl;
  report<< "R " << endl;
  report<< R << endl;

TOP_OF_MAIN_SECTION
  arrmblsize=20000000;
  gradient_structure::set_CMPDIF_BUFFER_SIZE(25000000);
  gradient_structure::set_GRADSTACK_BUFFER_SIZE(1000000);
\end{lstlisting}

