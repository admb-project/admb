% $Id$
%
% Author: David Fournier
% Copyright (c) 2008 Regents of the University of California
%

This chapter attempts to list and document all the functions
available in \ADM. It will always be incomplete, since functions are
continually being added. If you are aware of a function that is
not documented please, contact me at \href{mailto:otter@otter-rsch.com}{otter@otter-rsch.com} and
let me know.


\section{Naming conventions for documenting functions}

Wherever applicable, the \texttt{name} function has been supplied for
constant and variable objects (such as \texttt{double} and \texttt{dvariable}).
Instead of repeating the description for both kinds of objects,
the convention of referring to both types as ``number,'' ``vector,''
``matrix,'' etc., will be observed.


\section{Mathematical functions}

 The following functions have been included in \scAD, by overloading the 
\cplus\ library functions or adding additional functions where necessary:
\begin{lstlisting}
  acos atan  cos cosh cube exp (mfexp) fabs gammln (sfabs) log log_comb 
  log10 log_density_poisson pow square sqrt sin sinh tan tanh
\end{lstlisting}
\X{\fontindexentry{tt}{sin}} \X{\fontindexentry{tt}{cos}} \X{\fontindexentry{tt}{tan}} \X{\fontindexentry{tt}{asin}} \X{\fontindexentry{tt}{atan}} \X{\fontindexentry{tt}{acos}} \X{\fontindexentry{tt}{sinh}} 
\X{\fontindexentry{tt}{cosh}} \X{\fontindexentry{tt}{tanh}} \X{\fontindexentry{tt}{fabs}} \X{\fontindexentry{tt}{sfabs}} \X{\fontindexentry{tt}{exp}} \X{\fontindexentry{tt}{log}} 
\X{\fontindexentry{tt}{log10}} \X{\fontindexentry{tt}{sqrt}} \X{\fontindexentry{tt}{pow}} 
\X{\fontindexentry{tt}{gammln}}\X{\fontindexentry{tt}{log\_comb}}
\XX{functions}{\fontindexentry{tt}{sin}} \XX{functions}{\fontindexentry{tt}{cos}} \XX{functions}{\fontindexentry{tt}{tan}} 
\XX{functions}{\fontindexentry{tt}{asin}} \XX{functions}{\fontindexentry{tt}{atan}} \XX{functions}{\fontindexentry{tt}{acos}} 
\XX{functions}{\fontindexentry{tt}{sinh}} \XX{functions}{\fontindexentry{tt}{cosh}} \XX{functions}{\fontindexentry{tt}{tanh}} 
\XX{functions}{\fontindexentry{tt}{fabs}} \XX{functions}{\fontindexentry{tt}{sfabs}} \XX{functions}{\fontindexentry{tt}{exp}} 
\XX{functions}{\fontindexentry{tt}{log}} \XX{functions}{\fontindexentry{tt}{log10}} \XX{functions}{\fontindexentry{tt}{sqrt}} 
\XX{functions}{\fontindexentry{tt}{pow}} 
\XX{functions}{\fontindexentry{tt}{gammln}}\XX{functions}{\fontindexentry{tt}{log\_comb}}
These functions can be used on \texttt{number}s or \texttt{vector\_object}s
in the form
\begin{lstlisting}
number = function(number);
vector_object = function(vector_object);
\end{lstlisting}
When operating on \texttt{vector\_object}s, the functions operate  
elemen-by-element, so if \texttt{y} is a \texttt{dvector} whose
elements are $(y_1,\ldots,y_n)$, then \texttt{exp(y)} is a 
\texttt{dvector} whose elements are $(\exp(y_1),\ldots,\exp(y_n))$.

\XX{vector}{maximum element}\XX{\fontindexentry{tt}{max}}{operation on a vector}
\XX{vector}{minimum element}\XX{\fontindexentry{tt}{min}}{operation on a vector}
The functions \texttt{min} and \texttt{max}, when applied to a \texttt{vector\_object},
return a \texttt{number} that is equal to the minimum or maximum element of the
\texttt{vector\_object}.

The function \texttt{gammln} is the logarithm of the gamma function.
\XX{gamma function}{logarithm}

The function \texttt{log\_comb(n,k)} is the logarithm of the  function,
the combination of \texttt{n} things taken \texttt{k} at a time.
It is defined via the logarithm of the gamma function 
for non-integer values,
and is differentiable.

\X{\fontindexentry{tt}{log\_comb} function}


\section{Operations on arrays}

\subsection{Element-wise operations}

There are several operations familiar to users of spreadsheets that do
not appear as often in classical mathematical
calculations. For example, spreadsheet users often
wish to multiply one column in a spreadsheet by the corresponding
elements of another column. Spreadsheet users might find it much more
natural to define the product of matrices as an element-wise operation,
such as
$$z_{ij}= x_{ij}*y_{ij}$$
The ``classical'' mathematical definition for the
matrix product has been assigned to the overloaded operator `\texttt{*},'
so large mathematical formulas involving
vector and matrix operations can be written in a concise notation.
Typically, spreadsheet-type calculations are not so complicated
and do not suffer so much from being forced to adopt a
``function style'' of notation.

Since addition and subtraction are already defined in an element-wise manner,
it is only necessary to define element-wise operations 
for multiplication and
division. We have named these functions \texttt{elem\_prod} and \texttt{elem\_div}.

\XX{functions}{element-wise product of vectors}
\XX{vector}{element-wise product}\XX{\fontindexentry{tt}{elem\_prod}}{element-wise product}
\begin{lstlisting}
vector_object = elem_prod(vector_object,vector_object) // element-wise multiply 
\end{lstlisting}
\afterlistingthing{$z_i= x_i*y_i $}

\XX{vector}{element-wise division}\XX{\fontindexentry{tt}{elem\_div}}{element-wise division}
\XX{functions}{element-wise division of vectors}
\begin{lstlisting}
vector_object = elem_div(vector_object,vector_object) // element-wise divide
\end{lstlisting}
\afterlistingthing{$z_i= x_i/y_i $}

\XX{functions}{element-wise product of matrices}
\XX{matrix}{element-wise product}
\begin{lstlisting}
matrix_object = elem_prod(matrix_object,matrix_object) // element-wise multiply 
\end{lstlisting}
\afterlistingthing{$z_{ij}= x_{ij}*y_{ij} $}

\XX{matrix}{element-wise division}
\XX{functions}{element-wise division of matrices}
\begin{lstlisting}
matrix_object = elem_div(matrix_object,matrix_object) // element-wise divide
\end{lstlisting}
\afterlistingthing{$z_{ij}= x_{ij}/y_{ij} $}


\section{The identity matrix function \texttt{identity\_matrix}}
\XX{functions}{identity matrix function}
\XX{matrix}{identity matrix function}

\begin{lstlisting}
matrix_object = identity_matrix(int min,int max)
\end{lstlisting}
\afterlistingthing{creates a square identity matrix with minimum valid index \texttt{min} and 
maximum valid index \texttt{max}.}


\section{Probability densities and related functions:\BR \texttt{poisson negative binomial cauchy}}
\XX{functions}{Cauchy density}
\XX{probability densities}{Cauchy}

\begin{lstlisting}
number log_density_cauchy(number x);
\end{lstlisting}
\afterlistingthing{returns the logarithm of the Cauchy density function 
at \texttt{x}.}

\XX{functions}{Poisson density}
\XX{probability densities}{Poisson}
\begin{lstlisting}
number log_density_poisson(number x,number mu);
\end{lstlisting}
\afterlistingthing{returns the logarithm of the Poisson density function 
at \texttt{x}
with mean \texttt{mu}.}

\XX{functions}{negative binomial density}
\XX{probability densities}{negative binomial density}
\begin{lstlisting}
number log_negbinomial_density(number x,number mu,number tau);
\end{lstlisting}
\afterlistingthing{returns the logarithm of the negative  binomial density function 
with mean \texttt{mu} and over-dispersion (variance/mean) \texttt{tau}.
\texttt{tau} must be greater than~1.}


\section{The operations \texttt{det inv norm norm2 min max sum}}


\subsubsection{The determinant of a matrix object} 
\XX{matrix}{determinant}\X{determinant}
\XX{functions}{determinant of a matrix}
\X{\fontindexentry{tt}{det}}

(The matrix must be square, that is, the number of rows must equal the number of columns.)
\begin{lstlisting}
matrix_object = det(matrix_object)
\end{lstlisting}


\subsubsection{The inverse of a matrix object} 
\XX{matrix}{inverse}\X{inverse} \X{\fontindexentry{tt}{inv}}
\XX{functions}{inverse of a matrix}

(The matrix must be square, that is, the number of rows must equal the number of columns.)
\begin{lstlisting}
matrix_object = inv(matrix_object)
\end{lstlisting}


\subsubsection{The norm of a vector object}
\XX{functions}{norm of a vector}
\XX{vector}{\fontindexentry{tt}{norm}}

\begin{lstlisting}
number = norm(vector_object)
\end{lstlisting}
\afterlistingthing{$z=\sqrt{\sum_{i} x_{i}^2} $}
\bigskip %zzz


\subsubsection{The norm squared of a vector object}
\XX{vector}{norm squared}\X{\fontindexentry{tt}{norm2}}
\XX{matrix}{norm squared}\X{\fontindexentry{tt}{norm2}}
\XX{functions}{norm squared of a vector}

\begin{lstlisting}
number = norm2(vector_object)
\end{lstlisting}
\afterlistingthing{$z=\sum_{i} x_{i}^2 $}
\bigskip 


\subsubsection{The norm of a matrix object}
\XX{matrix}{\fontindexentry{tt}{norm}}\X{\fontindexentry{tt}{norm}}
\XX{functions}{norm of a matrix}

\begin{lstlisting}
number = norm(matrix_object)
\end{lstlisting}
\afterlistingthing{$z=\sqrt{\sum_{ij} x_{ij}^2} $}
\bigskip 


\subsubsection{The norm squared of a matrix object}
\XX{functions}{norm squared of a matrix}
\XX{matrix}{norm squared}\X{\fontindexentry{tt}{norm2}}

\begin{lstlisting}
number = norm2(matrix_object)
\end{lstlisting}
\afterlistingthing{$z_{ij}=x_{ji} $}
\bigskip 


\subsubsection{The transpose of a matrix object}
\XX{functions}{transpose of a matrix}
\XX{matrix}{transpose}

\begin{lstlisting}
matrix_object = trans(matrix_object)
\end{lstlisting}
\afterlistingthing{$z=\sum_{ij} x_{ij}^2 $}
\bigskip 


\subsubsection{The sum over the elements of a vector object}
\XX{functions}{sum over the elements of a vector}
\XX{vector}{sum over the elements}\XX{\fontindexentry{tt}{sum}}{operation on a vector}

\begin{lstlisting}
number = sum(vector_object)
\end{lstlisting}
\afterlistingthing{$z=\sum_i x_i $}
\bigskip 


\subsubsection{The row sums of a matrix object}
\XX{matrix}{\fontindexentry{tt}{rowsum}}\XX{\fontindexentry{tt}{rowsum}}{operation on a matrix}

\begin{lstlisting}
vector = rowsum(matrix_object)
\end{lstlisting}
\afterlistingthing{$z_i=\sum_j x_{ij} $}
\bigskip 


\subsubsection{The column sums of a matrix object}
\XX{matrix}{\fontindexentry{tt}{colsum}}\XX{\fontindexentry{tt}{colsum}}{operation on a matrix}
\XX{functions}{sum over the columns of a matrix}

\begin{lstlisting}
vector = colsum(matrix_object)
\end{lstlisting}
\afterlistingthing{$z_j=\sum_i x_{ij} $}
\bigskip 


\subsubsection{The minimum element of a vector object}
\XX{vector}{minimum element}\XX{\fontindexentry{tt}{min}}{operation on a vector}
\XX{functions}{minimum element of a vector}

\begin{lstlisting}
number = min(vector_object)
\end{lstlisting}
\bigskip 


\subsubsection{The maximum element of a vector object}
\XX{vector}{maximum element}\XX{\fontindexentry{tt}{max}}{operation on a vector}
\XX{functions}{maximum element of a vector}

\begin{lstlisting}
number = max(vector_object)
\end{lstlisting}
\bigskip 



\section{Eigenvalues and eigenvectors of a symmetric matrix}

\XX{eigenvalues}{not differentiable}
\XX{eigenvectors}{not differentiable}
\XX{functions}{eigenvalues of symmetric matrix}
\XX{functions}{eigenvectors of symmetric matrix}

While we have included eigenvalue and eigenvector routines for
both constant and variable matrix objects, you should be aware
that, in general, the eigenvectors and eigenvalues are not differentiable
functions of the variables determining the matrix.  

\XX{eigenvectors} {of a symmetric matrix}
\begin{lstlisting}
matrix_object = eigenvectors(matrix_object)
\end{lstlisting}
\afterlistingthing{returns in a matrix the eigenvectors of a symmetric matrix.}

 It is the user's responsibility to 
ensure that the matrix is actually symmetric. The routine symmetrizes 
the matrix, so the eigenvectors returned are actually those for
the symmetrized matrix. The eigenvectors are located in the
columns of the matrix. The $i^\textrm{th}$ eigenvalue returned by the
function \texttt{eigenvalues} corresponds to the $i^\textrm{th}$ eigenvector  
returned by the function \texttt{eigenvectors}.


\section{The Choleski decomposition of a\br positive definite symmetric matrix}
\X{Choleski decomposition of a symmetric matrix}
\XX{functions}{Choleski decomposition}
\XX{symmetric matrix}{Choleski decomposition}

For a positive definite symmetric matrix \texttt{S}, the
Choleski decomposition of \texttt{S} is a lower triangular matrix \texttt{T}
satisfying the relationship \texttt{S=T*trans(T)}. 
If \texttt{S} is a (positive definite symmetric) matrix object and
\texttt{T} is a matrix object, the line of code
\begin{lstlisting}
  T=choleski_decomp(S);
\end{lstlisting}
will calculate the Choleski decomposition of \texttt{S} and put it
into \texttt{T}.
\X{solving a system of linear equations}
\X{solve@\texttt{solve} function}
\XX{functions}{solving linear system of equations}
\XX{optimizing performance}{using the best operators for a calculation}


\section{Solving a system of linear equations}

If \texttt{y} is a vector and \texttt{M} is an invertible matrix, then finding
a vector \texttt{x} such that
\begin{lstlisting}
  x=inv(M)*y
\end{lstlisting}
will be referred to as ``solving the system of linear equations
determined by \texttt{y} and \texttt{M}.'' Of course, it is possible
to use the \texttt{inv} function to accomplish this task, but it is much
more efficient to use the \texttt{solve} function:
\begin{lstlisting}
  vector x=solve(M,y); // x will satisfy x=inv(M)*y;
\end{lstlisting}
It turns out that it is a simple matter to calculate the determinant
of the matrix \texttt{M} while solving the system of linear
equations.  Since this is useful in multivariate
analysis, we have also included a function that returns the 
determinant when the system of equations is solved.
To avoid floating point overflow, or underflow when working with
large matrices, the logarithm of the absolute value of the
determinant, together with the sign of the determinant, are returned.
The constant form of the \texttt{solve} function is
\begin{lstlisting}
double ln_det;
double sign;
dvector x=solve(M,y,ln_det,sign);
\end{lstlisting}
while the variable form is
\begin{lstlisting}
dvariable ln_det;
dvariable sign;
dvar_vector x=solve(M,y,ln_det,sign);
\end{lstlisting}
The \texttt{solve} function is useful for calculating the log-likelihood
function for a multivariate normal distribution.
Such a log-likelihood function involves a calculation similar to
\begin{lstlisting}
  l = -.5*log(det(S)) -.5*y*inv(S)*y
\end{lstlisting}
where \texttt{S} is a matrix object and \texttt{y} is a vector object.
It is much more efficient to carry out this calculation using
the \texttt{solve} function. The following code illustrates the
calculations for variable objects:
\XX{multivariate normal distribution}{calculation of 
  the log-likelihood function for}
\begin{lstlisting}
dvariable ln_det;
dvariable sign;
dvariable l;
dvar_vector tmp=solve(M,y,ln_det,sign);
l=-.5*ln_det-y*tmp;
\end{lstlisting}


\section{Methods for filling arrays and matrices}

While it is always possible to fill vectors and matrices by
using loops and filling them element by element, this is tedious
and prone to error. To simplify this task, a selection 
of methods for filling vectors and
matrices with either random numbers, or a specified sequence of numbers, is available.
There are also methods for filling row and columns of matrices with
vectors. In this section, the symbol \texttt{vector} can refer to either
a \texttt{dvector} or a \texttt{dvar\_vector},
while the symbol \texttt{matrix} can refer to either
a \texttt{dmatrix} or a \texttt{dvar\_matrix}.

\XX{functions}{filling arrays and matrices}
\X{filling arrays and matrices}
\XX{vector}{\fontindexentry{tt}{fill}}\XX{\fontindexentry{tt}{fill}}{filling a vector}
\begin{lstlisting}
void vector::fill("{m,n,...,}")
\end{lstlisting}
\afterlistingthing{fills a vector with a sequence of the form
\texttt{n, m, $\ldots$} The number of elements in the string must match the
size of the vector.}

\XX{vector}{\fontindexentry{tt}{fill\_seqadd}} \XX{\fontindexentry{tt}{fill\_seqadd}}{filling a vector} 
\begin{lstlisting}
void vector::fill_seqadd(double& base, double& offset)
\end{lstlisting}
\afterlistingthing{fills a vector with a sequence of the form
\texttt{base, base+offset, base+2*offset,$\ldots$}}

For example, if \texttt{v} is a \texttt{dvector} created by the statement
\begin{lstlisting}
  dvector v(0,4);
\end{lstlisting}
then the statement
\begin{lstlisting}
  v.fill_seqadd(-1,.5);
\end{lstlisting}
will fill \texttt{v} with the numbers $(-1.0,-0.5,0.0,0.5,1.0)$.

\XX{matrix}{\fontindexentry{tt}{rowfill\_seqadd}} \XX{\fontindexentry{tt}{rowfill\_seqadd}}{filling a matrix} 
\begin{lstlisting}
void matrix::rowfill_seqadd(int& i,double& base, double& offset)
\end{lstlisting}
\afterlistingthing{fills row \texttt{i} of a matrix with a sequence of the form
\texttt{base, base+offset, base+2*offset,$\ldots$}}

\XX{matrix}{\fontindexentry{tt}{colfill\_seqadd}} \XX{\fontindexentry{tt}{colfill\_seqadd}}{filling a matrix} 
\begin{lstlisting}
void matrix::colfill_seqadd(int& j,double& base, double& offset)
\end{lstlisting}
\afterlistingthing{fills column \texttt{j} of a matrix with a sequence of the form
\texttt{base, base+offset, base+2*offset,$\ldots$}}

\XX{matrix}{\fontindexentry{tt}{colfill}} \XX{\fontindexentry{tt}{colfill}}{filling a matrix column with a vector} 
\XX{matrix}{\fontindexentry{tt}{rowfill}}\XX{\fontindexentry{tt}{rowfill}}{filling a matrix row with a vector} 
\begin{lstlisting}
void matrix::colfill(int& j,vector&)
\end{lstlisting}
\afterlistingthing{fills the $j^\textrm{th}$ column of a matrix with a vector.}

\begin{lstlisting}
void matrix::rowfill(int& i,vector&)
\end{lstlisting}
\afterlistingthing{fills the $i^\textrm{th}$ row of a matrix with a vector.}


\section{Methods for filling arrays and matrices\br with random numbers}

This method of filling containers with random numbers is becoming
obsolete. The preferred method is to use the
\texttt{random\_number\_generator} class.
See Section~\ref{rng} for instructions on using this class.
In this section, a uniformly distributed random number is assumed to have
a uniform distribution on~$[0,1]$. A normally distributed random number
is assumed to have mean~$0$ and variance~$1$.
A binomially distributed random number is assumed to have a parameter~$p$,
where $1$ is returned with probability~$p$, and $0$ is returned with
probability~$1-p$.
A multinomially distributed random variable is assumed to have a vector
of parameters $P$, where $i$ is returned with probability $p_i$.
If the components of $P$ do not sum to~$1$, the vector will be normalized
so that the components {\it do} sum to~$1$.

\XX{vector}{\fontindexentry{tt}{fill\_randu}} \XX{\fontindexentry{tt}{fill\_randu}}{filling a vector with random numbers} 
\begin{lstlisting}
void vector::fill_randu(long int& n)
\end{lstlisting}
\afterlistingthing{fills a vector with a sequence of uniformly distributed 
random numbers. The \texttt{long int n} is a seed for the random number
generator. Changing \texttt{n} will produce a different sequence of
random numbers. 
{\it This function is now obsolete. You should use the}
\texttt{random\_number\_generator} \textit{class to generate random numbers.}}

\XX{matrix}{\fontindexentry{tt}{colfill\_randu}} 
\XX{\fontindexentry{tt}{colfill\_randu}}{filling a matrix with random numbers} 
\begin{lstlisting}
void matrix::colfill_randu(int& j,long int& n)
\end{lstlisting}
\afterlistingthing{fills column \texttt{j} of a matrix with a sequence of uniformly 
distributed random numbers
The \texttt{long int n} is a seed for the random number
generator. Changing \texttt{n} will produce a different sequence of
random numbers.}

\XX{matrix}{\fontindexentry{tt}{rowfill\_randu}}
\XX{\fontindexentry{tt}{rowfill\_randu}}{filling a matrix with random numbers} 
\begin{lstlisting}
void matrix::rowfill_randu(int& i,long int& n)
\end{lstlisting}
\afterlistingthing{fills row \texttt{i} of a matrix with a sequence of uniformly 
distributed random numbers.}

\XX{vector}{\fontindexentry{tt}{fill\_randbi}} \XX{\fontindexentry{tt}{fill\_randbi}}{filling a vector with random numbers} 
\begin{lstlisting}
void vector::fill_randbi(long int& n, double& p)
\end{lstlisting}
\afterlistingthing{fills a vector with a sequence of random numbers from
a binomial distribution.}

\XX{vector}{\fontindexentry{tt}{fill\_randn}} \XX{\fontindexentry{tt}{fill\_randn}}{filling a vector with random numbers} 
\begin{lstlisting}
void vector::fill_randn(long int& n)
\end{lstlisting}
\afterlistingthing{fills a vector with a sequence of normally distributed 
random numbers.
{\it This function is now obsolete. You should use the}
\texttt{random\_number\_generator} \textit{class to generate random numbers.}}

\XX{matrix}{\fontindexentry{tt}{rowfill\_randn}} 
\XX{\fontindexentry{tt}{rowfill\_randn}}{filling a matrix with random numbers} 
\begin{lstlisting}
void matrix::colfill_randn(int& j,long int& n)
\end{lstlisting}
\afterlistingthing{fills column \texttt{j} of a matrix with a sequence of normally 
distributed random numbers.}

\begin{lstlisting}
void matrix::rowfill_randn(int& i,long int& n)
\end{lstlisting}
\afterlistingthing{fills row \texttt{i} of a matrix with a sequence of normally 
distributed random numbers.}

\XX{vector}{\fontindexentry{tt}{fill\_multinomial}} \XX{\fontindexentry{tt}{fill\_multinomial}}{filling a vector with random numbers} 
\begin{lstlisting}
void vector::fill_multinomial(long int& n, dvector& p)
\end{lstlisting}
\afterlistingthing{fills a vector with a sequence random numbers from
a multinomial distribution. The parameter~$p$ is a \texttt{dvector} such that
\texttt{p[i]} is the probability of returning~$i$. The elements of \texttt{p}
must sum to~$1$.}


\section{Methods for obtaining shape information\br from containers}

When this code was first written, the maximum dimension of arrays
was about four. At this level, it perhaps make sense to think of
a 1-dimensional array as a vector, a 2-dimensional array as
a matrix, etc. For a matrix, one thinks in terms of rows and columns.
However, with the adoption of ragged container objects up to
eight dimensions (at present), a more generic method of obtaining
shape information of these objects was called for.

If \texttt{v} is a vector object, then
\begin{lstlisting}
int v.indexmin()
int v.indexmax()
\end{lstlisting}
return the minimum and maximum valid indices for \texttt{v}.
If \texttt{m} is a matrix object, then
\begin{lstlisting}
int v.rowmin()
int v.rowmax()
int v.colmin()
int v.colmax()
\end{lstlisting}
return the minimum and maximum valid row and column indices
for \texttt{m}. These functions make sense for a matrix where every
row is a vector with the same minimum and maximum valid
indices.  For a ragged matrix, this is no longer the case, so
the \texttt{rowmin()} and \texttt{rowmax()} functions don't make sense.
To deal with a ragged matrix, one may need to calculate
the minimum and maximum valid indices for each row of the ragged matrix. 
To facilitate this approach, the functions \texttt{indexmin} and 
\texttt{indexmax} have been defined for all container classes.  So,
for example, if \texttt{w} is a 6-dimensional array, then
\begin{lstlisting}
int w.indexmin()
int w.indexmax()
\end{lstlisting}
return the minimum and maximum valid indices for the first
index of \texttt{w}. For a matrix object \texttt{m},
\texttt{m.indexmin()} and \texttt{m.colmin()} are the same
and as long as \texttt{m} is not ragged, 
\begin{lstlisting}
  m(m.indexmin()).indexmin()
\end{lstlisting}
is the same as
\begin{lstlisting}
  m.colmin()
\end{lstlisting}
and 
\begin{lstlisting}
  m(m.indexmin()).indexmax()
\end{lstlisting}
is the same as
\begin{lstlisting}
  m.colmax()
\end{lstlisting}


\section{Methods for extracting from arrays and matrices}

\X{extracting data from arrays and matrices}
\XX{matrix}{\fontindexentry{tt}{extract\_column}}\XX{\fontindexentry{tt}{extract\_column}}{from a matrix}
\XX{matrix}{\fontindexentry{tt}{column}}
\begin{lstlisting}
vector column(matrix& M,int& j)
\end{lstlisting}
\afterlistingthing{extracts the $j^\textrm{th}$ column from a matrix and puts it into a vector.}


\begin{lstlisting}
vector extract_row(matrix& M,int& i)
\end{lstlisting}
\afterlistingthing{extracts a row from a matrix and puts it into a vector.
Note that the operation \texttt{M(i)} has the same effect.}

\XX{\fontindexentry{tt}{extract\_diagonal}}{from a matrix}
\begin{lstlisting}
vector extract_diagonal(matrix& M)
\end{lstlisting}
\afterlistingthing{extracts the diagonal elements from a matrix and puts them into
 a vector.}

\X{operator \fontindexentry{tt}{()}}
The function call operator \texttt{()} has been overloaded in two ways to
provide for the extraction of a subvector. 
\XX{\fontindexentry{tt}{vector}}{extracting a subvector}
\X{extracting a subvector}
\XX{matrix} {\fontindexentry{tt}{extract\_row}}\XX{\fontindexentry{tt}{extract\_row}}{from a matrix}
\XX{vector}{function call \fontindexentry{tt}{()} to extract subvector} 
\begin{lstlisting}
  vector(ivector&)
\end{lstlisting}
An \texttt{ivector} object is
used to specify the elements of the vector to be chosen. If 
\texttt{u} and \texttt{v} are \texttt{dvectors}  and \texttt{i} is an \texttt{ivector},
the construction
\begin{lstlisting}
  dvector u = v(i);
\end{lstlisting}
will extract the members of \texttt{v} indexed by \texttt{i} and put them in the
\texttt{dvector u}. The size of \texttt{u}
is equal to the size of \texttt{i}. The \texttt{dvector u} will have
minimum valid index and maximum valid index equal to the minimum
valid index and maximum valid index of \texttt{i}. 
The size of \texttt{i} can be larger than the size of \texttt{v},
in which case some elements of \texttt{v} must be repeated. The elements of
the \texttt{ivector i} must lie in the valid index range for~\texttt{v}. 

If \texttt{v} is a \texttt{dvector} and \texttt{i1} and \texttt{i2} are two integers,
\begin{lstlisting}
  u(i1,i2)
\end{lstlisting}
is a \texttt{dvector}, which is a subvector of \texttt{v} (provided, of
course, that \texttt{i1} and \texttt{i2} are valid indices for \texttt{v}). Subvectors
can appear on both the left and right hand side of an assignment:
\begin{lstlisting}
dvector  u(1,20);
dvector v(1,19);
v = 2.0;   // assigns the value 2 to all elements of v
u(1,19) = v; // assigns the value 2 to elements 1 through 19 of u
\end{lstlisting}
\X{operator \fontindexentry{tt}{++}}
\X{operator\fontindexentry{tt}{-{}-}}
\XX{operator \fontindexentry{tt}{++}}{use with subvectors}
\XX{operator \fontindexentry{tt}{-{}-}}{use with subvectors}
\XX{operator \fontindexentry{tt}{++}}{for \fontindexentry{tt}{dvectors}}
\XX{operator \fontindexentry{tt}{-{}-}}{for \fontindexentry{tt}{dvectors}}
\XX{functions}{\fontindexentry{tt}{++} use with vectors}
\XX{functions}{\fontindexentry{tt}{-{}-} use with vectors}
\XX{functions}{\fontindexentry{tt}{shift} use with vectors}
In the above example, suppose that we wanted to assign the vector \texttt{v}
to elements~2 through~20 of the vector \texttt{u}. To do this, we must first ensure that
they have the same valid index ranges. The operators \texttt{++} and \texttt{-{}-}
increment and decrement the index ranges by~1.
\begin{lstlisting}
dvector  u(1,20);
dvector  w(1,19);
dvector v(1,19);

v = 2.0;   // assigns the value 2 to all elements of v

--u(2,20) = v; // assigns the value 2 to elements 2 through 20 of u

u(2,20) = ++v; // assigns the value 2 to elements 2 through 20 of u
               // probably not what you want
w=v;   // error different index ranges
\end{lstlisting}
It is important to realize that from the point of view of the vector
\texttt{u}, both of the above assignments have the same effect. It will have
elements~2 through~20 set equal to~2. The difference is in the side
effects on the vector \texttt{v}. The operation \texttt{++v}
will increase the minimum and maximum valid index range of the vector 
\texttt{v} by~1. This increase is permanent. On the other hand,
the operation \texttt{-{}-u(2,20)} decrements the valid index bounds of
the {\it subvector} \texttt{u(2,20)}. This is a distinct object from
the vector \texttt{u}, although both objects share a common area for their
components. Thus, the valid index bounds of \texttt{u} are not
affected by this process.  
The use of subvectors, along with increment and decrement
operations, can be used to remove loops from the code. Note that
\XX{subvectors}{using to remove loops from code}
\begin{lstlisting}
dvector x(1,n)
dvector y(1,n)
dvector z(1,n)
for (int i=2;i<=n;i++)
{
  x(i)=y(i-1)*z(i-1);
}
\end{lstlisting}
can be written as
\begin{lstlisting}
dvector x(1,n)
dvector y(1,n)
dvector z(1,n)
x(2,n)=++elem_prod(y(1,n-1),z(1,n-1));  // elem_prod is element-wise 
                                       // multiplication of vectors
\end{lstlisting}

\XX{functions}{\fontindexentry{tt}{shift} use with vectors}
The \texttt{shift} function can be used to set the minimum (and maximum)
valid index for a vector.
\begin{lstlisting}
dvector u(10,100);  // minimum valid index is 10
                    // maximum valid index is 100
u.shift(25);        // minimum valid index is 25
                    // maximum valid index is 115
\end{lstlisting}
In particular, the operators \texttt{-{}-} and \texttt{++}
are just convenient shorthand for using the \texttt{shift}
function to change the minimum valid index by~1.
\begin{lstlisting}
dvector u(10,100);     // minimum valid index is 10
                       // maximum valid index is 100
u.shift(u.indexmin()-1); // minimum valid index is 9
--u;                   // same result as u.shift(u.indexmin()-1) 
u.shift(u.indexmin()+1); // minimum valid index is 11
++u;                   // same result as u.shift(u.indexmin()+1) 
\end{lstlisting}
 
\XX{subobjects}{accessing with\fontindexentry{tt}{sub}}
\XX{\fontindexentry{tt}{sub}}{accessing subobjects}


\section{Accessing subobjects of higher-dimensional arrays}

The  \texttt{()} operator cannot be used to access subobjects of
arrays of dimension~2 or greater, because this operator has
already been defined to do something else. For example,
for a \texttt{dmatrix M}, \texttt{M(1,2)} is an {\it element}
of \texttt{M}.  To access subobjects of higher-dimensional arrays,
use the \texttt{sub} member function. If \texttt{M} is a matrix object,
then \texttt{M.sub(2,6)} is a matrix object with minimum valid
index~2 and maximum valid index~6 (provided, of course, that the
minimum valid index for \texttt{M} is less than or equal to~2 and
the maximum valid index is greater than or equal to~6).
If \texttt{T} is a 3-dimensional object, then
\texttt{T.sub(2,5)} is a 3-dimensional object, provided that
the index bounds are legal. 


\section{Sorting vectors and matrices}
\X{sorting}
\XX{\fontindexentry{tt}{dvector}}{sorting a}
\XX{\fontindexentry{tt}{dmatrix}}{sorting a}
\XX{sorting}{\fontindexentry{tt}{dvector}}
\XX{sorting}{\fontindexentry{tt}{dmatrix}}

While sorting is not strictly a part of methods for calculating the
derivatives of differentiable functions (it is a highly non-differentiable
operation), it is so useful for pre and post-processing data that we
have included some functions for sorting \texttt{dvector} and \texttt{dmatrix}
objects. If \texttt{v} is a \texttt{dvector} the statement
\begin{lstlisting}
  dvector w=sort(v);
\end{lstlisting}
will sort the elements of \texttt{v} in ascending order and 
put them in the \texttt{dvector} object \texttt{w}. The minimum and
maximum valid indices of \texttt{w} will be the same as those of \texttt{v}.
If desired, an index table for the sort can be constructed by passing
an \texttt{ivector} along with the \texttt{dvector}. This index table can be
used to sort other vectors in the same order as the original vector
by using the \texttt{()} operator.
\X{operator \fontindexentry{tt}{()}}
\begin{lstlisting}
dvector u={4,2,1};
dvector v={1,6,5}
ivector ind(1,3);
dvector w=sort(u,ind); // ind will contain an index table for the sort
// Now w=(1,2,4) and ind=(3,2,1)
dvector ww=v(ind);     // This is the use of the ( ) operator for subset
                       // selection. 
// Now ww=(5,6,1)
\end{lstlisting}

The \texttt{sort} function for a \texttt{dmatrix} object sorts the columns of the
\texttt{dmatrix} into ascending order, using the column specified to do the
sorting. For example,
\begin{lstlisting}
  dmatrix MM = sort(M,3);
\end{lstlisting}
will put the sorted matrix into \texttt{MM}, and the third column of
\texttt{MM} will be sorted in ascending order.


\section{Statistical functions}

\begin{lstlisting}
cumd_norm
inv_cumd_norm
cumd_cauchy
inv_cumd_cauchy
\end{lstlisting}
These are the cumulative distribution function and the inverse cumulative distribution 
function for the normal and Cauchy distributions.


\section{The random number generator class}\label{rng}
\XX{cumulative distribution function}{normal}
\XX{cumulative distribution function}{Cauchy}
\XX{\fontindexentry{tt}{random\_number\_generator}}{class}
\XX{function}{\fontindexentry{tt}{random\_number\_generator}}

The random number generator class is used to manage the generation of
random numbers. A random number generator object is created with the
declaration
\begin{lstlisting}
  random_number_generator r(n);
\end{lstlisting}
where \texttt{n} is the seed that initializes the random number generator.
Any number of random number generators may be declared.
This class can be used to manage random number generation with the following functions:
\begin{lstlisting}
randpoisson(lambda,r);  // generate a Poisson with mean lambda
randnegbinomiual(mu,tau,r);  // generate a negative binomial with mean mu
                            // and over-dispersion tau (tau>1)
randn(r);     // generate a normally distributed random number
randu(r);    // generate a uniformly distributed random number
v.fill_randu(r)  // fill a vector v
v.fill_randn(r)  // fill a vector v
v.fill_randpoisson(mu,r)  // fill a vector v with Poisson distributed
                          // random variables with mean mu
v.fill_rand(mu,tau,r)  // fill a vector v with negative binomial distributed
            // random variables with mean mu and over-dispersion var/mu = tau
v.fill_multinomial(r,p)  // fill a vector v 
                      // p is a vector of probabilities
m.fill_randu(r)  // fill a matrix m
m.fill_randn(r)  // fill matrix m
m.fill_randpoisson(lambda,r)  // fill a matrix m
\end{lstlisting}
The incomplete beta function $I_x(a,b)$ is defined by
\begin{equation}
  I_x(a,b)=\frac{1}{B(a,b)}\int_0^x t^{a-1}(1-t)^{b-1}\,\textrm{d}t\qquad (a,b,>0)
\end{equation}
This is also the cumulative distribution function for the beta family
of probability distributions.
The function is named \texttt{betai} and is invoked by
\XX{cumulative distribution function}{\fontindexentry{tt}{beta}}
\XX{functions}{\fontindexentry{tt}{beta} function, incomplete}
\begin{lstlisting}
\\ .....
dvariable p=betai(a,b,x);
\end{lstlisting}


\section{The \texttt{adstring} class operations}
\XX{\fontindexentry{tt}{adstring} class}{operations on}
\X{operations on strings}
\X{string operations}

The \texttt{adstring} class was defined before there was a standardized
\cplus\ string class. It does not contain all the features
that a full string class should have. It is, however, easier to use
in many cases than the standard C string operations.
\begin{lstlisting}
adstring s;
adstring t;
s="first_part";
t="second_part";
adstring u = s + " ___ " + t;
cout << u << endl;
\end{lstlisting}
should print out
\begin{lstlisting}
  first_part ___ second_part
\end{lstlisting}
The operation \texttt{+} concatenates two \texttt{adstring} objects.
It can be used to concatenate C-style strings by first
turning them into \texttt{adstring} objects, as in
\begin{lstlisting}
  adstring u = adstring("xxx") + adstring("yyy");
\end{lstlisting}
One can also append to a string with the \texttt{+=}
operator, as in
\begin{lstlisting}
adstring u = "abc"; 
u += v; 
adstring w = "abc"; 
w += 'f'; 
\end{lstlisting}
which adds the \texttt{adstring} object \texttt{v} to \texttt{u} and the
character `f' to \texttt{w}.
It is also possible to cast an \texttt{adstring} object to
a C-like \texttt{char *} string, as in
\begin{lstlisting}
adstring u = "abc"
char * c = (char*)(u);
\end{lstlisting}
Then it may be used as you would use a C-like string.


\section{Miscellaneous functions}

\begin{lstlisting}
posfun(x,eps,pen)
\end{lstlisting}
\afterlistingthing{The \texttt{posfun} function constrains the argument \texttt{x} to be positive.
For $\texttt{x} > \texttt{eps}$, it is the identity function.}

The current source code for the \texttt{posfun} function appears below:
\XX{functions}{\fontindexentry{tt}{posfun}}
\X{\fontindexentry{tt}{posfun} function}
\begin{lstlisting}
dvariable posfun(const dvariable&x,const double eps,dvariable& pen)
{
  if (x>=eps) {
    return x;
  } else {
    pen+=.01*square(x-eps);
    return eps/(2-x/eps);
  }
}
\end{lstlisting}

\begin{lstlisting}
mfexp(_CONST prevariable& x)
\end{lstlisting}
\afterlistingthing{The \texttt{mfexp} function is the exponential function 
that is modified for large values of its argument, to prevent
floating point overflows.}
The current source code for the \texttt{mfexp} function appears below:
\XX{functions}{\fontindexentry{tt}{mfexp}}
\X{\fontindexentry{tt}{mfexp} function}
\begin{lstlisting}
  dvariable mfexp(_CONST prevariable& x)
  {
    double b=60;
    if (x<b) 
    {
      return exp(x);
    }
    else
    {
      return exp(b)*(1.+2.*(x-b))/(1.+x-b);
    }
  }
\end{lstlisting}

