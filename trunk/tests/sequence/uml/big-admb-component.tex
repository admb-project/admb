% $Id$
% Author: John Sibert
% Copyright (c) 2014 ADMB Foundation

\documentclass[letterpaper,12pt]{article}

%\usepackage[T1]{fontenc}
%\usepackage[utf8x]{inputenc}
\usepackage[plain]{fullpage}
\usepackage[pdftex]{pict2e}
%\usepackage{calc}
\usepackage[nomessages]{fp} % for computing positions in picture boxes
\usepackage[pdftex]{hyperref}
\usepackage{color}
\listfiles


\hypersetup{pdfauthor={John Sibert}}
\hypersetup{pdfsubject={ADMB  Project}}
\hypersetup{pdftitle={ADMB Component Diagram}}



\newcommand\includes{{\tt \#include}s}

   \newlength{\stackY}
% this version of the callout macro operates differently the original
% all (x,y) positions are relative to the outer ("global") box
% 1. height (y position) of callout dot
% 2,3 coordinates of Bezier control point
% 4 height of middle of oval
% 5 text to be displayed 
% 6 x position of dot
\newcommand\callout[6]{%
   {\small
%  stackY = \the\stackY\\
   \put(#6,#1){\circle*{1.6}}
   \FPeval{\qX}{\mainX+3}%- #6}
   \qbezier{(#6,#1)(#2,#3)(\qX,#4)}
%  \put(#2,#3){\circle{1}}
   \FPeval{\oX}{\qX+15.0}
%  \put(\oX,#4){\oval(30,12)}
   \put(\oX,#4){\oval(30,8)}
%  \put(\oX,#4){\circle{1}}
   \FPeval{\tX}{\qX+1.75}
   \FPeval{\tY}{#4-3.5}
   \put(\tX,\tY){\shortstack[l]{#5}}
   }% small
}

\title{ADMB UML Diagrams}
\author{John Sibert}

\begin{document}
\pagenumbering{gobble}
\noindent
textwidth = \the\textwidth , textheigth = \the\textheight\\
paperwitdth = \the\paperwidth\\ % , papersize = \the\papersize \\
\setlength{\unitlength}{0.01\textwidth}
unitlength = \the\unitlength\\
\FPeval{\globY}{\number\textheight / \number\unitlength}\\
\FPeval{\globX}{\number\textwidth / \number\unitlength}
globX = \FPprint{\globX} , globY = \FPprint{\globY}\\
\FPeval{\mainX}{0.65*\globX}
\FPeval{\mainY}{\globY-2}
mainX = \FPprint{\mainX} , mainY = \FPprint{\mainY}\\
\FPeval{\lineY}{1.2*\number\baselineskip/\number\unitlength}
lineY = \lineY\\
\newlength{\height}
\settoheight{\height}{\hbox{ADMB \includes}}
Value = \the\height\\
\settoheight{\height}{\shortstack{ADMB \includes\\ and global methods}}
Value = \the\height\\

%%%%%%%%%%%%%%%%%%%%%%%%%%%%%%%%%%%%%%%%%%%%%

\sffamily
%\FPeval{\result}{clip(5+6)}%

\begin{picture}(\globX,\globY)
%\put(50,102){\makebox(0,1){\large\bf ADMB Application}}
\linethickness{2.0pt}
% \put(0,0){\framebox(100.0,100.0){}} 
  \put(0,0){\framebox(\globX,\globY){}} 
  \thinlines
  \FPeval{\result}{\globY-\lineY}
  %result = \FPprint{\result}
  \FPeval{\result}{\globY - 2*\lineY}
  \put(1,\result){\shortstack[l]{ADMB \includes\ and global methods}}

   \put(1,1){\begin{picture}(\mainX,\mainY)
     % main() box
     \linethickness{1pt}
     \put(0,0){\framebox(\mainX,122){%
     \thinlines
        \put(1,118){{\tt int main(int argc, char * argv[])}}
        \put(0,117){\line(1,0){\mainX}}
        \put(1,113){Set {\tt new} and {\tt exit} routines}
 
        \FPeval{\mpX}{\mainX-2}
        \put(1,58){\begin{picture}(\mpX, 50)
        %  \color{red}
           \put(0,0){\framebox(\mpX,50){%
           \put(1,44){{\tt model\_parameters(sz, argc, argv)}}
           \put(0,43){\line(1,0){\mpX}}
% 
        %  \put(1,15){\circle{2.2}}
           \FPeval{\mdX}{\mpX-2}
           \put(1,12){\begin{picture}(\mdX,30)
               \put(0,0){\framebox(\mdX,30){%
               \put(1,27){{\tt model\_data(argc, argv)}}
               \put(0,26){\line(1,0){\mdX}}
 
               \FPeval{\commX}{\mdX-2}
               \put(1,12){\begin{picture}(\commX,13)
                   \put(0,0){\framebox(\commX,13){%
                   \put(1,11){{\tt ad\_comm(argc, argv)}}
                   \put(0,10){\line(1,0){\commX}}
                   \put(1,5){Initialize {\tt ad\_comm} member variables}
                   \put(1,2){Set file names for I/O operations}
                   }} % ad_comm box
                   \end{picture}}
               \put(1,7){Read data from {\tt .dat} or}
               \put(1,3){user-specified file.}
               }} %model_data box
               \end{picture}}
               \put(1,3){Append user variables}
               \put(1,0){ADMB parameter vector.}
           }} %model_parameters box
           \end{picture}}
 
               \color{red}
          \put(1,10){\begin{picture}(\mpX,42)
               \put(0,0){\framebox(\mpX,42){%
               \put(1,39){{\tt mp.computations(...)}}
               \put(0,38){\line(1,0){\mpX}}
               \put(1,35){Set up minimizer control structure.}
               \color{green}
               \FPeval{\commX}{\mainX-4}
                    \put(1,1){\begin{picture}(\commX,32)
                    \put(0,0){\framebox(\commX,32){%
                    \put(1,29){{\tt Minimizer Block}}
                    \put(0,28){\line(1,0){\commX}}
                    \put(1,25){Allocate independent variable and gradient vectors}
                    }}
               \end{picture}}
             % \put(1,12){No more room in this box.}
             % \put(1,8){See Figure Figure~\ref{fig:comp2}}
           }} % mp.computations box
           \end{picture}}             
     }} % main() box
     \end{picture}}
     %%%

% position callouts relative to global box
\callout{129}{52}{129}{125}{GLOBALS\_SECTION\\User \includes\\ User methods}{2}
\callout{113}{52}{113}{115}{TOP\_OF\_MAIN\_SECTION\\User-specified buffer sizes\\~}{3}
\callout{95}{52}{95}{105}{Process certain\\ commmand-line options\\~}{6}
\callout{76}{52}{76}{95}{Allocate variables in\\\color{blue}DATA\_SECTION\\~\\~}{5}
\callout{71}{52}{73}{82}{Execute user code in\\INITIALIZATION\_SECTION\\~\\~}{4}
\callout{69}{52}{69}{73}{Alllocate parameters in\\\color{blue}PARAMETER\_SECTION\\~\\~}{4}
\callout{57}{52}{57}{62}{Execute user code in\\PRELIMINARY\_CALCS\_\\ SECTION}{3}
\callout{37}{52}{37}{51}{Execute user code in\\\color{blue}PROCEDURE\_SECION\\~\\~}{5}
\end{picture}

%Text after figure.

\end{document}

